\section{测试分析报告}\label{test-analysis-report}

\subsection{文档目的}\label{doc-purpose}

本文件记录ATA系统测试环境、测试策略、用例覆盖、测试结果与缺陷修复情况,作为交付的测试分析报告。

\subsection{读者对象}\label{target-audience}

\begin{itemize}
\tightlist
\item
  团队内部测试负责人
\end{itemize}

\subsection{引用关系}\label{references}

\begin{itemize}
\tightlist
\item
  验收用例:\texttt{../20-需求与验收/21-验收标准与验收用例.md}
\item
  测试用例明细:\texttt{97-测试用例与结果明细.md}
\end{itemize}

\subsection{测试环境}\label{test-environment}

\subsubsection{硬件与系统}\label{hardware-system}

\begin{itemize}
\tightlist
\item
  设备:MacBook Air(Apple M4,32 GB 内存)
\item
  系统:macOS 26.1(Build 25B78),架构 arm64
\item
  时间基准:中心机入库与索引按 UTC 天滚动,前端展示按本地时区渲染
\end{itemize}

\subsubsection{依赖与基础软件版本}\label{software-versions}

\begin{itemize}
\tightlist
\item
  Docker:28.3.3
\item
  Docker Compose:v2.39.2
\item
  后端运行时:Python 3.12.12(uv 环境),uv 0.9.18
\item
  前端运行时:Node.js v25.2.1,pnpm 10.18.2
\item
  OpenSearch:3.0.0(\texttt{https://localhost:9200})
\item
  Neo4j:5.15.0(\texttt{http://localhost:7474},\texttt{bolt://localhost:7687})
\end{itemize}

\subsubsection{测试入口与环境开关}\label{test-entry-and-switches}

\begin{itemize}
\tightlist
\item
  自动化测试入口固定为:在 \texttt{backend/} 目录执行 \texttt{uv\ run\ pytest}
\item
  环境开关固定为:

  \begin{itemize}
  \tightlist
  \item
    \texttt{RUN\_OPENSEARCH\_TESTS=1}:启用 OpenSearch 相关用例
  \item
    \texttt{RUN\_NEO4J\_TESTS=0}:跳过 Neo4j 相关用例
  \item
    \texttt{KEEP\_NEO4J\_TEST\_DATA=0}:测试结束清理 Neo4j 测试数据
  \end{itemize}
\end{itemize}

\subsubsection{本次测试执行时间}\label{test-execution-time}

本报告记录的自动化测试结果执行时间固定为:

\begin{itemize}
\tightlist
\item
  2026-01-15T07:25:57+08:00(UTC+08:00)
\end{itemize}

\subsection{测试策略}\label{test-strategy}

\subsubsection{分层策略}\label{layered-strategy}

ATA系统测试按``自底向上''的分层策略组织,覆盖从数据入库到告警融合、再到图谱入库与溯源分析的关键链路:

\begin{enumerate}
\def\labelenumi{\arabic{enumi}.}
\tightlist
\item
  单元测试(unit):验证核心函数与模块边界,保证算法与数据口径稳定
\item
  集成测试(integration):验证 OpenSearch 与 Neo4j 交互,保证存储、查询与去重流程可运行
\item
  系统测试(system):按业务闭环验证端到端流程,保证``入库→融合→查询''可复现
\item
  验收测试(acceptance):严格按 \texttt{../20-需求与验收/21-验收标准与验收用例.md} 与 \texttt{../20-需求与验收/22-攻击场景与复现剧本.md} 执行,证据点与截图规则按 \texttt{../10-交付与验收/14-录像与截图清单.md} 固化
\end{enumerate}

\subsubsection{测试数据策略}\label{test-data-strategy}

测试数据分为两类:

\begin{itemize}
\tightlist
\item
  自动化测试数据:由 \texttt{backend/tests/opensearch/test\_utils.py} 构造固定结构事件与告警,确保字段口径与边界条件覆盖
\item
  验收复现数据:由 \texttt{../20-需求与验收/22-攻击场景与复现剧本.md} 产出多源证据,确保现场演示稳定输出
\end{itemize}

\subsubsection{隔离与可复现策略}\label{isolation-reproducibility}

\begin{itemize}
\tightlist
\item
  OpenSearch:每条用例开始前清理近 7 日索引(见 \texttt{backend/tests/opensearch/conftest.py}),保证跨次运行无污染
\item
  Neo4j:当 \texttt{KEEP\_NEO4J\_TEST\_DATA=0} 时,测试用例结束后清理写入的测试节点与边
\item
  时间口径:测试用例与索引日期统一使用 UTC,避免跨时区导致的索引名漂移
\end{itemize}

\subsection{覆盖范围}\label{coverage-scope}

\subsubsection{自动化测试覆盖模块}\label{automated-test-coverage}

自动化测试集中覆盖中心机后端核心模块:

\begin{itemize}
\tightlist
\item
  OpenSearch:索引管理、入库去重、告警融合、增量处理(\texttt{backend/tests/opensearch/})
\item
  Neo4j 图谱:ECS→Graph 入库、节点与边模型、图查询工具(\texttt{backend/tests/graph/})
\item
  溯源分析:FSA 构建、候选子图与 LLM 选择回退路径(\texttt{backend/tests/services/analyze/})
\end{itemize}

\subsubsection{与验收需求的对应关系}\label{acceptance-mapping}

自动化测试与验收用例的对应关系以``闭环能力''为主:

\begin{itemize}
\tightlist
\item
  覆盖 AC-02、AC-03(OpenSearch 入库与告警融合)
\item
  覆盖 AC-04(图谱查询结果可生成)
\item
  覆盖 AC-05、AC-06、AC-07 的后端关键组件(任务状态机、结果结构与写回字段口径)
\end{itemize}

多节点轮询、前端交互与导出材料的完整展示在验收用例中按固定步骤执行,并以截图与录像作为证据。

\subsection{测试结果}\label{test-results}

\subsubsection{自动化测试结果汇总}\label{automated-test-summary}

{\small
\def\LTcaptype{none} % do not increment counter
\begin{longtable}[]{@{}p{0.20\textwidth}p{0.45\textwidth}p{0.15\textwidth}p{0.10\textwidth}@{}}
\toprule\noalign{}
测试集 & 运行命令 & 结果 & 用时 \\
\midrule\noalign{}
\endhead
\bottomrule\noalign{}
\endlastfoot
后端快速回归(跳过 Neo4j) & {\tiny\texttt{cd\ backend\ \&\&\ RUN\_NEO4J\_TESTS=0\ uv\ run\ pytest\ -q}} & 121 passed,62 skipped & 0.20s \\
后端全量测试(OpenSearch + Neo4j) & {\tiny\texttt{cd\ backend\ \&\&\ RUN\_OPENSEARCH\_TESTS=1\ RUN\_NEO4J\_TESTS=1\ KEEP\_NEO4J\_TEST\_DATA=0\ uv\ run\ pytest\ -q}} & 182 passed,1 skipped & 196.58s \\
\end{longtable}
}

跳过用例明细:

{\small
\begin{itemize}
\tightlist
\item
  \texttt{tests/opensearch/test\_system\_opensearch.py:362}:需要 mock OpenSearch 连接(用例用于断连场景验证,当前以跳过方式记录)
\end{itemize}
}

\subsubsection{重要结论}\label{key-conclusions}

\begin{itemize}
\tightlist
\item
  OpenSearch、Neo4j 相关测试全量通过,核心闭环链路可复现
\item
  回退路径与异常分支通过单元测试覆盖,保证在 LLM 不可用或输出不合规时仍能产出结果
\end{itemize}

\subsection{缺陷与修复记录}\label{defect-fixes}

本节记录本次文档体系重构期间,测试暴露出的关键问题与对应修复:

\subsubsection{缺陷-001 OpenSearch 索引日期口径不一致导致测试污染}\label{defect-001-opensearch-index-date}

\begin{itemize}
\tightlist
\item
  现象:同一套用例在 UTC 与本地时区交界时生成不同日期索引,导致索引清理失效与跨次污染
\item
  修复:统一测试与索引命名使用 UTC,并在每条用例开始前清理近 7 日索引
\item
  影响:OpenSearch 相关用例稳定通过,跨次运行结果一致
\end{itemize}

\subsubsection{缺陷-002 增量状态字段缺少明确映射导致 term 查询失效}\label{defect-002-incremental-state-mapping}

\begin{itemize}
\tightlist
\item
  现象:\texttt{custom.finding.detector\_id} 未在 mapping 中显式声明,term 查询无法稳定命中
\item
  修复:在 OpenSearch mapping 中补齐 \texttt{custom.finding.detector\_id} 为 \texttt{keyword}
\item
  影响:增量处理相关辅助函数可正确返回上次处理时间戳
\end{itemize}

\subsubsection{缺陷-003 OpenSearch 增量测试用例误用 fixture 参数}\label{defect-003-fixture-misuse}

\begin{itemize}
\tightlist
\item
  现象:测试用例把 \texttt{initialized\_indices} 当作 client 传入,导致 \texttt{\_get\_last\_processed\_timestamp} 永远返回空
\item
  修复:统一使用 \texttt{opensearch\_client} 作为调用参数,并保留 \texttt{initialized\_indices} 负责索引初始化
\item
  影响:增量相关用例稳定通过
\end{itemize}

\subsection{结论}\label{conclusion}

\begin{enumerate}
\def\labelenumi{\arabic{enumi}.}
\tightlist
\item
  自动化测试全量通过,OpenSearch 与 Neo4j 的核心链路具备可复现性与稳定性
\item
  验收测试按固定剧本与验收用例执行,证据点口径在文档中固化,满足交付对``可演示、可解释、可复现''的要求
\end{enumerate}
