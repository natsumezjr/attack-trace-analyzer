% Manual polished content for: 测试分析报告(基于仓库现状 + 可复现证据)
% Note: Keep this file under manual/ to avoid being overwritten by generators.

\section{测试分析报告}

\begin{SideBar}
\textbf{文档定位:}本报告用于说明ATA系统的测试策略、测试范围、证据组织方式与实际测试结果(含缺陷与限制)。\\
\textbf{内容依据:}以仓库现有实现与可复现命令输出为唯一事实来源;不编造"代码中不存在"或"未执行过"的结论。\\
\textbf{证据形式:}本报告\Emph{不提供}截图/录屏;所有证据以\Emph{可复现命令输出}(.txt/.json/.log)与\Emph{自动化测试日志}(pytest 输出、覆盖率 htmlcov)为准。\\
\textbf{参考资料(仓库内权威文档):}
\begin{itemize}
  \item 后端测试说明:\texttt{backend/tests/README.md}
  \item 验收标准与验收用例:\texttt{docs/20-需求与验收/21-验收标准与验收用例.md}
  \item 攻击场景与复现剧本:\texttt{docs/20-需求与验收/22-攻击场景与复现剧本.md}
  \item 本报告的证据日志目录:\texttt{reports/测试分析报告/resources/}
\end{itemize}
\end{SideBar}

\begin{KeyBox}
\textbf{一页总结(以“可复核证据”为准):}
\begin{itemize}
  \item 已生成并归档多类证据:pytest 自动化测试日志、系统脚本输出(Neo4j 导入/ KillChain fallback)、API 烟雾测试(Health / Clients / Events / Findings / Graph / Tasks)。
  \item 自动化测试结果(2026-01-16):
    \begin{itemize}
      \item 单元测试(\texttt{-m unit}):\texttt{1 passed, 42 skipped, 183 deselected}(通过)。
      \item 集成测试(\texttt{tests/integration/}):\texttt{50 passed, 6 failed}(存在失败用例,详见第 6 节)。
      \item 端到端系统测试(\texttt{tests/e2e/}):因夹具作用域问题出现 \texttt{11 errors}(未达到“可直接运行”状态)。
    \end{itemize}
  \item 系统级脚本验证:
    \begin{itemize}
      \item \texttt{scripts/import\_test\_data.py}:成功导入 \texttt{142} 条事件至 Neo4j(\texttt{543} 节点 / \texttt{299} 边 / \texttt{25} 告警边)。
      \item \texttt{scripts/test\_killchain.py}(不含 LLM):成功生成 \texttt{1} 条 KillChain(\texttt{6} 个阶段段落,\texttt{5} 条选中路径)。
    \end{itemize}
  \item 重要限制:KillChain 的“LLM 增强解释”测试脚本 \texttt{scripts/test\_analyze.py} 强制要求真实 LLM(如 DeepSeek API Key);在未配置 API Key 的环境下无法通过(本报告如实记录为\Emph{受限项})。
\end{itemize}
\end{KeyBox}

\begin{figure}[htbp]
  \centering
  \includegraphics[width=0.96\textwidth]{figures/evidence_flow.pdf}
  \caption{测试证据组织方式(以可复现命令输出为准,不使用截图/录屏)}
  \label{fig:test-evidence-flow}
\end{figure}

\subsection{测试范围与目标}

\subsubsection{测试范围}

本报告聚焦中心机侧(\texttt{backend/})的可复核能力,覆盖:
\begin{itemize}
  \item \textbf{基础可用性:}服务健康检查、核心 API 路由可达;
  \item \textbf{数据面:}Telemetry(事件)与 Findings(告警)查询接口可用,字段结构符合 ECS 子集;
  \item \textbf{分析面:}图谱查询接口可用;溯源任务(Trace Task)结果可读取;
  \item \textbf{自动化测试:}pytest 单元/集成/系统级测试的可运行性与结果(含失败项);
  \item \textbf{系统脚本:}基于 fixtures 的 Neo4j 导入、KillChain 流水线(fallback)验证。
\end{itemize}

不在本报告中\Emph{强行宣称已完成}的内容(若需完整验收,建议按附录复现剧本现场演示):
\begin{itemize}
  \item 客户机侧 Falco/Filebeat/Suricata 在 \textbf{真实 Linux 节点}上的长期稳定采集;
  \item \textbf{真实 ≥5 台物理/虚拟主机}靶场环境下的全链路压测与录像证据(ATA系统不提供录屏)。
\end{itemize}

\subsubsection{测试目标}

\begin{itemize}
  \item \textbf{可复现:}所有“通过/失败/受限”均可由命令复现,不依赖截图说明;
  \item \textbf{可追溯:}关键分析结果能回链到证据事件(如 \texttt{event.id}、\texttt{custom.evidence.event\_ids}、\texttt{analysis.task\_id});
  \item \textbf{不夸大:}测试失败的用例与原因明确记录,不以“理论上可以”替代“实际通过”。
\end{itemize}

\subsection{测试环境与依赖}

本次测试在 macOS + Docker 的本地开发环境完成(中心机单机验证),核心依赖如下:

\begin{itemize}
  \item Docker Desktop + Docker Compose v2(用于 OpenSearch/Neo4j);
  \item Python 依赖管理:\texttt{uv}(在 \texttt{backend/.venv} 创建 CPython 3.12 环境);
  \item 后端:FastAPI(本地监听 \texttt{8001} 端口);
  \item 存储:OpenSearch(\texttt{https://localhost:9200},自签名证书)、Neo4j(\texttt{bolt://localhost:7687})。
\end{itemize}

\textbf{环境快照与证据:}详见 \texttt{reports/测试分析报告/resources/env-snapshot-*.txt}。

\subsection{测试方法与证据组织}

\subsubsection{自动化测试(pytest)}

后端测试位于 \texttt{backend/tests/},按层级划分为:
\begin{itemize}
  \item \textbf{unit:}不依赖外部服务(适合快速回归);
  \item \textbf{integration:}依赖 OpenSearch/Neo4j 等外部服务;
  \item \textbf{system/e2e:}以更贴近业务流程的方式进行黑盒验证。
\end{itemize}

\begin{figure}[htbp]
  \centering
  \includegraphics[width=0.86\textwidth]{figures/test_pyramid.pdf}
  \caption{后端自动化测试分层(unit/integration/system)与运行开关}
  \label{fig:test-pyramid}
\end{figure}

项目采用环境变量控制外部依赖测试是否运行(参见 \texttt{backend/tests/conftest.py} 与 \texttt{backend/tests/README.md}):
\begin{itemize}
  \item \texttt{RUN\_OPENSEARCH\_TESTS=1}:启用 OpenSearch 相关测试(默认跳过);
  \item \texttt{RUN\_NEO4J\_TESTS=0}:跳过 Neo4j 相关测试(默认启用);
  \item \texttt{RUN\_LLM\_TESTS=1}:启用需要 LLM 的测试(默认跳过)。
\end{itemize}

\subsubsection{系统脚本(fixtures 驱动)}

除 pytest 外,仓库还提供了可直接运行的脚本用于快速验证关键链路:
\begin{itemize}
  \item \texttt{backend/scripts/import\_test\_data.py}:将 \texttt{tests/fixtures/graph/testExample.json} 导入 Neo4j,验证入图能力;
  \item \texttt{backend/scripts/test\_killchain.py}:在不依赖 LLM 的情况下运行 KillChain 流水线(fallback),验证攻击链构建与回写;
  \item \texttt{backend/scripts/test\_analyze.py}:强制依赖真实 LLM(受限项;未配置 API Key 时不可通过)。
\end{itemize}

\subsubsection{API 烟雾测试(无截图)}

为验证“服务可用 + 结果可读取”,本次测试补充了最小 API 烟雾测试(curl 输出即证据):
\begin{itemize}
  \item \texttt{/health}:服务健康;
  \item \texttt{/api/v1/clients}:客户机注册/轮询状态;
  \item \texttt{/api/v1/events/search}:事件检索;
  \item \texttt{/api/v1/findings/search}:告警检索(raw/canonical);
  \item \texttt{/api/v1/graph/query}:图谱查询(告警边);
  \item \texttt{/api/v1/analysis/tasks}:溯源任务结果读取。
\end{itemize}

\subsection{测试执行与结果(以证据文件为准)}

为便于复核,本次测试的所有命令输出均已落盘到 \texttt{reports/测试分析报告/resources/}。下表给出本次测试执行清单:

{\def\LTcaptype{none}
\begin{longtable}[]{@{}p{0.16\textwidth}p{0.52\textwidth}p{0.12\textwidth}p{0.16\textwidth}@{}}
\toprule\noalign{}
类别 & 命令(简写) & 结果 & 证据文件 \\
\midrule\noalign{}
\endhead
\bottomrule\noalign{}
\endlastfoot
Unit & \texttt{cd backend \&\& uv run pytest -m unit -q} & 通过 & \texttt{pytest-unit-*.txt} \\
Integration & \texttt{cd backend \&\& RUN\_OPENSEARCH\_TESTS=1 uv run pytest tests/integration -q} & 部分失败 & \texttt{pytest-integration-only-*.txt} \\
E2E/System & \texttt{cd backend \&\& RUN\_OPENSEARCH\_TESTS=1 uv run pytest tests/e2e -q} & 夹具错误 & \texttt{pytest-e2e-*.txt} \\
Neo4j 导入 & \texttt{cd backend \&\& uv run python scripts/import\_test\_data.py} & 通过 & \texttt{system-import-test-data-*.txt} \\
KillChain (fallback) & \texttt{cd backend \&\& uv run python scripts/test\_killchain.py} & 通过 & \texttt{system-killchain-fallback-*.txt} \\
API Smoke & \texttt{curl /health,/clients,/events,/findings,/graph,/tasks} & 通过 & \texttt{api-*-*.txt/.json} \\
\end{longtable}
}

\subsection{验收项覆盖情况(实事求是)}

本节将验收用例(见附录)与“本次已执行的证据”对齐。为避免“报告写了但代码/环境并未验证”,本表区分\Emph{已执行}与\Emph{仅提供复现步骤}。

符号定义(避免使用截图/特殊符号字体):
\begin{itemize}
  \item \textbf{PASS(已执行):}已在本机环境运行并产出证据文件;
  \item \textbf{PARTIAL(部分):}接口可达/能力存在,但场景要素(多节点真实采集、LLM key 等)不完整或未全量覆盖;
  \item \textbf{NOT RUN(未执行):}本次未执行,仅在附录给出复现步骤。
\end{itemize}

{\def\LTcaptype{none}
\begin{longtable}[]{@{}p{0.10\textwidth}p{0.40\textwidth}p{0.12\textwidth}p{0.30\textwidth}@{}}
\toprule\noalign{}
验收项 & 目标(简述) & 本次执行 & 证据/备注 \\
\midrule\noalign{}
\endhead
\bottomrule\noalign{}
\endlastfoot
AC-01 & 多节点在线与轮询拉取 & PARTIAL & \texttt{/api/v1/clients} 返回 \texttt{total=5};是否为真实 5 台主机取决于靶场部署。 \\
AC-02 & 多源事件入库与字段规范 & PARTIAL & \texttt{/api/v1/events/search} 有数据(含 \texttt{event.dataset});本次样本含模拟数据,未代表真实 Linux 采集全覆盖。 \\
AC-03 & 告警产生与融合生成 Canonical & PASS & \texttt{/api/v1/findings/search} 同时存在 raw/canonical;canonical 含 \texttt{custom.evidence.event\_ids}。 \\
AC-04 & 图谱查询与可视化基础数据 & PASS & \texttt{/api/v1/graph/query} 可返回 nodes/edges;含 \texttt{analysis.task\_id} 等属性。 \\
AC-05 & 创建溯源任务与状态轮询 & PARTIAL & 本次读取到 \texttt{/api/v1/analysis/tasks} 中已完成任务;“从前端创建任务”未在本报告中强行宣称已执行。 \\
AC-06 & 溯源结果写回与可解释展示 & PARTIAL & 已观察到图边属性中包含 \texttt{analysis.*} 写回字段;解释内容是否来自 LLM 取决于环境配置。 \\
AC-07 & APT 相似度匹配输出 & PARTIAL & \texttt{/api/v1/analysis/tasks} 返回相似 APT 列表;需 CTI 数据与(可选)LLM 配置配合。 \\
AC-08 & 报告导出可复现 & NOT RUN & 本次未验证前端导出;建议现场以导出的 Markdown/JSON 文件作为文本证据。 \\
\end{longtable}
}

\subsection{失败用例与原因分析(不改代码,记录事实)}

本次在启用外部依赖(OpenSearch/Neo4j)后运行集成测试,出现部分失败用例;端到端系统测试出现夹具作用域错误。根据 pytest 输出,主要问题类别如下:

\subsubsection{集成测试失败(\texttt{tests/integration/})}

见证据:\texttt{pytest-integration-only-*.txt}。失败项集中在 OpenSearch 增量处理与分析辅助函数:
\begin{itemize}
  \item \textbf{时间戳状态:}部分用例断言“last\_processed\_timestamp 非空”,但实际返回 \texttt{None};
  \item \textbf{去重逻辑:}部分用例期望过滤结果为空,但实际返回非空(可能与索引状态/时间窗边界有关);
  \item \textbf{接口变更:}用例尝试导入 \texttt{\_should\_trigger\_scan},但目标模块中已不存在该函数(典型的测试/实现不同步)。
\end{itemize}

以上失败为\Emph{测试结果事实};由于ATA系统代码不再改动,本报告不在此处给出“已修复”结论,仅作为风险与改进建议记录。

\subsubsection{E2E/System 测试错误(\texttt{tests/e2e/})}

见证据:\texttt{pytest-e2e-*.txt}。错误核心为夹具不可见:
\begin{itemize}
  \item \texttt{tests/e2e/test\_opensearch\_system.py} 依赖夹具 \texttt{initialized\_indices};
  \item 该夹具定义在 \texttt{tests/integration/conftest.py},pytest 默认作用域不会向 \texttt{tests/e2e/} 目录暴露;
  \item 因此出现 \texttt{fixture 'initialized\_indices' not found} 的系统性错误。
\end{itemize}

\subsubsection{LLM 相关脚本受限}

见证据:\texttt{system-killchain-analyze-*.txt}。脚本 \texttt{scripts/test\_analyze.py} 强制要求真实 LLM(DeepSeek API Key),未配置时会直接抛出错误并退出。该项在本次测试中按“受限项”记录。

\subsection{结论}

\begin{Takeaway}
\textbf{结论:}
\begin{itemize}
  \item ATA系统在中心机本地环境下完成了多类可复核验证:API 可用、数据可查、告警可融合、图谱可查询、溯源任务结果可读取、KillChain fallback 可运行。
  \item 自动化测试体系已具备雏形(unit/integration/system 分层),但当前版本存在部分失败与夹具作用域错误;如需提升“可交付质量”,应优先修复测试与实现不同步、以及 e2e 夹具组织问题。
  \item LLM 增强解释能力在工程上可选:未配置 API Key 时仍可使用 fallback 生成基础连通的攻击链;如需更强解释性,可在有 API Key 的环境下启用。
\end{itemize}
\end{Takeaway}
