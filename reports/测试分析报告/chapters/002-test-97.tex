\section{测试用例与结果明细}\label{test-cases-results}

\subsection{文档目的}\label{doc-purpose-97}

本文件列出测试用例的执行步骤、期望结果、实际结果与证据位置,便于复核与复现。

\subsection{读者对象}\label{target-audience-97}

\begin{itemize}
\tightlist
\item
  团队内部测试负责人
\item
  负责演示的同学
\end{itemize}

\subsection{引用关系}\label{references-97}

\begin{itemize}
\tightlist
\item
  验收用例:\texttt{../20-需求与验收/21-验收标准与验收用例.md}
\end{itemize}

\subsection{用例列表}\label{test-case-list}

ATA系统测试用例分为两类:

\begin{itemize}
\tightlist
\item
  自动化测试用例:以 \texttt{pytest} 为入口,结果以命令输出为准
\item
  验收测试用例:以 AC-01 到 AC-08 为入口,结果以截图与录像证据为准
\end{itemize}

用例清单固定如下:

{\def\LTcaptype{none} % do not increment counter
\begin{longtable}[]{@{}lllll@{}}
\toprule\noalign{}
用例编号 & 用例名称 & 类型 & 执行入口 & 结果 \\
\midrule\noalign{}
\endhead
\bottomrule\noalign{}
\endlastfoot
TC-AUTO-01 & 后端快速回归测试 & 自动化 & \texttt{cd\ backend\ \&\&\ RUN\_NEO4J\_TESTS=0\ uv\ run\ pytest\ -q} & 通过 \\
TC-AUTO-02 & 后端全量集成测试 & 自动化 & \texttt{cd\ backend\ \&\&\ RUN\_OPENSEARCH\_TESTS=1\ RUN\_NEO4J\_TESTS=1\ KEEP\_NEO4J\_TEST\_DATA=0\ uv\ run\ pytest\ -q} & 通过 \\
AC-01 & 多节点在线与轮询拉取 & 验收 & \texttt{../20-需求与验收/21-验收标准与验收用例.md} & 待现场验收 \\
AC-02 & 多源事件入库与字段规范 & 验收 & \texttt{../20-需求与验收/21-验收标准与验收用例.md} & 待现场验收 \\
AC-03 & 告警产生与融合生成 Canonical & 验收 & \texttt{../20-需求与验收/21-验收标准与验收用例.md} & 待现场验收 \\
AC-04 & 图谱可视化查询 & 验收 & \texttt{../20-需求与验收/21-验收标准与验收用例.md} & 待现场验收 \\
AC-05 & 创建溯源任务与状态轮询 & 验收 & \texttt{../20-需求与验收/21-验收标准与验收用例.md} & 待现场验收 \\
AC-06 & 溯源结果写回与可解释展示 & 验收 & \texttt{../20-需求与验收/21-验收标准与验收用例.md} & 待现场验收 \\
AC-07 & APT 相似度匹配输出 & 验收 & \texttt{../20-需求与验收/21-验收标准与验收用例.md} & 待现场验收 \\
AC-08 & 报告导出可复现 & 验收 & \texttt{../20-需求与验收/21-验收标准与验收用例.md} & 待现场验收 \\
\end{longtable}
}

\subsection{用例明细}\label{test-case-details}

\subsubsection{TC-AUTO-01 后端快速回归测试}\label{tc-auto-01-backend-regression}

\textbf{目的}

在不启动 Neo4j 的条件下验证后端核心逻辑可运行,保证每次提交的快速回归结果稳定。

\textbf{执行步骤}

\begin{enumerate}
\def\labelenumi{\arabic{enumi}.}
\tightlist
\item
  进入后端目录:\texttt{cd\ backend}
\item
  执行测试:\texttt{RUN\_NEO4J\_TESTS=0\ uv\ run\ pytest\ -q}
\end{enumerate}

\textbf{期望结果}

\begin{itemize}
\tightlist
\item
  测试完成,退出码为 0
\item
  输出包含:\texttt{121\ passed,\ 62\ skipped}
\end{itemize}

\textbf{实际结果}

\begin{itemize}
\tightlist
\item
  通过:\texttt{121\ passed,\ 62\ skipped\ in\ 0.20s}
\end{itemize}

\textbf{证据位置}

\begin{itemize}
\tightlist
\item
  终端输出截图:见本文件第 3 节命名规则
\end{itemize}

\subsubsection{TC-AUTO-02 后端全量集成测试}\label{tc-auto-02-backend-integration}

\textbf{目的}

在 OpenSearch 与 Neo4j 均运行的条件下验证后端完整链路可运行,覆盖存储、查询、去重、图谱入库与溯源分析关键路径。

\textbf{执行步骤}

\begin{enumerate}
\def\labelenumi{\arabic{enumi}.}
\tightlist
\item
  启动基础设施(OpenSearch + Neo4j):\texttt{cd\ backend\ \&\&\ docker-compose\ up\ -d}
\item
  执行测试:\texttt{cd\ backend\ \&\&\ RUN\_OPENSEARCH\_TESTS=1\ RUN\_NEO4J\_TESTS=1\ KEEP\_NEO4J\_TEST\_DATA=0\ uv\ run\ pytest\ -q}
\end{enumerate}

\textbf{期望结果}

\begin{itemize}
\tightlist
\item
  测试完成,退出码为 0
\item
  输出包含:\texttt{182\ passed,\ 1\ skipped}
\end{itemize}

\textbf{实际结果}

\begin{itemize}
\tightlist
\item
  通过:\texttt{182\ passed,\ 1\ skipped,\ 36\ warnings\ in\ 196.58s}
\item
  跳过用例:\texttt{tests/opensearch/test\_system\_opensearch.py:362}(需要 mock OpenSearch 连接)
\end{itemize}

\textbf{证据位置}

\begin{itemize}
\tightlist
\item
  终端输出截图:见本文件第 3 节命名规则
\end{itemize}

\subsubsection{AC-01 到 AC-08 现场验收用例}\label{ac-01-to-ac-08-acceptance}

\textbf{统一执行入口}

\begin{enumerate}
\def\labelenumi{\arabic{enumi}.}
\tightlist
\item
  固定复现剧本:\texttt{../20-需求与验收/22-攻击场景与复现剧本.md}
\item
  固定验收步骤:\texttt{../20-需求与验收/21-验收标准与验收用例.md}
\item
  固定证据规则:\texttt{../10-交付与验收/14-录像与截图清单.md}、\texttt{../90-运维与靶场/94-验证清单.md}
\end{enumerate}

\textbf{期望结果}

\begin{itemize}
\tightlist
\item
  每条验收用例均能定位到 OpenSearch 事件证据 \texttt{event.id}
\item
  图谱与溯源任务输出字段满足规范文档要求
\end{itemize}

\textbf{证据位置}

\begin{itemize}
\tightlist
\item
  截图与录像存放目录固定为提交包内:\texttt{09-截屏录像/}
\end{itemize}

\subsection{证据索引}\label{evidence-index}

\subsubsection{截图与录像命名规则}\label{screenshot-video-naming}

文件命名与存放规则统一以 \texttt{../10-交付与验收/14-录像与截图清单.md} 为准:

\begin{itemize}
\tightlist
\item
  截图:\texttt{YYYYMMDD-序号-证据主题.png}
\item
  录像:\texttt{YYYYMMDD-项目演示.mp4}
\end{itemize}

\subsubsection{自动化测试证据清单}\label{automated-test-evidence}

自动化测试证据固定为两张截图:

\begin{enumerate}
\def\labelenumi{\arabic{enumi}.}
\tightlist
\item
  \texttt{YYYYMMDD-01-pytest-后端快速回归.png}(对应 TC-AUTO-01)
\item
  \texttt{YYYYMMDD-02-pytest-后端全量集成.png}(对应 TC-AUTO-02)
\end{enumerate}
