\section{验证清单}\label{verification-checklist}

本文件把靶场验收拆成 4 个必须通过的检查段落:网络、采集、中心机入库、证据点。每一项都给出固定命令与固定通过标准。

\subsection{固定变量}\label{fixed-vars-94}

本文件命令默认在靶场宿主机执行,并使用以下固定变量:

\begin{verbatim}
export BASE=/home/ubuntu/attack-trace-analyzer
export REPO="$BASE"/repo/attack-trace-analyzer
\end{verbatim}

\subsection{网络连通性(必须)}\label{network-connectivity}

\subsubsection{中心机端口(center-01)}\label{center-ports-check}

在宿主机执行:

\begin{verbatim}
ss -lntup | egrep ':(9200|9600|7474|7687|8001|3000)\\b' || true
\end{verbatim}

通过标准(固定):

\begin{itemize}
\tightlist
\item
  端口 \texttt{9200}、\texttt{9600}、\texttt{7474}、\texttt{7687}、\texttt{8001}、\texttt{3000} 均处于监听状态。
\end{itemize}

\subsubsection{客户机端口(client-01..04)}\label{client-ports-check}

在宿主机执行:

\begin{verbatim}
ss -lntup | egrep ':(18881|18882|18883|18884)\\b' || true
\end{verbatim}

通过标准(固定):

\begin{itemize}
\tightlist
\item
  端口 \texttt{18881}、\texttt{18882}、\texttt{18883}、\texttt{18884} 均处于监听状态。
\end{itemize}

\subsubsection{后端健康检查}\label{backend-health-check}

\begin{verbatim}
curl -sS http://localhost:8001/health
\end{verbatim}

通过标准(固定):

\begin{verbatim}
{"status":"ok"}
\end{verbatim}

\subsubsection{OpenSearch 健康检查}\label{opensearch-health-check}

\begin{verbatim}
curl -k -sS -u admin:OpenSearch@2024!Dev https://localhost:9200/_cluster/health
\end{verbatim}

通过标准(固定):

\begin{itemize}
\tightlist
\item
  输出包含字段 \texttt{status},其值为 \texttt{yellow} 或 \texttt{green}。
\end{itemize}

\subsubsection{C2 健康检查}\label{c2-health-check}

按 \texttt{93-C2部署与证据点.md} 的第 7 节执行 DNS、HTTP 与抓包验证,通过即为 C2 可用。

\subsection{采集链路(必须)}\label{collection-chain}

\subsubsection{客户机拉取接口(四实例)}\label{client-api-check}

在宿主机执行:

\begin{verbatim}
for p in 18881 18882 18883 18884; do
  echo "== client api :$p =="
  curl -s "http://<CENTER_IP>:$p/filebeat" | python3 -c "import sys,json; print('filebeat total=', json.load(sys.stdin).get('total'))"
  curl -s "http://<CENTER_IP>:$p/falco"    | python3 -c "import sys,json; print('falco total=',   json.load(sys.stdin).get('total'))"
  curl -s "http://<CENTER_IP>:$p/suricata" | python3 -c "import sys,json; print('suricata total=',json.load(sys.stdin).get('total'))"
done
\end{verbatim}

通过标准(固定):

\begin{itemize}
\tightlist
\item
  对 4 个实例的 3 个接口均返回 JSON;
\item
  每个返回体包含 \texttt{total} 字段;
\item
  执行一次证据动作后,至少一个实例的 \texttt{suricata\ total} 大于 0。
\end{itemize}

\subsection{中心机入库与查询(必须)}\label{data-ingestion-query}

\subsubsection{客户机注册表可见}\label{client-registration-check}

\begin{verbatim}
curl -sS http://localhost:8001/api/v1/clients | grep -n 'client-0' || true
\end{verbatim}

通过标准(固定):

\begin{itemize}
\tightlist
\item
  返回 \texttt{status="ok"};
\item
  返回结果中包含 \texttt{client-01}、\texttt{client-02}、\texttt{client-03}、\texttt{client-04};
\item
  对每个客户端,\texttt{poll.status} 最终变为 \texttt{ok} 或 \texttt{partial}。
\end{itemize}

\subsubsection{Telemetry 入库可查询}\label{telemetry-query-check}

执行一次证据动作(见第 4 节)后等待 10 秒,再执行:

\begin{verbatim}
curl -sS -X POST http://localhost:8001/api/v1/events/search \
  -H "Content-Type: application/json" \
  -d '{"size": 10, "offset": 0, "sort_order": "desc"}' | grep -n '\"total\"'
\end{verbatim}

通过标准(固定):

\begin{itemize}
\tightlist
\item
  返回 \texttt{status="ok"};
\item
  \texttt{total} 大于 0。
\end{itemize}

\subsubsection{Findings 入库可查询}\label{findings-query-check}

\begin{verbatim}
curl -sS -X POST http://localhost:8001/api/v1/findings/search \
  -H "Content-Type: application/json" \
  -d '{"stage":"raw","size": 10, "offset": 0, "sort_order": "desc"}' | grep -n '\"total\"'
\end{verbatim}

通过标准(固定):

\begin{itemize}
\tightlist
\item
  返回 \texttt{status="ok"};
\item
  \texttt{total} 大于 0。
\end{itemize}

\subsection{证据点(必须)}\label{evidence-verification}

证据点固定由两条命令产生(见 \texttt{93-C2部署与证据点.md}):

\begin{enumerate}
\def\labelenumi{\arabic{enumi}.}
\tightlist
\item
  DNS:\texttt{dig\ @\textless{}C2\_DNS\_IP\textgreater{}\ \textless{}C2\_DOMAIN\textgreater{}\ +short}
\item
  HTTP:\texttt{curl\ -s\ http://\textless{}C2\_HTTP\_IP\textgreater{}/health\ \&\&\ echo}
\end{enumerate}

通过标准(固定):

\begin{itemize}
\tightlist
\item
  \texttt{"\$BASE/run/client-01/data/eve.json"} 文件存在且文件大小大于 0;
\item
  \texttt{http://\textless{}CENTER\_IP\textgreater{}:18881/suricata} 返回 \texttt{total\textgreater{}0};
\item
  中心机入库后,\texttt{/api/v1/events/search} 的 \texttt{total} 增加。
\end{itemize}

\subsection{截图与录像证据(必须)}\label{screenshot-recording-evidence}

ATA系统固定产出以下截图/输出,用于 PPT 与录像:

\begin{itemize}
\tightlist
\item
  \texttt{ss\ -lntup} 的端口监听输出
\item
  DNS 解析输出:\texttt{\textless{}C2\_DOMAIN\textgreater{}\ -\textgreater{}\ \textless{}C2\_HTTP\_IP\textgreater{}}
\item
  HTTP 访问输出:\texttt{/health\ -\textgreater{}\ ok}
\item
  \texttt{GET\ /suricata} 的 \texttt{total\textgreater{}0}
\item
  \texttt{POST\ /api/v1/events/search} 返回 \texttt{status="ok"} 且 \texttt{total\textgreater{}0}
\end{itemize}

\begin{quote}
\textbf{注意}:后端端口固定为 \texttt{8001},避免与服务器上其他服务(如Django)的 \texttt{8000} 端口冲突。
\end{quote}
