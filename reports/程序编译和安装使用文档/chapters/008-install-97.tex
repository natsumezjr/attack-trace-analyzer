\section{单机靶场落地步骤}\label{single-host-range-setup}

本文面向靶场负责人,给出在单台内网 Linux 主机上落地 \textbf{≥5 节点靶场}的固定步骤,确保可部署、可复现、可观测。

本文只覆盖靶场底座与采集链路。中心机业务接口与算法接线不影响本文目标达成。

\subsection{前置:连接服务器}\label{prereq-connect-server}

如果你使用WSL环境,需要先连接到服务器。详细步骤见:

\begin{itemize}
\tightlist
\item
  \texttt{98-WSL连接服务器指南.md}
\end{itemize}

连接成功后,在服务器上继续执行本文后续步骤。

\subsection{固定目标状态}\label{target-status}

靶场最终应达到以下状态:

\begin{itemize}
\tightlist
\item
  节点:\texttt{center-01} + \texttt{client-01..04} + \texttt{c2-01}
\item
  端口:\texttt{91-靶场部署.md} 的端口规划全部可达
\item
  采集:\texttt{94-验证清单.md} 中 4 个 client 的三接口可返回 \texttt{total}
\item
  证据:对 \texttt{\textless{}C2\_DOMAIN\textgreater{}} 发起 DNS 与 HTTP 请求后,\texttt{macvlan0} 抓包可见 53 与 80 流量
\end{itemize}

\subsection{宿主机条件检查}\label{host-requirements}

在宿主机执行:

\begin{verbatim}
ip -br a
ip route
docker --version
docker-compose --version
\end{verbatim}

网卡口径固定为:

\begin{itemize}
\tightlist
\item
  物理网卡:\texttt{ens5f1}
\item
  C2 抓包网卡:\texttt{macvlan0}
\end{itemize}

\subsection{目录规划与变量}\label{directory-planning}

本文使用固定根目录:

\begin{verbatim}
export BASE=/home/ubuntu/attack-trace-analyzer
export REPO="$BASE"/repo/attack-trace-analyzer
\end{verbatim}

准备目录:

\begin{verbatim}
mkdir -p "$BASE"/{repo,run}
\end{verbatim}

\subsection{部署中心机基础设施}\label{deploy-center-infra}

在仓库目录执行:

\begin{verbatim}
cd "$REPO"/backend
docker-compose up -d
\end{verbatim}

验证端口监听:

\begin{verbatim}
ss -lntup | egrep ':(9200|9600|7474|7687)\\b' || true
\end{verbatim}

验证 OpenSearch:

\begin{verbatim}
curl -k -s https://<CENTER_IP>:9200 >/dev/null && echo "opensearch ok"
\end{verbatim}

\subsection{部署 c2-01}\label{deploy-c2}

按 \texttt{93-C2部署与证据点.md} 的第 3 节到第 7 节依次执行,完成:

\begin{enumerate}
\def\labelenumi{\arabic{enumi}.}
\tightlist
\item
  创建 Docker macvlan 网络 \texttt{c2-macvlan}
\item
  创建宿主机 \texttt{macvlan0} 并添加两条 \texttt{/32} 路由
\item
  启动 \texttt{c2-dns} 与 \texttt{c2-http}
\item
  完成 DNS、HTTP 与抓包验收
\end{enumerate}

\subsection{部署 4 套 client(单机多实例)}\label{deploy-4-clients}

按 \texttt{92-一键编排.md} 的第 2.3 节依次执行,完成:

\begin{enumerate}
\def\labelenumi{\arabic{enumi}.}
\tightlist
\item
  创建 \texttt{run/client-01..04} 目录
\item
  放置每个实例的 \texttt{docker-compose.yml} 与 \texttt{sensor}、\texttt{backend} 软链接
\item
  写入每个实例的 \texttt{.env}(端口与 Host 标识)
\item
  启动 4 套实例
\end{enumerate}

验证(以 client-01 为例):

\begin{verbatim}
curl -s http://<CENTER_IP>:18881/filebeat | head
curl -s http://<CENTER_IP>:18881/falco | head
curl -s http://<CENTER_IP>:18881/suricata | head
\end{verbatim}

\subsection{做一次证据动作闭环验收}\label{verify-evidence-loop}

在宿主机执行:

\begin{verbatim}
dig @<C2_DNS_IP> <C2_DOMAIN> +noall +answer +time=1 +tries=1
curl -s http://<C2_HTTP_IP>/health && echo
\end{verbatim}

然后按 \texttt{94-验证清单.md} 依次完成:

\begin{enumerate}
\def\labelenumi{\arabic{enumi}.}
\tightlist
\item
  网络端口检查
\item
  4 个 client 的三接口产数检查
\item
  中心机注册表与检索接口检查
\end{enumerate}

\subsection{重置与复现}\label{reset-reproduce}

演示前重置与排障入口为:

\begin{itemize}
\tightlist
\item
  \texttt{95-重置复现与排障.md}
\end{itemize}
