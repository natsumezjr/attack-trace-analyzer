\section{重置复现与排障}\label{reset-reproduce-troubleshoot}

本文件给出靶场的固定重置步骤、固定复现步骤与固定排障顺序,保证每次演示从干净状态开始。

\subsection{固定变量}\label{fixed-variables}

本文件命令默认在靶场宿主机执行,并使用以下固定变量:

\begin{verbatim}
export BASE=/home/ubuntu/attack-trace-analyzer
export REPO="$BASE"/repo/attack-trace-analyzer
\end{verbatim}

\subsection{重置(固定步骤)}\label{reset-steps}

\subsubsection{停机(固定)}\label{stop-services}

\begin{enumerate}
\def\labelenumi{\arabic{enumi}.}
\tightlist
\item
  停止前端(运行 \texttt{npm\ run\ dev} 的终端按 \texttt{Ctrl+C})
\item
  停止后端(运行 \texttt{uvicorn} 的终端按 \texttt{Ctrl+C})
\item
  停止 4 套客户机容器
\item
  停止 C2 容器
\item
  停止中心机依赖并清空存储
\end{enumerate}

\begin{verbatim}
for i in 01 02 03 04; do
  cd "$BASE"/run/client-$i
  docker-compose -p client-$i down
done

docker rm -f c2-dns c2-http || true

cd "$REPO"/backend
docker-compose down -v
\end{verbatim}

\subsubsection{清理数据(固定)}\label{cleanup-data}

清理客户机数据目录(清空 Falco/Filebeat/Suricata 输出与状态文件):

\begin{verbatim}
for i in 01 02 03 04; do
  rm -rf "$BASE"/run/client-$i/data/*
done
\end{verbatim}

重置 C2 静态内容(保留网络与容器配置口径):

\begin{verbatim}
rm -rf "$BASE"/run/c2/html/*
printf "ok\n" > "$BASE"/run/c2/html/health
printf "hello from c2 (benign)\n" > "$BASE"/run/c2/html/payload
\end{verbatim}

通过标准(固定):

\begin{itemize}
\tightlist
\item
  \texttt{"\$BASE/run/client-01/data"} 到 \texttt{"\$BASE/run/client-04/data"} 目录为空;
\item
  \texttt{"\$BASE/run/c2/html/health"} 与 \texttt{"\$BASE/run/c2/html/payload"} 文件存在且内容可读。
\end{itemize}

\subsection{复现(固定步骤)}\label{reproduce-steps}

复现步骤固定为:

\begin{enumerate}
\def\labelenumi{\arabic{enumi}.}
\tightlist
\item
  按 \texttt{92-一键编排.md} 启动全部服务;
\item
  按 \texttt{93-C2部署与证据点.md} 执行 DNS 与 HTTP 两条证据命令;
\item
  按 \texttt{94-验证清单.md} 完成网络、采集、入库与证据点验证。
\end{enumerate}

\subsection{排障顺序(固定)}\label{troubleshoot-order}

排障严格按以下顺序执行,不允许跳步:

\begin{enumerate}
\def\labelenumi{\arabic{enumi}.}
\tightlist
\item
  \textbf{后端健康检查}:\texttt{curl\ -sS\ http://localhost:8001/health} 返回 \texttt{\{"status":"ok"\}}
\item
  \textbf{OpenSearch 可用}:\texttt{curl\ -k\ -u\ admin:OpenSearch@2024!Dev\ https://localhost:9200/\_cluster/health} 的 \texttt{status} 为 \texttt{yellow\textbar{}green}
\item
  \textbf{C2 可用}:\texttt{dig\ @\textless{}C2\_DNS\_IP\textgreater{}\ \textless{}C2\_DOMAIN\textgreater{}\ +short} 返回 \texttt{\textless{}C2\_HTTP\_IP\textgreater{}},且 \texttt{curl\ -s\ http://\textless{}C2\_HTTP\_IP\textgreater{}/health} 返回 \texttt{ok}
\item
  \textbf{客户机接口可用}:\texttt{curl\ -sS\ http://\textless{}CENTER\_IP\textgreater{}:18881/suricata} 返回 JSON 且包含 \texttt{total}
\item
  \textbf{注册表可见}:\texttt{curl\ -sS\ http://localhost:8001/api/v1/clients} 的结果包含 \texttt{client-01..04}
\item
  \textbf{Suricata 输出存在}:\texttt{"\$BASE/run/client-01/data/eve.json"} 文件存在且文件大小大于 0
\end{enumerate}

当上述任一项不满足时,直接在该项对应的组件上定位问题,不进入下一项。

\subsection{常见问题与解决方案}\label{common-solutions}

\subsubsection{容器名称冲突(\texttt{Cannot create container for service rabbitmq: Conflict})}\label{container-name-conflict}

\textbf{问题}:多个 client 实例启动时,容器名称冲突。

\textbf{原因}:多个实例使用相同的 docker-compose.yml,没有指定不同的项目名称。

\textbf{解决方案}:

\begin{enumerate}
\def\labelenumi{\arabic{enumi}.}
\item
  先停止所有冲突的容器:

\begin{verbatim}
docker rm -f rabbitmq || true
for i in 01 02 03 04; do
  cd "$BASE"/run/client-$i
  docker-compose -p client-$i down
done
\end{verbatim}
\item
  使用 \texttt{-p} 参数指定项目名称重新启动:

\begin{verbatim}
for i in 01 02 03 04; do
  cd "$BASE"/run/client-$i
  docker-compose -p client-$i up -d --build
done
\end{verbatim}
\end{enumerate}

\subsubsection{检查当前运行状态}\label{check-running-status}

\begin{verbatim}
## 检查所有client实例的容器
for i in 01 02 03 04; do
  echo "=== client-$i ==="
  docker ps --filter "name=client-$i" --format "table {{.Names}}\t{{.Status}}\t{{.Ports}}"
done

## 检查端口占用
ss -lntup | grep -E ':(18881|18882|18883|18884)\b'

## 检查中心机服务
ss -lntup | grep -E ':(9200|7474|7687|8001|3000)\b'
\end{verbatim}

\subsubsection{清理所有残留容器}\label{cleanup-all-containers}

如果遇到无法启动的情况,可以强制清理:

\begin{verbatim}
## 停止并删除所有client相关容器
for i in 01 02 03 04; do
  cd "$BASE"/run/client-$i
  docker-compose -p client-$i down -v
done

## 删除可能残留的容器
docker ps -a | grep -E 'client-|rabbitmq|falco|filebeat|suricata' | awk '{print $1}' | xargs -r docker rm -f
\end{verbatim}
