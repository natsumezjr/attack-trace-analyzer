\section{编译安装与使用}\label{compile-install}

\subsection{文档目的}\label{doc-purpose}

本文件给出从零开始部署并运行ATA系统的固定步骤,包含前置依赖、启动顺序、健康检查、常用操作与停机清理。

\subsection{读者对象}\label{target-audience}

\begin{itemize}
\tightlist
\item
  负责部署与联调的同学
\item
  负责现场演示的同学
\end{itemize}

\subsection{引用关系}\label{references}

\begin{itemize}
\tightlist
\item
  靶场部署:\texttt{91-靶场部署.md}
\item
  一键编排:\texttt{92-一键编排.md}
\item
  验证清单:\texttt{94-验证清单.md}
\end{itemize}

\subsection{前置依赖}\label{prerequisites}

ATA系统运行依赖固定为:

\begin{itemize}
\tightlist
\item
  Docker Engine + docker-compose
\item
  Node.js + npm(用于前端)
\item
  Python 3.12 + uv(用于后端)
\end{itemize}

\begin{quote}
说明:Go 用于客户机服务的镜像构建,ATA系统运行时不要求宿主机安装 Go。
\end{quote}

\subsection{启动顺序}\label{startup-order}

启动顺序固定为:

\begin{enumerate}
\def\labelenumi{\arabic{enumi}.}
\tightlist
\item
  启动中心机依赖(OpenSearch、Neo4j)
\item
  启动 C2(DNS+HTTP)
\item
  启动客户机采集栈
\item
  启动中心机后端(FastAPI)
\item
  启动中心机前端(Next.js)
\item
  注册客户机到中心机
\item
  执行证据动作并按验证清单验收
\end{enumerate}

\subsection{健康检查}\label{health-check}

\subsubsection{后端健康检查}\label{backend-health}

\begin{verbatim}
curl -sS http://localhost:8001/health
\end{verbatim}

预期输出固定为:

\begin{verbatim}
{"status":"ok"}
\end{verbatim}

\subsubsection{OpenSearch 健康检查}\label{opensearch-health}

\begin{verbatim}
curl -k -sS -u admin:OpenSearch@2024!Dev https://localhost:9200/_cluster/health
\end{verbatim}

预期输出包含字段 \texttt{status},其值为 \texttt{yellow} 或 \texttt{green}。

\subsection{常用操作}\label{common-operations}

\subsubsection{启动全部服务}\label{start-all-services}

启动命令在 \texttt{92-一键编排.md} 定义,本文件不重复。

\subsubsection{服务访问入口(固定)}\label{service-endpoints}

{\small
\begin{itemize}
\tightlist
\item
  前端:\texttt{http://\textless{}CENTER\_IP\textgreater{}:3000}
\item
  后端健康检查:\texttt{http://\textless{}CENTER\_IP\textgreater{}:8001/health}
\item
  Neo4j Browser:\texttt{http://\textless{}CENTER\_IP\textgreater{}:7474}
\item
  OpenSearch:\texttt{https://\textless{}CENTER\_IP\textgreater{}:9200}
\item
  客户机拉取接口(4 个实例):

  \begin{itemize}
  \tightlist
  \item
    client-01:\texttt{http://\textless{}CENTER\_IP\textgreater{}:18881/*}
  \item
    client-02:\texttt{http://\textless{}CENTER\_IP\textgreater{}:18882/*}
  \item
    client-03:\texttt{http://\textless{}CENTER\_IP\textgreater{}:18883/*}
  \item
    client-04:\texttt{http://\textless{}CENTER\_IP\textgreater{}:18884/*}
  \end{itemize}
\end{itemize}
}

\subsubsection{导入 Neo4j 演示数据(固定)}\label{import-neo4j-data}

ATA系统的图查询与溯源任务依赖 Neo4j 图数据。图数据通过固定脚本导入:

\begin{verbatim}
cd backend
uv run python scripts/import_test_data.py
\end{verbatim}

\subsection{停机与清理}\label{shutdown-cleanup}

停机与清理在 \texttt{95-重置复现与排障.md} 定义,本文件不重复。
