\section{靶场部署}\label{range-deployment}

本文件定义演示靶场的固定拓扑、固定地址规划、固定端口规划与固定数据目录规划。

靶场在单机条件下提供 \textbf{≥5 个逻辑节点},用于满足验收中的``多节点''口径。

\subsection{固定拓扑(6 节点)}\label{fixed-topology}

ATA系统靶场固定由以下 6 个节点构成,全部部署在同一台内网 Linux 宿主机 \texttt{\textless{}CENTER\_IP\textgreater{}} 上:

{\def\LTcaptype{none} % do not increment counter
\begin{longtable}[]{@{}lrll@{}}
\toprule\noalign{}
节点 & 数量 & 部署形态 & 作用 \\
\midrule\noalign{}
\endhead
\bottomrule\noalign{}
\endlastfoot
\texttt{center-01} & 1 & OpenSearch+Neo4j 为 Docker Compose;后端与前端为宿主进程 & 数据入库、检测融合、图查询与任务执行、前端展示 \\
\texttt{client-01..04} & 4 & 每个实例一套 Docker Compose & Falco/Filebeat/Suricata 采集 → RabbitMQ 队列缓冲 → Client API 拉取 \\
\texttt{c2-01} & 1 & Docker macvlan + docker run & CoreDNS + Nginx,提供 DNS+HTTP,用于产生稳定可观测的网络证据 \\
\end{longtable}
}

\subsection{固定地址与网卡规划}\label{fixed-address-plan}

靶场网络口径固定为:

\begin{itemize}
\tightlist
\item
  宿主机内网 IP:\texttt{\textless{}CENTER\_IP\textgreater{}}
\item
  宿主机物理网卡:\texttt{ens5f1}
\item
  宿主机 macvlan 子接口:\texttt{macvlan0}(宿主 IP 固定为 \texttt{\textless{}MACVLAN\_HOST\_IP\textgreater{}/24})
\item
  C2 DNS IP:\texttt{\textless{}C2\_DNS\_IP\textgreater{}}
\item
  C2 HTTP IP:\texttt{\textless{}C2\_HTTP\_IP\textgreater{}}
\item
  C2 域名:\texttt{\textless{}C2\_DOMAIN\textgreater{}}(固定解析到 \texttt{\textless{}C2\_HTTP\_IP\textgreater{}})
\end{itemize}

\begin{quote}
说明:C2 使用 macvlan 的目标是让 DNS/HTTP 访问具备可观测的二层链路,并能在宿主机 \texttt{macvlan0} 上稳定抓包验收。
\end{quote}

\subsection{固定端口规划}\label{fixed-port-plan}

\subsubsection{中心机端口(宿主机)}\label{center-ports}

\begin{itemize}
\tightlist
\item
  OpenSearch:\texttt{9200/tcp}、\texttt{9600/tcp}
\item
  Neo4j:\texttt{7474/tcp}、\texttt{7687/tcp}
\item
  后端 API:\texttt{8001/tcp}
\item
  前端:\texttt{3000/tcp}
\end{itemize}

\subsubsection{客户机端口(宿主机)}\label{client-ports}

客户机对宿主机暴露的端口固定为 4 个 Client API 端口(容器内端口为 8888):

{\def\LTcaptype{none} % do not increment counter
\begin{longtable}[]{@{}lrl@{}}
\toprule\noalign{}
实例 & 宿主端口 & 说明 \\
\midrule\noalign{}
\endhead
\bottomrule\noalign{}
\endlastfoot
\texttt{client-01} & \texttt{18881/tcp} & \texttt{GET\ /filebeat}、\texttt{GET\ /falco}、\texttt{GET\ /suricata} \\
\texttt{client-02} & \texttt{18882/tcp} & 同上 \\
\texttt{client-03} & \texttt{18883/tcp} & 同上 \\
\texttt{client-04} & \texttt{18884/tcp} & 同上 \\
\end{longtable}
}

RabbitMQ 端口固定只在容器网络内部使用:

\begin{itemize}
\tightlist
\item
  AMQP:\texttt{5672/tcp}
\item
  管理端口:\texttt{15672/tcp}
\end{itemize}

\subsubsection{C2 端口(C2 独立 IP)}\label{c2-ports}

由于 C2 拥有独立 IP,不做宿主端口映射,端口固定为:

\begin{itemize}
\tightlist
\item
  DNS:\texttt{\textless{}C2\_DNS\_IP\textgreater{}:53/udp}、\texttt{\textless{}C2\_DNS\_IP\textgreater{}:53/tcp}
\item
  HTTP:\texttt{\textless{}C2\_HTTP\_IP\textgreater{}:80/tcp}
\end{itemize}

\subsection{固定数据目录}\label{fixed-data-dirs}

靶场运行产生的数据固定落在以下位置:

\begin{itemize}
\tightlist
\item
  统一根目录:\texttt{\$BASE=/home/ubuntu/attack-trace-analyzer}
\item
  代码目录:\texttt{\$BASE/repo/attack-trace-analyzer/}
\item
  运行目录:\texttt{\$BASE/run/}

  \begin{itemize}
  \tightlist
  \item
    \texttt{client-01/}、\texttt{client-02/}、\texttt{client-03/}、\texttt{client-04/}

    \begin{itemize}
    \tightlist
    \item
      \texttt{docker-compose.yml}:客户机运行编排
    \item
      \texttt{.env}:该实例的端口与 Host 标识
    \item
      \texttt{sensor}、\texttt{backend}:指向仓库 \texttt{client/} 的软链接,用于 compose 构建
    \item
      \texttt{data/}:采集输出(\texttt{falco.jsonl}、\texttt{eve.json} 等)
    \end{itemize}
  \item
    \texttt{c2/}:C2 配置与静态页面(见 \texttt{93-C2部署与证据点.md})
  \end{itemize}
\item
  OpenSearch/Neo4j 持久化:Docker named volumes(由 \texttt{backend/docker-compose.yml} 创建)
\end{itemize}

重置与清理步骤见:

\begin{itemize}
\tightlist
\item
  \texttt{95-重置复现与排障.md}
\end{itemize}

\subsection{下一步}\label{next-steps-91}

一键启动/停止与启动顺序见:

\begin{itemize}
\tightlist
\item
  \texttt{92-一键编排.md}
\end{itemize}
