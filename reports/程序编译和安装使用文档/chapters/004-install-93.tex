\section{C2部署与证据点}\label{c2-deployment-evidence}

本文件定义 C2(DNS+HTTP)在靶场中的固定部署方式与固定证据生成方式,用于稳定生成 DNS 与 HTTP 两类网络证据,并可在宿主机抓包验收。

\subsection{固定约定}\label{fixed-conventions}

ATA系统靶场对 C2 的固定约定如下:

\begin{itemize}
\tightlist
\item
  Docker macvlan 网络:\texttt{c2-macvlan}
\item
  C2 DNS IP:\texttt{\textless{}C2\_DNS\_IP\textgreater{}}
\item
  C2 HTTP IP:\texttt{\textless{}C2\_HTTP\_IP\textgreater{}}
\item
  C2 域名:\texttt{\textless{}C2\_DOMAIN\textgreater{}}(固定解析到 \texttt{\textless{}C2\_HTTP\_IP\textgreater{}})
\item
  HTTP 路径:

  \begin{itemize}
  \tightlist
  \item
    \texttt{http://\textless{}C2\_HTTP\_IP\textgreater{}/health}
  \item
    \texttt{http://\textless{}C2\_HTTP\_IP\textgreater{}/payload}
  \end{itemize}
\end{itemize}

\subsection{macvlan 前置条件(必须满足)}\label{macvlan-prerequisites}

\begin{itemize}
\tightlist
\item
  宿主机内网网卡:\texttt{ens5f1}
\item
  内网网段:\texttt{\textless{}LAN\_CIDR\textgreater{}}
\item
  C2 IP \texttt{\textless{}C2\_DNS\_IP\textgreater{}} 与 \texttt{\textless{}C2\_HTTP\_IP\textgreater{}} 未被占用
\end{itemize}

\subsection{创建 Docker macvlan 网络(固定命令)}\label{create-macvlan-network}

在宿主机执行:

\begin{verbatim}
sudo docker network create -d macvlan \
  --subnet=<LAN_CIDR> \
  --gateway=<LAN_GW_IP> \
  -o parent=ens5f1 \
  c2-macvlan
\end{verbatim}

\subsection{宿主机创建 macvlan0 并添加路由(固定命令)}\label{create-macvlan0-routing}

macvlan 的默认行为会导致宿主机无法直接访问 macvlan 容器 IP。ATA系统固定通过创建宿主机 \texttt{macvlan0} 并添加两条 \texttt{/32} 路由解决该问题。

宿主机 \texttt{macvlan0} 的 IP 固定为 \texttt{\textless{}MACVLAN\_HOST\_IP\textgreater{}/24}:

\begin{verbatim}
sudo ip link add macvlan0 link ens5f1 type macvlan mode bridge
sudo ip addr add <MACVLAN_HOST_IP>/24 dev macvlan0
sudo ip link set macvlan0 up

sudo ip route add <C2_DNS_IP>/32 dev macvlan0 || true
sudo ip route add <C2_HTTP_IP>/32 dev macvlan0 || true
\end{verbatim}

\subsection{准备 C2 配置与静态内容(固定命令)}\label{prepare-c2-config}

ATA系统固定将 C2 的配置与静态文件放在:\texttt{\$BASE/run/c2/}。

仓库内同时包含 \texttt{scripts/靶场编排/c2\_config/} 目录,其内容用于保存 C2 的配置文件模板。靶场运行命令以本文第 6 节为准。

\begin{verbatim}
export BASE=/home/ubuntu/attack-trace-analyzer
mkdir -p "$BASE"/run/c2/html
\end{verbatim}

CoreDNS 配置文件固定为:\texttt{\$BASE/run/c2/Corefile}

\begin{verbatim}
cat > "$BASE"/run/c2/Corefile <<'EOF'
.:53 {
  log
  errors
  hosts /etc/coredns/hosts {
    reload 1s
    fallthrough
  }
  forward . 223.5.5.5 114.114.114.114
}
EOF
\end{verbatim}

hosts 记录固定为:\texttt{\$BASE/run/c2/hosts}

\begin{verbatim}
cat > "$BASE"/run/c2/hosts <<'EOF'
<C2_HTTP_IP> <C2_DOMAIN>
EOF
\end{verbatim}

HTTP 静态内容固定为两个文件:

\begin{itemize}
\tightlist
\item
  \texttt{\$BASE/run/c2/html/health}
\item
  \texttt{\$BASE/run/c2/html/payload}
\end{itemize}

\begin{verbatim}
printf "ok\n" > "$BASE"/run/c2/html/health
printf "hello from c2 (benign)\n" > "$BASE"/run/c2/html/payload
\end{verbatim}

\subsection{启动 C2(固定命令)}\label{start-c2-containers}

ATA系统固定启动两个容器:

\begin{itemize}
\tightlist
\item
  \texttt{c2-dns}:CoreDNS(IP=\textless C2\_DNS\_IP\textgreater)
\item
  \texttt{c2-http}:Nginx(IP=\textless C2\_HTTP\_IP\textgreater)
\end{itemize}

\begin{verbatim}
docker rm -f c2-dns c2-http || true

docker run -d --name c2-dns \
  --network c2-macvlan --ip <C2_DNS_IP> \
  -v "$BASE"/run/c2/Corefile:/etc/coredns/Corefile:ro \
  -v "$BASE"/run/c2/hosts:/etc/coredns/hosts:ro \
  coredns/coredns:latest

docker run -d --name c2-http \
  --network c2-macvlan --ip <C2_HTTP_IP> \
  -v "$BASE"/run/c2/html:/usr/share/nginx/html:ro \
  nginx:latest
\end{verbatim}

\subsection{验证 C2(固定命令 + 固定预期)}\label{verify-c2}

\subsubsection{DNS 验证}\label{dns-verification}

\begin{verbatim}
dig @<C2_DNS_IP> <C2_DOMAIN> +short
\end{verbatim}

预期输出固定为:

\begin{verbatim}
<C2_HTTP_IP>
\end{verbatim}

\subsubsection{HTTP 验证}\label{http-verification}

\begin{verbatim}
curl -s http://<C2_HTTP_IP>/health && echo
curl -s http://<C2_HTTP_IP>/payload && echo
\end{verbatim}

预期输出固定包含:

\begin{itemize}
\tightlist
\item
  \texttt{ok}
\item
  \texttt{hello\ from\ c2\ (benign)}
\end{itemize}

\subsubsection{抓包验收(固定)}\label{packet-capture-verification}

\begin{verbatim}
sudo timeout 3 tcpdump -i macvlan0 -nn '(host <C2_DNS_IP> and (udp port 53 or tcp port 53)) or (host <C2_HTTP_IP> and tcp port 80)' &
sleep 0.2
dig @<C2_DNS_IP> <C2_DOMAIN> +noall +answer +time=1 +tries=1
curl -s http://<C2_HTTP_IP>/health && echo
wait
\end{verbatim}

\subsection{证据点(固定生成方式)}\label{evidence-points}

执行第 7 节的 DNS 与 HTTP 命令会固定产生:

\begin{itemize}
\tightlist
\item
  DNS 查询:\texttt{\textless{}CENTER\_IP\textgreater{}\ -\textgreater{}\ \textless{}C2\_DNS\_IP\textgreater{}:53}(包含 \texttt{\textless{}C2\_DOMAIN\textgreater{}})
\item
  HTTP 访问:\texttt{\textless{}CENTER\_IP\textgreater{}\ -\textgreater{}\ \textless{}C2\_HTTP\_IP\textgreater{}:80}(包含 \texttt{/health}、\texttt{/payload})
\end{itemize}

这些证据的验证步骤见:

\begin{itemize}
\tightlist
\item
  \texttt{94-验证清单.md}
\end{itemize}
