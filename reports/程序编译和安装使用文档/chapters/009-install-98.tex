\section{WSL连接服务器指南}\label{wsl-server-connection}

本文面向使用WSL环境的用户,提供从WSL连接到靶场服务器\texttt{\textless{}CENTER\_IP\textgreater{}}的完整步骤。

\subsection{重要说明}\label{important-notes}

{\small
\begin{itemize}
\tightlist
\item
  \textbf{网卡\texttt{ens5f1}是服务器端的配置},你的WSL本地环境不需要这个网卡
\item
  你只需要能SSH连接到服务器即可,所有靶场配置都在服务器端完成
\item
  服务器IP \texttt{\textless{}CENTER\_IP\textgreater{}} 是内网地址,需要通过VPN或内网路由访问
\end{itemize}
}

\subsection{检查网络连通性}\label{check-network-connectivity}

\subsubsection{检查是否能ping通服务器}\label{check-ping-server}

在WSL中执行:

\begin{verbatim}
ping -c 3 <CENTER_IP>
\end{verbatim}

\subsubsection{检查SSH端口是否开放}\label{check-ssh-port}

\begin{verbatim}
## 方法1:使用nc(如果已安装)
nc -zv <CENTER_IP> 22

## 方法2:使用ssh测试(推荐)
ssh -v ubuntu@<CENTER_IP>
\end{verbatim}

\subsubsection{检查WSL网络配置(可选)}\label{check-wsl-network}

\begin{verbatim}
## 查看WSL IP和路由
ip addr show
ip route
\end{verbatim}

\subsection{SSH连接服务器}\label{ssh-connection}

\subsubsection{基本连接命令}\label{basic-ssh-command}

\begin{verbatim}
ssh ubuntu@<CENTER_IP>
\end{verbatim}

\textbf{密码}:通过线下渠道分发

\subsubsection{如果连接失败,尝试以下方案}\label{connection-failures}

\paragraph{方案A:指定SSH端口(如果不是22)}\label{solution-specify-ssh-port}

\begin{verbatim}
ssh -p <端口号> ubuntu@<CENTER_IP>
\end{verbatim}

\paragraph{方案B:使用跳板机(如果需要)}\label{solution-jump-host}

\begin{verbatim}
ssh -J jump_user@jump_host ubuntu@<CENTER_IP>
\end{verbatim}

\paragraph{方案C:使用SSH密钥(如果提供了密钥文件)}\label{solution-ssh-key}

\begin{verbatim}
ssh -i /path/to/private_key ubuntu@<CENTER_IP>
\end{verbatim}

\paragraph{方案D:详细调试模式(查看连接过程)}\label{solution-verbose-debug}

\begin{verbatim}
ssh -v ubuntu@<CENTER_IP>
\end{verbatim}

\subsection{连接成功后的验证}\label{verify-connection}

连接成功后,在服务器上依次执行以下命令:

\subsubsection{确认工作目录}\label{verify-work-directory}

\begin{verbatim}
export BASE=/home/ubuntu/attack-trace-analyzer
export REPO="$BASE"/repo/attack-trace-analyzer

cd "$BASE"
pwd
ls -la
\end{verbatim}

\subsubsection{检查Docker配置}\label{verify-docker}

\begin{verbatim}
docker --version
docker-compose --version
docker ps
\end{verbatim}

\subsubsection{检查服务器网卡(这是服务器端的,你本地不需要)}\label{verify-server-nic}

\begin{verbatim}
ip -br a | grep ens5f1
ip route
\end{verbatim}

\subsubsection{检查项目文件结构}\label{verify-project-structure}

\begin{verbatim}
cd "$REPO"
ls -la
\end{verbatim}

\subsubsection{检查已运行的服务(如果有)}\label{verify-running-services}

\begin{verbatim}
ss -lntup | grep -E ':(9200|7474|7687|18881|18882|18883|18884)\b'
\end{verbatim}

\subsection{如果无法连接,排查步骤}\label{troubleshoot-connection}

\subsubsection{检查Windows防火墙}\label{check-windows-firewall}

在Windows PowerShell中执行:

\begin{verbatim}
## 检查防火墙状态
Get-NetFirewallProfile | Select-Object Name, Enabled
\end{verbatim}

\subsubsection{检查WSL网络模式}\label{check-wsl-network-mode}

在WSL中执行:

\begin{verbatim}
## 查看WSL IP
hostname -I

## 查看路由
ip route

## 测试DNS解析
nslookup <CENTER_IP>
\end{verbatim}

\subsubsection{检查是否需要VPN}\label{check-vpn-required}

\begin{itemize}
\tightlist
\item
  询问交付人是否需要连接VPN才能访问内网\texttt{\textless{}LAN\_CIDR\textgreater{}}
\item
  如果需要VPN,先连接VPN再尝试SSH
\end{itemize}

\subsubsection{询问交付人的问题清单}\label{questions-for-deliverer}

如果以上步骤都无法连接,请向交付人确认:

\begin{enumerate}
\def\labelenumi{\arabic{enumi}.}
\tightlist
\item
  是否需要VPN或特殊网络配置?
\item
  是否需要跳板机?如果有,跳板机地址和用户名是什么?
\item
  SSH端口是否为22?如果不是,端口号是多少?
\item
  是否需要SSH密钥?如果需要,密钥文件在哪里?
\item
  WSL是否能访问\texttt{\textless{}LAN\_CIDR\textgreater{}}网段?
\item
  是否有其他网络限制或配置要求?
\end{enumerate}

\subsection{连接成功后的下一步}\label{next-steps-after-connection}

连接成功后,按照以下文档进行部署:

\begin{enumerate}
\def\labelenumi{\arabic{enumi}.}
\tightlist
\item
  \textbf{完整部署步骤}:\texttt{97-单机靶场落地步骤.md}
\item
  \textbf{一键编排}:\texttt{92-一键编排.md}
\item
  \textbf{C2部署}:\texttt{93-C2部署与证据点.md}
\item
  \textbf{验证清单}:\texttt{94-验证清单.md}
\item
  \textbf{重置与排障}:\texttt{95-重置复现与排障.md}
\end{enumerate}

\subsection{快速参考}\label{quick-reference}

\subsubsection{服务器信息}\label{server-information}

\begin{itemize}
\tightlist
\item
  \textbf{IP}:\texttt{\textless{}CENTER\_IP\textgreater{}}
\item
  \textbf{用户名}:\texttt{ubuntu}
\item
  \textbf{工作目录}:\texttt{/home/ubuntu/attack-trace-analyzer}
\item
  \textbf{固定变量}:

\begin{verbatim}
export BASE=/home/ubuntu/attack-trace-analyzer
export REPO="$BASE"/repo/attack-trace-analyzer
\end{verbatim}
\end{itemize}

\subsubsection{常用端口(服务器端)}\label{common-ports-server}

\begin{itemize}
\tightlist
\item
  OpenSearch:\texttt{9200}
\item
  Neo4j Browser:\texttt{7474}
\item
  Neo4j Bolt:\texttt{7687}
\item
  后端:\texttt{8001}
\item
  前端:\texttt{3000}
\item
  Client-01 API:\texttt{18881}
\item
  Client-02 API:\texttt{18882}
\item
  Client-03 API:\texttt{18883}
\item
  Client-04 API:\texttt{18884}
\end{itemize}
