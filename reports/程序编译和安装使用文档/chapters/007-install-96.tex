\section{靶场交付说明}\label{range-delivery-guide}

本文面向组内同学与验收人员,给出靶场登录方式、逻辑拓扑、服务端口与数据目录位置。本文只描述如何连接与如何使用,不讨论实现细节。

\subsection{登录服务器}\label{login-server}

\begin{itemize}
\tightlist
\item
  服务器 IP:\texttt{\textless{}CENTER\_IP\textgreater{}}
\item
  用户名:\texttt{ubuntu}
\item
  密码:\texttt{{[}请在此处填写SSH密码{]}}
\item
  登录命令:
\end{itemize}

\begin{verbatim}
ssh ubuntu@<CENTER_IP>
\end{verbatim}

\begin{quote}
密码:\#EDCvfr4\%TGB10
\end{quote}

\begin{quote}
\textbf{WSL用户注意}:如果你使用WSL环境,详细连接步骤见 \texttt{98-WSL连接服务器指南.md}。
\end{quote}

\subsection{靶场逻辑拓扑}\label{range-logical-topology}

靶场在单机条件下提供 \textbf{6 个逻辑节点}:

\begin{itemize}
\tightlist
\item
  \texttt{center-01}:中心机(OpenSearch + Neo4j + 后端 API + 前端)
\item
  \texttt{client-01..04}:4 套客户机实例(Falco / Filebeat / Suricata → RabbitMQ → Client API)
\item
  \texttt{c2-01}:C2(DNS + HTTP,用于产生稳定网络证据)
\end{itemize}

\subsection{服务与端口}\label{services-ports}

\subsubsection{center-01}\label{31-center-01}

\begin{itemize}
\tightlist
\item
  OpenSearch:\texttt{https://\textless{}CENTER\_IP\textgreater{}:9200}
\item
  OpenSearch perf:\texttt{\textless{}CENTER\_IP\textgreater{}:9600}
\item
  Neo4j Browser:\texttt{http://\textless{}CENTER\_IP\textgreater{}:7474}
\item
  Neo4j Bolt:\texttt{\textless{}CENTER\_IP\textgreater{}:7687}
\item
  后端:\texttt{http://\textless{}CENTER\_IP\textgreater{}:8001}
\item
  前端:\texttt{http://\textless{}CENTER\_IP\textgreater{}:3000}
\end{itemize}

\subsubsection{client-01..04}\label{32-client-0104}

每个 client 对外提供 3 个接口(返回已消费到的数据,便于验收):

\begin{itemize}
\tightlist
\item
  \texttt{GET\ /filebeat}
\item
  \texttt{GET\ /falco}
\item
  \texttt{GET\ /suricata}
\end{itemize}

端口规划固定为:

\begin{itemize}
\tightlist
\item
  client-01:\texttt{http://\textless{}CENTER\_IP\textgreater{}:18881/*}
\item
  client-02:\texttt{http://\textless{}CENTER\_IP\textgreater{}:18882/*}
\item
  client-03:\texttt{http://\textless{}CENTER\_IP\textgreater{}:18883/*}
\item
  client-04:\texttt{http://\textless{}CENTER\_IP\textgreater{}:18884/*}
\end{itemize}

\subsubsection{c2-01}\label{33-c2-01}

\begin{itemize}
\tightlist
\item
  DNS:\texttt{\textless{}C2\_DNS\_IP\textgreater{}:53/udp}、\texttt{\textless{}C2\_DNS\_IP\textgreater{}:53/tcp}
\item
  HTTP:\texttt{http://\textless{}C2\_HTTP\_IP\textgreater{}:80}
\item
  域名:\texttt{\textless{}C2\_DOMAIN\textgreater{}\ -\textgreater{}\ \textless{}C2\_HTTP\_IP\textgreater{}}
\end{itemize}

\subsection{目录与数据位置}\label{directories-data}

靶场宿主机的根目录固定为:

\begin{itemize}
\tightlist
\item
  \texttt{BASE=/home/ubuntu/attack-trace-analyzer}
\end{itemize}

目录结构固定为:

\begin{itemize}
\tightlist
\item
  代码:\texttt{\$BASE/repo/attack-trace-analyzer/}
\item
  运行目录:\texttt{\$BASE/run/}

  \begin{itemize}
  \tightlist
  \item
    \texttt{client-01..04/}:每个客户机实例一个目录
  \item
    \texttt{c2/}:C2 配置与静态内容
  \end{itemize}
\end{itemize}

\subsection{常用操作入口}\label{common-operations}

\begin{itemize}
\tightlist
\item
  启停顺序与命令:\texttt{92-一键编排.md}
\item
  验收命令与通过标准:\texttt{94-验证清单.md}
\item
  演示前重置与排障:\texttt{95-重置复现与排障.md}
\end{itemize}
