% Manual polished content for: installation/build/run guide.
% This is the "main narrative" used for the course deliverable.
%
% Sources (single source of truth):
% - docs/README.md
% - README.md
% - docs/80-规范/*
% - docs/90-运维与靶场/*
%
% Notes:
% - Prefer placeholders over fixed IPs/ports in the main text.
% - Keep raw lab parameters (if any) in appendices only.

\section{文档说明与使用方式}

\begin{SideBar}
\textbf{目标:}本手册用于“从零开始把系统跑起来”,并能按固定步骤完成验收闭环(采集→入库→可视化→溯源)。\\
\textbf{原则:}ATA系统代码不再变更,本文档只描述仓库中已存在的实现与脚本;不引入超出代码的功能口径。\\
\textbf{证据形式:}本交付不提供录屏/截图;验收证据以\textbf{可复现命令输出(文本/JSON)}与\textbf{日志/导出文件}为准。
\end{SideBar}

\subsection{系统组成(你要启动哪些东西)}

系统由三部分组成:
\begin{itemize}
  \item \textbf{中心机后端(FastAPI)}:提供 API、轮询拉取、入库/检测融合、入图、溯源任务执行;
  \item \textbf{中心机前端(Next.js)}:可视化查询、图谱展示、溯源任务管理、报告导出;
  \item \textbf{客户机采集栈(Linux)}:Falco(主机行为)/ Filebeat(主机日志)/ Suricata(网络流量)采集 + RabbitMQ 缓冲 + 拉取接口(Go)。
\end{itemize}

\begin{KeyBox}
\textbf{重要限制(必须写清):}
\begin{itemize}
  \item 客户机采集栈依赖特权能力(Falco/Suricata 抓包/内核事件),\Emph{仅建议在 Linux 靶场环境运行};不建议在 macOS 上直接跑客户机。
  \item macOS 环境主要用于“中心机本地开发/演示”(OpenSearch + Neo4j + 后端 + 前端)。
\end{itemize}
\end{KeyBox}

\begin{figure}[htbp]
  \centering
  \includegraphics[width=0.98\textwidth]{figures/deployment_topology.pdf}
  \caption{系统部署拓扑概览(中心机 + 可选 Linux 客户机,Graphviz 重绘)}
  \label{fig:install-topology}
\end{figure}

\subsection{占位符与变量约定}

为避免在正文中散落固定内网参数,本文统一采用占位符:

{\def\LTcaptype{none}
\begin{longtable}[]{@{}p{0.26\textwidth}p{0.64\textwidth}@{}}
\toprule\noalign{}
占位符 & 含义 \\
\midrule\noalign{}
\endhead
\bottomrule\noalign{}
\endlastfoot
\texttt{<CENTER\_IP>} & 中心机 IP / 主机名(本地开发可用 \texttt{localhost}) \\
\texttt{<BACKEND\_PORT>} & 后端端口(默认 \texttt{8001}) \\
\texttt{<FRONTEND\_PORT>} & 前端端口(默认 \texttt{3000}) \\
\texttt{<OPENSEARCH\_PORT>} & OpenSearch HTTPS 端口(默认 \texttt{9200}) \\
\texttt{<NEO4J\_HTTP\_PORT>} & Neo4j Browser HTTP 端口(默认 \texttt{7474}) \\
\texttt{<NEO4J\_BOLT\_PORT>} & Neo4j Bolt 端口(默认 \texttt{7687}) \\
\texttt{<C2\_DNS\_IP>} / \texttt{<C2\_HTTP\_IP>} & C2 组件 DNS/HTTP IP(靶场专用) \\
\texttt{<C2\_DOMAIN>} & C2 域名(靶场专用) \\
\texttt{<BASE\_DIR>} & 靶场运行目录根(同机多实例部署时用于隔离) \\
\end{longtable}
}

\begin{SideBar}
\textbf{环境变量:}配置项与默认值以 \texttt{docs/80-规范/89-环境变量与配置规范.md} 为准。\\
\textbf{注意:}Docker Compose 的 \texttt{.env} 用于 compose 变量替换;后端进程不会自动加载 \texttt{backend/.env}。
\end{SideBar}

\section{路径 A:本地单机快速启动(中心机开发/演示)}

\begin{figure}[htbp]
  \centering
  \includegraphics[width=0.92\textwidth]{figures/startup_sequence.pdf}
  \caption{推荐启动顺序(最小闭环:基础设施→后端→前端→验证,Graphviz)}
  \label{fig:install-startup-seq}
\end{figure}

\subsection{先决条件}

\begin{itemize}
  \item Docker Desktop + Docker Compose v2(\texttt{docker compose})
  \item Python 3.12 + \texttt{uv}(后端)
  \item Node.js(建议 20+ LTS)+ npm/pnpm(前端)
\end{itemize}

\subsection{启动 OpenSearch + Neo4j(基础设施)}

\begin{verbatim}
cd backend
cp .env.example .env
docker compose up -d
\end{verbatim}

健康检查(示例):
\begin{verbatim}
# OpenSearch(开发环境自签名证书,可用 -k)
curl -k -u admin:<OPENSEARCH_PASSWORD> https://localhost:<OPENSEARCH_PORT>/_cluster/health

# Neo4j Browser
open http://localhost:<NEO4J_HTTP_PORT>
\end{verbatim}

\subsection{启动后端(FastAPI)}

\begin{verbatim}
cd backend
uv sync
uv run uvicorn main:app --reload --host 0.0.0.0 --port <BACKEND_PORT>
\end{verbatim}

健康检查:
\begin{verbatim}
curl -sS http://localhost:<BACKEND_PORT>/health
\end{verbatim}

\subsection{启动前端(Next.js)}

\begin{verbatim}
cd frontend
cp .env.example .env.local
# 编辑 .env.local:设置 BACKEND_BASE_URL=http://<CENTER_IP>:<BACKEND_PORT>
npm install
npm run dev
\end{verbatim}

访问:
\begin{verbatim}
open http://localhost:<FRONTEND_PORT>
\end{verbatim}

\begin{KeyBox}
\textbf{可选:离线 CTI(ATT\&CK)}

若要启用 TTP 相似度分析,按仓库脚本下载离线 CTI:
\begin{verbatim}
cd backend
./scripts/fetch_mitre_attack_cti.sh
\end{verbatim}
\end{KeyBox}

\section{路径 B:Linux 靶场(≥5 节点闭环验收)}

\begin{SideBar}
靶场部署的“固定顺序”与“证据点”已经在 \texttt{docs/90-运维与靶场/} 固化。\\
本节仅给出验收演示的最小操作闭环;完整脚本与细节见附录(自动导入的原始文档)。
\end{SideBar}

\subsection{固定启动顺序(务必按顺序执行)}

\begin{enumerate}
  \item 启动中心机依赖(OpenSearch、Neo4j)
  \item 启动 C2(DNS+HTTP)
  \item 启动客户机采集栈(client-01..04)
  \item 启动中心机后端(FastAPI)
  \item 启动中心机前端(Next.js)
  \item 注册客户机到中心机
  \item 执行证据动作并按验证清单验收
\end{enumerate}

\subsection{最小验收闭环(无截图/无录屏)}

验收不依赖截图/录屏,统一以命令输出作为证据:
\begin{itemize}
  \item 后端:\texttt{GET /health} 返回 \texttt{\{\"status\":\"ok\"\}}
  \item 客户机:\texttt{GET /filebeat} / \texttt{/falco} / \texttt{/suricata} 返回 JSON 且 \texttt{total>0}
  \item C2:\texttt{dig @<C2\_DNS\_IP> <C2\_DOMAIN>} 与 \texttt{curl http://<C2\_HTTP\_IP>/health} 可验证连通
\end{itemize}

\begin{KeyBox}
\textbf{权威验收标准:}
\begin{itemize}
  \item 验证清单:\texttt{docs/90-运维与靶场/94-验证清单.md}
  \item 攻击复现剧本:\texttt{docs/20-需求与验收/22-攻击场景与复现剧本.md}
  \item 验收标准与用例:\texttt{docs/20-需求与验收/21-验收标准与验收用例.md}
\end{itemize}
\end{KeyBox}

\section{停机、重置与排障(现场稳定性保障)}

\subsection{停机与重置}

停机与清理由 \texttt{docs/90-运维与靶场/95-重置复现与排障.md} 固定定义;现场演示前建议进行一次“重置→自检”,确保结果可复现。

\subsection{常见问题(按优先级排障)}

\begin{enumerate}
  \item \textbf{后端不可用}:先查 \texttt{/health},再查后端日志与环境变量;
  \item \textbf{OpenSearch 连接失败}:注意 HTTPS 自签名证书(\texttt{curl -k}),以及 Docker 内存不足导致启动慢/失败;
  \item \textbf{Neo4j 不可用}:检查 bolt/http 端口与密码配置;
  \item \textbf{客户机 total=0}:检查 Falco/Suricata 特权与网卡配置,RabbitMQ 队列是否堆积;
  \item \textbf{C2 不可用}:先验证 DNS/HTTP 的网络与路由,再看容器是否启动。
\end{enumerate}

\begin{Takeaway}
正文到此结束。后续附录为"仓库原始运维与靶场文档",用于追溯细节与固定口径。
\end{Takeaway}
