\documentclass[12pt]{article} % 也支持 14pt/17pt/20pt

% ==== 编译引擎 ====
\usepackage{ifxetex,ifluatex}
\ifxetex\else\ifluatex\else
  \errmessage{请使用 XeLaTeX 或 LuaLaTeX 编译}
\fi\fi

% ==== 字体与版式 ====
\usepackage{ctex,fontspec,geometry,xcolor}
\usepackage{ragged2e}  % 提供更好的左对齐文本支持
\geometry{paperwidth=7.5in,paperheight=10in, margin=0.7in}
% 采用 XeLaTeX 默认西文字体与 ctex 自动中文字体,减少依赖
\linespread{1.4}
\pagecolor{white}\color{black}

% 全局左对齐所有文本(包括 textbf 等)
\RaggedRight  % 使用 ragged2e 的版本,避免单词间隙过大

% ==== 配色 ====
\definecolor{brand}{HTML}{1A9D8F}      % 主品牌色
\definecolor{brandD}{HTML}{2A7F6F}     % 品牌深色
\definecolor{ink}{HTML}{1A1A1A}
\definecolor{keybg}{HTML}{E8F4F1}      % 浅绿色块
\definecolor{keydark}{HTML}{2F5E58}    % 深绿色文字
\definecolor{accent}{HTML}{FF6B6B}     % 强调红
\definecolor{accentD}{HTML}{D94F4F}
\definecolor{soft}{HTML}{F3F8F7}

% 定理配色
\definecolor{secblue}{HTML}{2B44C6}          % 定理深蓝
\definecolor{thmBg}{RGB}{235,245,255}        % 定理底色浅蓝
\definecolor{lemBg}{RGB}{255,240,240}        % 引理底色暖粉
\definecolor{defBg}{RGB}{232,244,241}        % 定义底色浅绿
% 左侧彩条
\colorlet{thmBar}{secblue}      % 定理左条
\colorlet{lemBar}{accentD}      % 引理左条
\colorlet{defBar}{brand}        % 定义左条

% 高亮辅助色
\definecolor{error}{HTML}{E53935}      % 错误/存疑红
\definecolor{info}{HTML}{64B5F6}       % 信息提示蓝
\definecolor{termblue}{HTML}{1A4C9A}   % 专业名词深蓝

% ==== 日期宏 ====
\newcommand{\padzero}[1]{\ifnum#1<10 0\fi\number#1}
\newcommand{\BrandYMD}{{\the\year·\padzero{\the\month}·\padzero{\the\day}}}

% ==== 封面信息 ====
% 说明:本项目交付封面仅保留学校/学院两行信息(不展示姓名/学号/班级)。
\newcommand{\DocTitle}{作品技术原理介绍}
\newcommand{\SchoolName}{北京邮电大学\\网络空间安全学院}

% ==== 超链接 ====
\usepackage[colorlinks=true, linkcolor=brandD, citecolor=brandD, urlcolor=brandD]{hyperref}
\AtBeginDocument{\hypersetup{pdfauthor={}, pdftitle={\DocTitle}}}

% ==== Markdown → LaTeX(pandoc 片段兼容) ====
% 说明:详细设计正文可能由 docs/*.md 自动转换生成(先自动化导入,再逐章润色)。
% 这些宏/包用于保证 pandoc 输出的 longtable/booktabs/tightlist 能直接编译。
\usepackage{graphicx,longtable,booktabs,array,tabularx}
\newcounter{none} % pandoc 可能设置 \LTcaptype=none,用一个空计数器避免 longtable 报错
\providecommand{\tightlist}{%
  \setlength{\itemsep}{0pt}\setlength{\parskip}{0pt}%
}

% ==== 修复表格溢出问题 ====
% 允许 longtable 跨页,并自动调整列宽
\setlength{\LTleft}{0pt}
\setlength{\LTright}{0pt}

% 减小 longtable 列间距
\setlength{\tabcolsep}{3pt}

% 允许表格内容适当缩放以适应页面宽度
\usepackage{adjustbox}

% 设置 longtable 默认字号为 \small
\AtBeginEnvironment{longtable}{\small}

% ==== 修复长路径和代码块溢出 ====
% 允许 URL 和长路径在任意字符处断行
% 注意:XeLaTeX 不需要 inputenc/fontenc,breakurl 与 hyperref 不兼容
% \usepackage[utf8]{inputenc}
% \usepackage[T1]{fontenc}
\usepackage{url}
% \usepackage{breakurl}  % 与 XeLaTeX + hyperref 不兼容,已禁用

% 设置 \texttt 允许断行(用于文件路径和代码片段)
\usepackage{seqsplit}
\renewcommand{\texttt}[1]{%
  \begingroup
    \urlstyle{tt}%
    \expandafter\url\expandafter{\detokenize{#1}}%
  \endgroup
}

% 启用紧急断行,避免长单词溢出
\emergencystretch 3em

% 设置段落容差(允许行溢出更宽容)
\tolerance 1000
\hyphenpenalty 5000
\exhyphenpenalty 5000

% ==== 全局微调版面以容纳更多内容 ====
% 减小段落间距
\setlength{\parskip}{0.3em}

% 减小列表项间距
\setlength{\itemsep}{0.2em}

% 允许段落在页边缘附近断行
\setlength{\parfillskip}{0pt}

% 减小左右边距(通过调整 geometry)
% 注意:已在前面设置 margin=0.7in

% 处理 verbatim 环境溢出
\usepackage{fancyvrb}
\RecustomVerbatimEnvironment{verbatim}{Verbatim}{fontsize=\scriptsize}

% ==== 封面页 ====
\newcommand{\makecoverpage}{%
  \begin{titlepage}
    \thispagestyle{coverstyle}
    \centering
    \vspace*{2.0cm}
    {\bfseries\color{brandD}\fontsize{34pt}{36pt}\selectfont \DocTitle\par}
    \vspace{1.2cm}
    {\color{brand}\rule{0.78\textwidth}{0.6pt}}\\[1.0cm]
    {\sffamily\color{keydark}\Large \textbf{\seriesnum}}\\[2.2cm]
    {\sffamily\color{brandD}\Large \SchoolName\par}
    \vspace{2.0cm}
    {\color{brand}\rule{0.78\textwidth}{0.6pt}}\\[1.0cm]
    {\sffamily\color{keydark}\large\BrandYMD\par}
    \vspace*{0.6cm}
  \end{titlepage}%
}

% ==== 标题样式 ====
\usepackage{titlesec}
\titleformat{\section}[block]
  {\bfseries\color{keydark}\fontsize{24pt}{26pt}\selectfont\raggedright}
  {\thesection}{0.6em}{}
\titlespacing*{\section}{0pt}{0.9em}{0.5em}

\titleformat{\subsection}
  {\Large\bfseries\color{brand}\raggedright}
  {\thesubsection}{0.6em}{}
\titlespacing*{\subsection}{0pt}{0.8em}{0.4em}

% ==== 页眉页脚 ====
\usepackage{fancyhdr}
\fancypagestyle{coverstyle}{%
  \fancyhf{}\renewcommand{\headrulewidth}{0pt}\renewcommand{\footrulewidth}{0pt}
}
\fancyhf{}
\renewcommand{\headrulewidth}{0pt}
\renewcommand{\footrulewidth}{0pt}
\renewcommand{\sectionmark}[1]{} % 移除页眉显示
\fancyhead{} % 清空页眉
\fancyfoot{} % 不显示页脚(含页码)
\pagestyle{fancy}
\setlength{\headheight}{15pt} % 避免 fancyhdr 警告

% ==== 系列编号占位 ====
\newcommand{\seriesnum}{Attack Trace Analyzer(ATA)} % 封面副标题

% ==== 目录页面 ====
\renewcommand{\contentsname}{目录}
\makeatletter
\newcommand{\plainTOC}{% 去掉默认节标题导致的额外垂直间距
  \begingroup
  \parskip=0pt \parindent=0pt
  \@starttoc{toc}%
  \endgroup
}
\makeatother

\newcommand{\makeTOCpage}{%
  \clearpage
  \thispagestyle{coverstyle}
  \begin{center}
    \vspace*{0.8cm}
    {\bfseries\color{brandD}\fontsize{24pt}{26pt}\selectfont 目录}\par
    \vspace{0.18cm}
    {\color{brand}\rule{0.78\textwidth}{0.6pt}}\par
    \vspace{0.08cm}
  \end{center}
  \vspace{-0.28cm}
  {\small\plainTOC}
  \clearpage
}

% ==== 信息块 ====
\usepackage[most]{tcolorbox}
\tcbset{enhanced, boxrule=0.9pt, arc=3pt, left=10pt,right=10pt,top=10pt,bottom=10pt}
\newtcolorbox{KeyBox}{
  colback=keybg, colframe=brandD,
  colbacktitle=brandD, title=\textbf{\color{white}要点},
  fonttitle=\bfseries\large, coltitle=white, breakable
}
\newtcolorbox{Takeaway}{
  colback=accent!12, colframe=accentD,
  colbacktitle=accentD, title=\textbf{\color{white}结论},
  fonttitle=\bfseries\large, coltitle=white, breakable
}

% ==== 侧边栏 ====
\usepackage{mdframed}
\newmdenv[
  topline=false, bottomline=false, rightline=false,
  linewidth=4pt, linecolor=brand, backgroundcolor=soft,
  innerleftmargin=10pt, innerrightmargin=10pt,
  skipabove=6pt, skipbelow=6pt
]{SideBar}

% ==== 数学与定理环境 ====
\usepackage{amsmath,amssymb,bm,amsthm}
\DeclareMathOperator{\Var}{Var}
\newcommand{\E}{\mathbb{E}}
\setcounter{secnumdepth}{3} % 确保子小节编号

% 定理样式(按节编号)
\newtheoremstyle{customthm}{0.8em}{0.8em}{\itshape}{}{\bfseries}{.}{0.5em}{}
\newtheoremstyle{customdef}{0.8em}{0.8em}{}{}{\bfseries}{.}{0.5em}{}

\theoremstyle{customthm}
\newtheorem{theorem}{定理}[section]
\newtheorem{lemma}[theorem]{引理}

\theoremstyle{customdef}
\newtheorem{definition}[theorem]{定义}

% --- 定理类环境(tcolorbox 包裹)---
% 颜色:定理=蓝 (secblue/thmBg),引理=红 (accentD/lemBg),定义=绿 (brand/defBg)
% 风格:左侧 2pt 彩条、圆角、统一内边距,标题加粗;[] 为空则不加括号

% 定理环境
\let\oldtheorem\theorem
\let\endoldtheorem\endtheorem
\renewenvironment{theorem}[1][]{%
  \begin{tcolorbox}[
    enhanced,
    breakable,
    colback=thmBg,
    colframe=thmBar,
    boxrule=0pt,
    borderline west={2pt}{0pt}{thmBar},
    sharp corners,
    arc=2pt,
    left=10pt,right=10pt,top=8pt,bottom=8pt,
    before skip=10pt,after skip=10pt
  ]%
  \if\relax\detokenize{#1}\relax
    \begin{oldtheorem}\itshape
  \else
    \begin{oldtheorem}[\textbf{#1}]\itshape
  \fi
}{%
    \end{oldtheorem}%
  \end{tcolorbox}%
}

% 引理环境
\let\oldlemma\lemma
\let\endoldlemma\endlemma
\renewenvironment{lemma}[1][]{%
  \begin{tcolorbox}[
    enhanced,
    breakable,
    colback=lemBg,
    colframe=lemBar,
    boxrule=0pt,
    borderline west={2pt}{0pt}{lemBar},
    sharp corners,
    arc=2pt,
    left=10pt,right=10pt,top=8pt,bottom=8pt,
    before skip=10pt,after skip=10pt
  ]%
  \if\relax\detokenize{#1}\relax
    \begin{oldlemma}\itshape
  \else
    \begin{oldlemma}[\textbf{#1}]\itshape
  \fi
}{%
    \end{oldlemma}%
  \end{tcolorbox}%
}

% 定义环境
\let\olddefinition\definition
\let\endolddefinition\enddefinition
\renewenvironment{definition}[1][]{%
  \begin{tcolorbox}[
    enhanced,
    breakable,
    colback=defBg,
    colframe=defBar,
    boxrule=0pt,
    borderline west={2pt}{0pt}{defBar},
    sharp corners,
    arc=2pt,
    left=10pt,right=10pt,top=8pt,bottom=8pt,
    before skip=10pt,after skip=10pt
  ]%
  \if\relax\detokenize{#1}\relax
    \begin{olddefinition}\itshape
  \else
    \begin{olddefinition}[\textbf{#1}]\itshape
  \fi
}{%
    \end{olddefinition}%
  \end{tcolorbox}%
}

% ==== 章节机制(自定义章号/章题,仅影响编号显示) ====
\newcounter{chapter}
\newcommand{\setchapter}[2]{%
  \setcounter{chapter}{#1}%
  \setcounter{section}{0}%
  \def\ChapterTitle{#2}%
}
\renewcommand{\thesection}{\arabic{chapter}.\arabic{section}}
\renewcommand{\thesubsection}{\arabic{chapter}.\arabic{section}.\arabic{subsection}}
\newcommand{\ChapterHeading}{%
  \vspace*{0.6cm}
  {\centering
    {\fontsize{26pt}{28pt}\selectfont\bfseries\color{keydark}
      第\arabic{chapter}章\ \ChapterTitle\par}
  }%
  \vspace{0.6cm}
}
\newcommand{\BeginChapter}[2]{%
  \clearpage
  \setchapter{#1}{#2}%
  \ChapterHeading
}



% ==== 示例盒子 ====
\definecolor{exframe}{HTML}{6C5CE7}
\definecolor{exbg}{HTML}{F2E9FF}
\newtcolorbox{Example}[1][]{%
  enhanced, colback=exbg, colframe=exframe,
  coltitle=white, colbacktitle=exframe,
  title=\textbf{示例},
  fonttitle=\bfseries, boxrule=1.1pt, arc=4pt,
  left=10pt,right=10pt,top=10pt,bottom=10pt,
  attach boxed title to top left = {yshift=-2mm, xshift=2mm},
  breakable, #1
}

% ==== 证明环境 ====
\renewenvironment{proof}[1][证明]{\par
  \pushQED{\qed}%
  \normalfont \topsep6pt \partopsep0pt
  \trivlist
  \item[\hskip\labelsep\bfseries #1.]\ignorespaces
}{\popQED\endtrivlist}



% ==== 高亮(手绘底色) ====
\usepackage{tikz}
\newcommand{\hltext}[2][yellow!30]{%
  \tikz[baseline=(X.base)] \node[rectangle, rounded corners=2pt,
  fill=#1, inner sep=2pt, anchor=base] (X) {#2};%
}



% ==== 语义化快捷命令 ====
\newcommand{\KeyIdea}[1]{\hltext[keybg]{\textbf{#1}}}            % 关键想法:浅绿底
\newcommand{\Term}[1]{\textcolor{termblue}{\textbf{#1}}}           % 专业名词:品牌深绿字
\newcommand{\Emph}[1]{\textcolor{accentD}{\textbf{#1}}}          % 强调词汇:红
\newcommand{\Confuse}[1]{\hltext[error!20]{\textcolor{error}{#1}}} % 疑问或存疑处

\newcommand{\Info}[1]{\textcolor{info!90!black}{#1}}             % 说明或注解

% ========================= 文档开始 =========================
\begin{document}

% 标题页(信息由上述命令给出,可根据需要修改)
\makecoverpage
\setcounter{page}{1}

% 目录页
\makeTOCpage

% 使用标准 section 编号(用于 pandoc 生成正文)
% 注意:本模板为了"章号"效果重写了 \thesection,这里显式还原为标准编号,避免出现 0.1 之类的编号。
\renewcommand{\thesection}{\arabic{section}}
\renewcommand{\thesubsection}{\thesection.\arabic{subsection}}
\renewcommand{\thesubsubsection}{\thesubsection.\arabic{subsubsection}}

% 自动生成正文(chapters/*.tex → index.tex → \input)
\IfFileExists{index.tex}{%
  \section*{概要设计报告}
\addcontentsline{toc}{section}{概要设计报告}

\begin{SideBar}
\textbf{文档定位:}本报告用于描述 Attack Trace Analyzer(ATA)的总体架构与关键机制,回答"系统由哪些模块构成、模块如何协作、数据如何流转、关键设计为何这样取舍"。\\
\textbf{与《详细设计报告》的关系:}本报告强调"模块边界 + 数据/控制流";细节(字段、索引治理、图谱 schema、任务状态机实现)请参阅《详细设计报告》与对应代码目录。
\end{SideBar}

\begin{KeyBox}
\textbf{一页总结(系统核心要点):}
\begin{itemize}
  \item \textbf{多源采集(客户端):}Falco(主机行为)、Filebeat(+Sigma)(主机日志)、Suricata(网络流量)统一转换为 ECS 子集字段。
  \item \textbf{中心机流水线:}采用"单定时器顺序流水线"模型:轮询拉取增量数据 $\rightarrow$ Store-first 入 OpenSearch $\rightarrow$ 检测融合(Raw $\rightarrow$ Canonical)$\rightarrow$ ECS 入图写入 Neo4j。
  \item \textbf{双存储分工:}OpenSearch 负责检索/检测/去重/索引治理;Neo4j 负责实体关系图、时间窗内关系遍历与溯源结果写回(边属性 \texttt{analysis.*})。
  \item \textbf{溯源任务模型:}前端/接口触发异步 Trace Task(目标节点 + 时间窗),后台计算后写回图谱,并提供任务状态与结果查询。
  \item \textbf{可解释性原则:}每条溯源结论必须可回链到证据事件(如 \texttt{event.id}、\texttt{custom.evidence.event\_ids}),保证可复核。
\end{itemize}
\end{KeyBox}

\clearpage

\section{概要设计报告}\label{outline-design-report}

本文件定义系统的\textbf{总体架构与关键机制},重点包括:

\begin{itemize}
\tightlist
\item
  中心机单定时器顺序流水线架构
\item
  前端可视化查询链路
\item
  面向教师交互的异步溯源任务模型
\item
  三大模块(OpenSearch、Neo4j、Analysis)的边界与协作方式
\end{itemize}

本文件不重复以下文档的细节:

\begin{itemize}
\tightlist
\item
  字段口径:\texttt{../80-规范/81-ECS字段规范.md}
\item
  图谱口径:\texttt{../80-规范/84-Neo4j实体图谱规范.md}
\item
  环境变量:\texttt{../80-规范/89-环境变量与配置规范.md}
\item
  客户机与中心机接口:\texttt{../80-规范/87-客户机与中心机接口.md}
\item
  OpenSearch 存储:\texttt{../50-详细设计/中心机/}\\ \texttt{62-OpenSearch存储与索引治理.md}
\item
  Neo4j 入图与图查询:\texttt{../50-详细设计/中心机/}\\ \texttt{64-Neo4j入图与图查询.md}
\item
  Analysis 任务模型:\texttt{../50-详细设计/分析/}\\ \texttt{70-任务模型与状态机.md}
\end{itemize}

\subsection{总体架构}\label{overall-architecture}

\subsubsection{术语与缩略语}\label{terminology}

为保证文档表述的一致性,本节定义核心术语:

{\def\LTcaptype{none} % do not increment counter
\small
\begin{longtable}[]{@{}p{0.2\textwidth}p{0.25\textwidth}p{0.45\textwidth}@{}}
\toprule\noalign{}
术语 & 英文 & 定义 \\
\midrule\noalign{}
\endhead
\bottomrule\noalign{}
\endlastfoot
\textbf{客户机(Client)} & Client & 部署在被监控主机上的数据采集与转换节点,采集 Falco、Filebeat、Suricata 三类数据源,转换为 ECS 格式并缓冲在本地队列 \\
\textbf{中心机(Center)} & Center Server & 汇聚数据、执行检测与分析的核心服务节点,承担轮询拉取、入库、检测融合、入图、对外 API 与异步溯源任务 \\
\textbf{Telemetry} & Telemetry & 遥测数据,指未经检测规则触发的原始观测事件,\texttt{event.kind="event"} \\
\textbf{Raw Finding} & Raw Finding & 由单一检测规则(如 Sigma 规则、Security Analytics 检测器)产生的初步告警,\texttt{event.kind="alert"} 且 \texttt{dataset!="canonical"} \\
\textbf{Canonical Finding} & Canonical Finding & 经融合去重后的标准化告警,\texttt{event.kind="alert"} 且 \texttt{dataset="finding.canonical"},用于图谱入图与溯源分析主输入 \\
\textbf{溯源任务(Trace Task)} & Trace Task & 以目标节点为起点、时间窗为边界的溯源分析任务,异步执行并在完成后将结果写回 Neo4j 边属性 \\
\textbf{TTP} & Tactics, Techniques, and Procedures & 战术、技术与过程,指 MITRE ATT\&CK 框架中对攻击行为的分类与描述 \\
\textbf{ECS} & Elastic Common Schema & Elastic 公司定义的通用事件字段规范,用于统一不同数据源的字段命名 \\
\textbf{实体图谱} & Entity Graph & 以 Host、User、Process、File、IP、Domain 等实体为节点、以事件为边构成的属性图,存储于 Neo4j \\
\textbf{关键路径} & Critical Path & 溯源任务计算出的攻击传播路径,由边属性 \texttt{analysis.is\_path\_edge=true} 标记 \\
\textbf{顺序流水线} & Sequential Pipeline & 中心机使用单个定时器串行执行轮询、存储、检测、入图四个步骤的架构模式 \\
\end{longtable}
}

\subsubsection{系统组成}\label{system-components}

系统由客户机侧和中心机侧组成:

\begin{itemize}
\tightlist
\item
  客户机侧:采集、转换、本地缓冲、对外提供拉取接口
\item
  中心机侧:定时拉取、入库、检测、融合、入图、对外 API、异步溯源与报告
\end{itemize}

\subsubsection{系统架构图(逻辑视图)}\label{architecture-logic-view}

\begin{figure}[htbp]
\centering
\includegraphics[width=\textwidth,height=0.5\textheight,keepaspectratio]{figures/outline-40-01.pdf}
\caption{系统架构图}
\label{fig:outline-40-01}
\end{figure}

\subsection{核心数据分层(中心机视角)}\label{data-layers}

\subsubsection{数据分层架构}\label{data-layer-architecture}

系统采用以下数据层次(从底到顶):

\begin{figure}[htbp]
\centering
\includegraphics[width=\textwidth,height=0.5\textheight,keepaspectratio]{figures/outline-40-02.pdf}
\caption{数据分层架构}
\label{fig:outline-40-02}
\end{figure}

\subsubsection{数据层次说明}\label{data-layer-description}

{\def\LTcaptype{none} % do not increment counter
\small
\begin{longtable}[]{@{}p{0.12\textwidth}p{0.22\textwidth}p{0.25\textwidth}p{0.2\textwidth}p{0.15\textwidth}@{}}
\toprule\noalign{}
层次 & 名称 & 判定条件 & 存储位置 & 用途 \\
\midrule\noalign{}
\endhead
\bottomrule\noalign{}
\endlastfoot
\textbf{Layer 1} & Telemetry(事实层) & \texttt{event.kind="event"} & OpenSearch \texttt{ecs-events-*} & 原始事件检索、检测输入 \\
\textbf{Layer 2} & Raw Findings(原始告警层) & \texttt{event.kind="alert"} 且 \texttt{dataset!="canonical"} & OpenSearch \texttt{raw-findings-*} & 原始告警审计、融合输入 \\
\textbf{Layer 3} & Canonical Findings(规范告警层) & \texttt{event.kind="alert"} 且 \texttt{dataset="canonical"} & OpenSearch \texttt{canonical-findings-*} & 图谱与溯源主输入 \\
\textbf{Layer 4} & Entity Graph(图谱层) & Telemetry + Canonical 转换 & Neo4j 节点/边 & 图查询、路径分析 \\
\textbf{Layer 5} & Trace Task(溯源任务层) & 异步任务创建 & OpenSearch \texttt{analysis-tasks-*} + Neo4j 边属性 & 任务状态、溯源结果 \\
\end{longtable}
}

\begin{quote}
说明:攻击链作为展示结构,由图谱与任务结果共同承载。前端展示时以图谱边序列及边属性为主,不依赖单独的链路索引。
\end{quote}

\subsection{中心机定时流水线(单定时器顺序流水线)}\label{sequential-pipeline}

\subsubsection{调度原则}\label{scheduling-principles}

\begin{itemize}
\tightlist
\item
  中心机必须使用\textbf{单一}定时器触发流水线
\item
  定时器周期由环境变量 \texttt{CENTER\_POLL\_INTERVAL\_SECONDS} 控制,默认值为 \texttt{5} 秒
\item
  同一时间只允许存在一个流水线实例运行:当上一次流水线未结束时,新的定时触发必须被跳过,避免并发写入导致的数据漂移与重复工作
\end{itemize}

\subsubsection{设计决策:为什么采用单定时器顺序流水线架构}\label{sequential-pipeline-decision}

\textbf{决策内容}:采用单个定时器串行执行轮询、存储、检测、入图四个步骤,而非并发或事件驱动架构

\textbf{决策依据}:

\begin{enumerate}
\def\labelenumi{\arabic{enumi}.}
\tightlist
\item
  \textbf{数据一致性保证}:串行执行避免了并发写入导致的事件顺序错乱,确保 Telemetry → Raw → Canonical → Graph 的严格因果顺序
\item
  \textbf{资源可控性}:单线程模型使 CPU、内存、网络带宽消耗可预测,避免高并发场景下的资源竞争
\item
  \textbf{故障排查简化}:串行流水线的调用链清晰,便于通过日志追踪问题根因
\item
  \textbf{实现简洁性}:避免引入分布式锁、消息队列去重等复杂机制
\end{enumerate}

\textbf{代价与权衡}:

\begin{itemize}
\tightlist
\item
  吞吐量上限受限于单线程处理速度,预计可支撑约 10 个客户机规模
\item
  单步骤故障会导致后续步骤阻塞,需通过快速失败与错误隔离缓解
\end{itemize}

\subsubsection{流水线输入与输出}\label{pipeline-inputs-outputs}

\begin{itemize}
\tightlist
\item
  输入:所有已注册客户机的新数据(由轮询拉取接口返回,格式为 ECS 文档)
\item
  输出:

  \begin{itemize}
  \tightlist
  \item
    OpenSearch:Telemetry、Raw Findings、Canonical Findings、Tasks
  \item
    Neo4j:Entity Graph(含溯源任务写回的边属性)
  \end{itemize}
\end{itemize}

\subsubsection{四步顺序与职责(每次 tick 内严格顺序执行)}\label{four-steps-sequential}

\paragraph{Step 1:从客户机拉取数据}\label{step-1-poll-data}

对每个已注册客户机:

\begin{enumerate}
\def\labelenumi{\arabic{enumi})}
\tightlist
\item
  中心机读取注册表条目(含 \texttt{listen\_url}、\texttt{client\_token\_hash} 等元数据)
\item
  调用客户机拉取接口(见 \texttt{../80-规范/87-客户机与中心机接口.md})
\item
  客户机从 RabbitMQ 队列中取出消息并返回 ECS 文档列表:

  \begin{itemize}
  \tightlist
  \item
    队列为空则返回空数组
  \item
    拉取行为会消费队列中的消息(队列语义保证增量)
  \end{itemize}
\end{enumerate}

\paragraph{Step 2:写入 OpenSearch(字段处理)}\label{step-2-write-opensearch}

中心机必须对每条 ECS 文档执行字段处理(细节见 \texttt{../50-详细设计/中心机/62-OpenSearch存储与索引治理.md} 与 \texttt{../80-规范/81-ECS字段规范.md}):

\begin{itemize}
\tightlist
\item
  三时间字段补齐与覆盖规则
\item
  \texttt{event.kind} 与 \texttt{event.dataset} 校验
\item
  \texttt{event.id} 补齐与幂等去重
\item
  按 \texttt{event.kind/event.dataset} 路由写入对应索引
\end{itemize}

\paragraph{Step 3:Store-first 检测 + Raw→Canonical 融合}\label{step-3-detection-fusion}

中心机必须在每个 tick 内执行:

\begin{enumerate}
\def\labelenumi{\arabic{enumi})}
\tightlist
\item
  触发 Store-first 检测:由 OpenSearch Security Analytics 对 Telemetry 索引执行扫描
\item
  读取检测产生的 Findings,并转换为 ECS Finding 文档,写入 \texttt{raw-findings-*}
\item
  对指定时间窗内的 Raw Findings 执行融合去重,生成 Canonical Findings,写入 \texttt{canonical-findings-*}
\end{enumerate}

融合去重的指纹规则、provider 合并规则与幂等规则由 \texttt{../50-详细设计/中心机/62-OpenSearch存储与索引治理.md} 定义。

\begin{quote}
实现口径(与当前代码对齐):Step 3 在每个 tick 内固定执行(不提供开关),用于持续产出 Raw→Canonical 的规范告警层
\end{quote}

\paragraph{Step 4:触发 ECS→Graph 写入 Neo4j}\label{step-4-write-neo4j}

中心机必须在每个 tick 内执行入图:

\begin{enumerate}
\def\labelenumi{\arabic{enumi})}
\tightlist
\item
  以 Canonical Findings 为主,补充必要的 Telemetry(用于补边与证据回溯)
\item
  对每条输入文档执行 ECS→Graph 转换
\item
  将节点/边写入 Neo4j,并满足 \texttt{../80-规范/84-Neo4j实体图谱规范.md} 的唯一键与边属性规范
\item
  边必须写入 \texttt{ts\_float}(数值时间戳),支撑时间窗查询与图算法投影
\end{enumerate}

\begin{quote}
实现口径(与当前代码对齐):Step 4 在每个 tick 内固定执行;入图以 \texttt{event.ingested}(中心机入库时间)为窗口边界,确保本 tick 生成的数据不会遗漏
\end{quote}

\subsubsection{流水线时序图}\label{pipeline-timing-diagram}

\begin{figure}[htbp]
\centering
\includegraphics[width=\textwidth,height=0.5\textheight,keepaspectratio]{figures/outline-40-03.pdf}
\caption{流水线时序图}
\label{fig:outline-40-03}
\end{figure}

流水线按照以下严格顺序执行(每个 tick 内):

\begin{enumerate}
\def\labelenumi{\arabic{enumi})}
\item \textbf{Step 1: 拉取数据} - 轮询服务从客户机集群拉取 ECS 事件
\item \textbf{Step 2: 写入 OpenSearch} - 批量存储事件,字段处理与去重
\item \textbf{Step 3: 检测与融合} - Store-first 检测生成 Raw Findings,融合去重生成 Canonical Findings
\item \textbf{Step 4: 入图 Neo4j} - 将 Canonical Findings 和 Telemetry 批量写入图谱
\end{enumerate}

整个流程由定时器(5s 周期)触发,单线程串行执行,确保数据一致性。

\subsection{前端可视化查询链路}\label{frontend-visualization}

\subsubsection{查询边界}\label{query-boundaries}

\begin{itemize}
\tightlist
\item
  Telemetry 与 Findings 的检索由 OpenSearch 提供
\item
  图谱可视化(节点/边)与路径查询由 Neo4j 提供
\item
  前端不得直连数据库,只通过后端 API 访问
\end{itemize}

\subsubsection{图可视化最小闭环}\label{graph-visualization-loop}

\begin{enumerate}
\def\labelenumi{\arabic{enumi})}
\tightlist
\item
  前端请求后端图查询接口(动作包含:时间窗边查询、告警边查询、时间窗最短路)
\item
  后端调用 Neo4j 模块,返回图数据(nodes/edges)
\item
  前端渲染图,允许教师点击节点查看其属性与相关边的证据引用
\end{enumerate}

对应 API 的对外形状由后端路由定义(实现已存在:\texttt{/api/v1/graph/query}),模块内部规则见 \texttt{../50-详细设计/中心机/64-Neo4j入图与图查询.md}。

\subsection{异步溯源任务(create\_task)}\label{async-trace-tasks}

\subsubsection{为什么必须异步}\label{why-async}

节点溯源可能涉及:

\begin{itemize}
\tightlist
\item
  子图扩展与多跳查询
\item
  图算法(最短路、风险权重路径等)
\item
  解释性生成(TTP 解释与文本说明)
\end{itemize}

这些步骤耗时不稳定,必须以异步任务执行,并对前端提供可轮询的进度状态。

\subsubsection{任务模型(状态机)}\label{task-state-machine}

每个溯源任务在 OpenSearch \texttt{analysis-tasks-*} 里保存一条任务文档,任务状态只能取以下值之一:

\begin{figure}[htbp]
\centering
\includegraphics[width=\textwidth,height=0.5\textheight,keepaspectratio]{figures/outline-40-04.pdf}
\caption{任务状态机}
\label{fig:outline-40-04}
\end{figure}

\textbf{状态说明}:

\begin{itemize}
\tightlist
\item
  \texttt{queued}:已创建,等待执行
\item
  \texttt{running}:执行中
\item
  \texttt{succeeded}:已完成
\item
  \texttt{failed}:失败(含错误信息)
\end{itemize}

\textbf{状态转移约束}:

\begin{itemize}
\tightlist
\item
  只允许 \texttt{queued\ →\ running\ →\ succeeded/failed} 单向转移
\item
  禁止回退与跳转(如 \texttt{running\ →\ queued}、\texttt{queued\ →\ succeeded})
\end{itemize}

\subsubsection{创建、执行与结果写回}\label{task-execution-writeback}

\begin{enumerate}
\def\labelenumi{\arabic{enumi})}
\tightlist
\item
  前端在图上选定目标节点后,请求后端创建任务
\item
  后端立即返回 \texttt{task\_id}
\item
  Analysis 模块异步执行任务:从 Neo4j 读取数据、执行算法、得到关键边集合、风险评分、Technique/Tactic 摘要及解释文本
\item
  Analysis 模块把结果写回 Neo4j 边属性,形成被标注的图
\item
  前端轮询任务状态,任务完成后再次请求图查询接口,读取边属性并展示溯源结果
\end{enumerate}

对应后端 API(固定):

\begin{itemize}
\tightlist
\item
  创建任务:\texttt{POST\ /api/v1/analysis/tasks}
\item
  查询任务状态:\texttt{GET\ /api/v1/analysis/tasks/\{task\_id\}}
\item
  拉取写回边:\texttt{POST /api/v1/graph/query}\\ \hspace*{1.5em}(\texttt{analysis\_edges\_by\_task} 或 \texttt{edges\_in\_window} + 前端过滤)
\end{itemize}

写回字段命名、覆盖规则与数据类型由 \texttt{../50-详细设计/分析/70-任务模型与状态机.md} 与 \texttt{../50-详细设计/中心机/64-Neo4j入图与图查询.md} 定义。

\subsubsection{KillChain 分析(内部实现)}\label{killchain-analysis}

Analysis 模块的溯源任务内部使用 KillChain 算法重建攻击路径。

KillChain 算法基于 MITRE ATT\&CK 战术分段,通过有限状态自动机 (FSA) 识别攻击阶段,并使用大语言模型 (LLM) 选择最合理的段间连接路径,生成完整攻击链解释。

\textbf{核心能力}:

\begin{enumerate}
\def\labelenumi{\arabic{enumi}.}
\tightlist
\item
  \textbf{战术分段}:基于 MITRE ATT\&CK 战术将攻击事件自动分段(Initial Access → Execution → Privilege Escalation → Lateral Movement → C2 → Impact)
\item
  \textbf{路径重建}:识别各战术阶段之间的连接路径
\item
  \textbf{智能选择}:使用 LLM 从候选路径中选择最合理的攻击链
\item
  \textbf{可解释性}:生成 10-20 句中文的全链路解释,包含主谓宾结构
\item
  \textbf{置信度评估}:输出可信度评分(0.0-1.0)
\end{enumerate}

\textbf{输出结果}:

\begin{itemize}
\tightlist
\item
  \texttt{segments{[}{]}}:战术分段列表,每个段包含状态、时间范围、锚点、异常边摘要
\item
  \texttt{selected\_paths{[}{]}}:段间连接路径列表
\item
  \texttt{explanation}:LLM 生成的完整攻击链解释
\item
  \texttt{confidence}:可信度评分
\end{itemize}

\textbf{持久化方式}:

\begin{itemize}
\tightlist
\item
  Neo4j 边属性:\texttt{custom.killchain.uuid}(ECS 合规字段)
\item
  任务文档:\texttt{task.result.killchain\_uuid} 和 \texttt{task.result.killchain}
\end{itemize}

详细设计见:\texttt{50-详细设计/分析/69-KillChain概览设计.md} 与 \texttt{74-KillChain结果展示规范.md}

\subsection{安全与审计边界(演示)}\label{security-audit}

\begin{itemize}
\tightlist
\item
  运行环境:靶场内网。中心机所有服务与数据库端口只对靶场内可达网络开放
\item
  CTI 数据:使用离线 ATT\&CK Enterprise CTI 数据包;运行时不依赖外网
\item
  数据留痕:所有结论必须能回溯到 \texttt{event.id},并能定位到数据源类型与时间窗范围(证据链)
\end{itemize}

\section{数据流与时序}\label{data-flow-timing}

系统涉及五类核心数据对象:

\begin{enumerate}
\def\labelenumi{\arabic{enumi}.}
\tightlist
\item
  \textbf{Telemetry(遥测数据)}:事实事件(ECS 格式,\texttt{event.kind="event"}),\\
  \hspace*{1.5em}存储于 OpenSearch \texttt{ecs-events-*} 索引。
\item
  \textbf{Raw Finding(原始告警)}:原始告警(ECS 格式,\texttt{event.kind="alert"} 且非 canonical),\\
  \hspace*{1.5em}存储于 OpenSearch \texttt{raw-findings-*} 索引。
\item
  \textbf{Canonical Finding(规范告警)}:规范告警(ECS 格式,\texttt{event.kind="alert"} 且\\
  \hspace*{1.5em}\texttt{event.dataset="finding.canonical"}),存储于 OpenSearch \texttt{canonical-findings-*} 索引。
\item
  \textbf{Entity Graph(实体关系图)}:实体关系图(Neo4j),输入数据为 Telemetry 与 Canonical Finding。
\item
  \textbf{Trace Task(溯源任务)}:溯源任务(OpenSearch 任务索引 + Neo4j 边属性写回)。
\end{enumerate}

\subsection{端到端数据流(中心机单定时器流水线)}\label{end-to-end-data-flow}

\subsubsection{客户机侧(每台主机)}\label{client-side-flow}

\begin{enumerate}
\def\labelenumi{\arabic{enumi})}
\tightlist
\item
  \textbf{数据采集}:传感器(Filebeat / Falco / Suricata)采集并输出原始数据。
\item
  \textbf{格式转换}:将采集结果转换为 ECS 标准文档(字段规范见 \texttt{../80-规范/81-ECS字段规范.md})。
\item
  \textbf{队列缓冲}:将 ECS 文档写入本地 RabbitMQ 队列进行缓冲。
\item
  \textbf{接口暴露}:对外提供数据拉取接口,从队列中取出数据并返回,供中心机轮询调用(接口规范见 \texttt{../80-规范/87-客户机与中心机接口.md})。
\end{enumerate}

\subsubsection{中心机侧(每次 tick,严格顺序)}\label{center-side-flow}

中心机定时器每次触发时,均按以下顺序执行:

\begin{enumerate}
\def\labelenumi{\arabic{enumi})}
\tightlist
\item
  \textbf{数据拉取}:从所有已注册客户机拉取新增数据。
\item
  \textbf{数据入库}:将数据写入 OpenSearch(包括 Telemetry/Raw Findings 路由、三时间字段处理及幂等去重)。
\item
  \textbf{检测与融合}:通过 Store-first 检测机制产出 Raw Findings,并融合生成 Canonical Findings 后写回 OpenSearch。
\item
  \textbf{图谱构建}:以 Canonical Findings 为主体,辅以必要的 Telemetry 数据,转换为图结构并写入 Neo4j。
\end{enumerate}

上述四个步骤由同一定时器驱动,形成"顺序流水线"架构,详细系统规格见 \texttt{40-概要设计报告.md}。

\begin{quote}
\textbf{实现细节}(与当前代码对齐):

\begin{itemize}
\tightlist
\item
  后端在每个 tick 固定执行 Step 1/2/3/4,构成"单定时器顺序流水线"架构;
\item
  Step 4 入图时以本 tick 的 \texttt{event.ingested} 时间窗为边界,确保"本 tick 生成的 Canonical Finding"不会因 \texttt{@timestamp} 较早而遗漏。
\end{itemize}
\end{quote}

\subsection{前端可视化数据流}\label{frontend-visualization-flow}

\begin{enumerate}
\def\labelenumi{\arabic{enumi})}
\tightlist
\item
  \textbf{页面访问}:用户打开前端页面。
\item
  \textbf{数据请求}:前端向后端发起请求:

  \begin{itemize}
  \tightlist
  \item
    查询事件/告警:后端查询 OpenSearch;
  \item
    查询图谱:后端查询 Neo4j。
  \end{itemize}
\item
  \textbf{页面渲染}:前端进行可视化渲染:

  \begin{itemize}
  \tightlist
  \item
    事件时间线(数据来源:OpenSearch);
  \item
    实体关系图谱(数据来源:Neo4j 返回的 nodes/edges)。
  \end{itemize}
\end{enumerate}

\subsection{溯源任务数据流}\label{trace-task-flow}

\begin{enumerate}
\def\labelenumi{\arabic{enumi})}
\tightlist
\item
  \textbf{节点选择}:用户在图谱上选定一个节点(node uid)。
\item
  \textbf{任务创建}:前端请求后端创建溯源任务,后端立即返回 \texttt{task\_id}。
\item
  \textbf{异步分析}:Analysis 模块异步执行:读取 Neo4j 子图 → 执行算法分析 → 生成关键路径与解释。
\item
  \textbf{结果写回}:Analysis 模块将分析结果写回 Neo4j 边属性(字段规范见 \texttt{32/33})。
\item
  \textbf{结果展示}:前端轮询任务状态,任务完成后再次请求图查询接口,读取边属性并展示溯源结果。
\item
  \textbf{报告导出}:前端导出报告,包含告警信息、图谱与溯源结果。
\end{enumerate}

\section{部署拓扑与网络规划}\label{deployment-network-planning}

本文档定义项目靶场的节点拓扑、网络规划、端口分配与访问边界,确保现场演示的部署一致性与可复现性。

\subsection{读者对象}\label{outline-42-audience}

\begin{itemize}
\tightlist
\item
  靶场负责人
\item
  部署与编排人员
\item
  演示与验收人员
\end{itemize}

\subsection{引用关系}\label{outline-42-references}

\begin{itemize}
\tightlist
\item
  详细部署步骤:\texttt{../90-运维与靶场/91-靶场部署.md}
\item
  一键编排说明:\texttt{../90-运维与靶场/92-一键编排.md}
\item
  接口与端口规范:\texttt{../80-规范/87-客户机与中心机接口.md}
\end{itemize}

\subsection{节点与角色}\label{nodes-roles}

靶场节点角色划分如下:

\begin{itemize}
\tightlist
\item
  \texttt{center}:中心机(部署 OpenSearch、Neo4j、后端 API 与前端服务)
\item
  \texttt{client-01} 至 \texttt{client-04}:客户机节点(负责数据采集与缓冲)
\item
  \texttt{c2}:命令与控制服务器(用于生成可观测通信证据)
\end{itemize}

\subsection{网络规划}\label{network-planning}

网络规划遵循以下原则:

\begin{enumerate}
\def\labelenumi{\arabic{enumi}.}
\tightlist
\item
  网络隔离:靶场网络与宿主机外网隔离,所有演示数据在靶场内闭环流转。
\item
  端口暴露:中心机对外端口仅面向靶场网络可达。
\item
  服务隔离:客户机仅向中心机开放数据拉取接口,不对外提供其他服务端口。
\end{enumerate}

\subsection{端口规划}\label{port-planning}

端口分配与用途如下:

{\def\LTcaptype{none} % do not increment counter
\small
\begin{longtable}[]{@{}lrll@{}}
\toprule\noalign{}
组件 & 端口 & 协议 & 访问控制 \\
\midrule\noalign{}
\endhead
\bottomrule\noalign{}
\endlastfoot
中心机后端 & 8001 & HTTP & 仅内网 \\
中心机前端 & 3000 & HTTP & 公网(可选) \\
OpenSearch & 9200 & HTTP & 仅本地 \\
Neo4j Bolt & 7687 & Bolt & 仅本地 \\
Neo4j Browser & 7474 & HTTP & 仅本地 \\
客户机拉取接口 & 8888 & HTTP & 仅内网 \\
\end{longtable}
}

\subsection{部署拓扑}\label{deployment-topology}

部署拓扑遵循"中心机集中、客户机分布"的架构:

\begin{enumerate}
\def\labelenumi{\arabic{enumi}.}
\tightlist
\item
  客户机采集数据并缓冲至本地队列
\item
  中心机周期性轮询拉取客户机数据
\item
  中心机将数据写入 OpenSearch,并根据需要触发检测与告警融合
\item
  中心机将数据入图至 Neo4j,前端通过中心机 API 查询与展示
\end{enumerate}

\subsection{安全边界与隔离}\label{security-boundaries-isolation}

\begin{enumerate}
\def\labelenumi{\arabic{enumi}.}
\tightlist
\item
  代码隔离:中心机不运行未知样本,不执行来自外部的不可信代码。
\item
  访问控制:靶场网络中所有组件的访问鉴权与令牌由接口规范统一定义。
\item
  运维保障:现场演示以稳定性为首要目标,任何需要人工介入的步骤必须写入运维文档。
\end{enumerate}

\section{非功能设计}\label{non-functional-design}

本文档将需求中的非功能要求转化为确定的工程设计约束,涵盖幂等性、一致性、可解释性、性能、稳定性与可复现性六个方面。

\subsection{读者对象}\label{outline-43-audience}

\begin{itemize}
\tightlist
\item
  系统架构与后端实现人员
\item
  测试与验收人员
\end{itemize}

\subsection{引用关系}\label{outline-43-references}

\begin{itemize}
\tightlist
\item
  需求分析:\texttt{../20-需求与验收/20-需求分析报告.md}
\item
  数据与规范:\texttt{../80-规范/}
\item
  测试报告:\texttt{../96-测试/96-测试分析报告.md}
\end{itemize}

\subsection{幂等与去重}\label{idempotency-deduplication}

\subsubsection{OpenSearch 幂等}\label{opensearch-idempotency}

系统以 \texttt{event.id} 作为全局幂等键:

\begin{itemize}
\tightlist
\item
  同一 \texttt{event.id} 重复写入 OpenSearch 不产生重复文档
\item
  \texttt{event.id} 的生成规则由 \texttt{../80-规范/81-ECS字段规范.md} 定义
\end{itemize}

\subsubsection{图谱幂等}\label{graph-idempotency}

Neo4j 节点写入通过唯一键约束保证幂等;边写入携带证据引用并提供可过滤的去重机制,具体规范见:

\begin{itemize}
\tightlist
\item
  \texttt{../80-规范/84-Neo4j实体图谱规范.md}
\item
  \texttt{../80-规范/85-溯源结果写回规范.md}
\end{itemize}

\subsection{一致性与可回放}\label{consistency-replayability}

\subsubsection{数据规模假设}\label{data-scale-assumptions}

系统设计与性能目标基于以下数据规模假设制定:

{\def\LTcaptype{none} % do not increment counter
\begin{longtable}[]{@{}lll@{}}
\toprule\noalign{}
指标 & 规模 & 说明 \\
\midrule\noalign{}
\endhead
\bottomrule\noalign{}
\endlastfoot
单客户机每分钟事件量 & \textasciitilde1000 条 & 三传感器(Falco/Filebeat/Suricata)合计 \\
单客户机每天事件量 & \textasciitilde144 万条 & 按 24 小时连续运行计算 \\
10 客户机集群每天事件量 & \textasciitilde1440 万条 & 典型部署规模 \\
图谱节点规模(30天窗口) & \textasciitilde10 万节点 & Host/User/Process/File/IP/Domain 等实体 \\
图谱边规模(30天窗口) & \textasciitilde50 万边 & 包含时间属性的关系边 \\
Raw Findings 每天生成量 & \textasciitilde1000 条 & 基于 Sigma 与 Security Analytics 检测 \\
Canonical Findings 每天生成量 & \textasciitilde100 条 & 经融合去重后的规范告警 \\
单次溯源任务查询时间窗 & 1-60 分钟 & 用户可选的时间窗范围 \\
\end{longtable}
}

\subsubsection{一致性目标}\label{consistency-objectives}

系统需满足以下一致性要求:

\begin{enumerate}
\def\labelenumi{\arabic{enumi}.}
\tightlist
\item
  同一批输入事件在重复拉取与重复执行下,OpenSearch 与 Neo4j 的关键输出保持一致。
\item
  任意告警与溯源结论均可回溯到 Telemetry 的 \texttt{event.id} 证据。
\end{enumerate}

\subsection{可解释与证据链}\label{explainability-evidence}

系统需满足以下可解释性要求:

\begin{itemize}
\tightlist
\item
  前端展示的告警、关系边与溯源结果必须包含证据引用
\item
  溯源写回字段使用统一前缀 \texttt{analysis.},\\
  字段集合与覆盖规则由 \texttt{../80-规范/85-溯源结果写回规范.md} 定义
\end{itemize}

\subsection{性能目标与容量边界}\label{performance-capacity}

系统需满足以下性能要求:

\begin{itemize}
\tightlist
\item
  图查询应在秒级内返回满足可视化展示所需的数据量
\item
  溯源任务允许长耗时,但必须提供可轮询的状态与进度信息
\end{itemize}

容量边界与数据保留策略详见:

\begin{itemize}
\tightlist
\item
  \texttt{../80-规范/80-数据对象与生命周期.md}
\end{itemize}

\subsubsection{前端查询性能}\label{frontend-query-performance}

{\def\LTcaptype{none} % do not increment counter
\begin{longtable}[]{@{}lll@{}}
\toprule\noalign{}
指标 & 目标值 & 测量方法 \\
\midrule\noalign{}
\endhead
\bottomrule\noalign{}
\endlastfoot
图查询响应时间 & \textless3 秒 & 实际负载测试 \\
事件检索响应时间 & \textless1 秒 & 实际负载测试 \\
页面首次渲染 & \textless2 秒 & 浏览器性能测试 \\
\end{longtable}
}

\subsubsection{中心机轮询性能}\label{center-polling-performance}

{\def\LTcaptype{none} % do not increment counter
\begin{longtable}[]{@{}lll@{}}
\toprule\noalign{}
指标 & 目标值 & 测量方法 \\
\midrule\noalign{}
\endhead
\bottomrule\noalign{}
\endlastfoot
单次轮询耗时 & \textless10 秒 & 轮询日志统计 \\
单轮拉取吞吐量 & \textgreater1000 evt/s & 轮询日志统计 \\
数据积压恢复时间 & \textless5 分钟 & 积压场景测试 \\
\end{longtable}
}

\subsubsection{Neo4j 入图性能}\label{neo4j-ingestion-performance}

{\def\LTcaptype{none} % do not increment counter
\begin{longtable}[]{@{}lll@{}}
\toprule\noalign{}
指标 & 目标值 & 测量方法 \\
\midrule\noalign{}
\endhead
\bottomrule\noalign{}
\endlastfoot
100 事件入图耗时 & \textless1 秒 & 性能测试 \texttt{test\_batch\_ingest\_performance} \\
1000 事件入图耗时 & \textless5 秒 & 性能测试 \texttt{test\_batch\_ingest\_performance} \\
单事件平均网络往返 & \textless2 次 & 通过批量 UNWIND MERGE 实现 \\
批量写入吞吐量 & \textgreater100 evt/s & 实际负载测试 \\
\end{longtable}
}

\subsection{稳定性与故障处理}\label{stability-fault-handling}

\subsubsection{故障处理原则}\label{fault-handling-principles}

系统故障处理遵循以下原则:

\begin{enumerate}
\def\labelenumi{\arabic{enumi}.}
\tightlist
\item
  单次轮询失败不影响后续轮询继续执行
\item
  LLM 调用失败必须走固定回退路径,确保演示不因外部服务中断而失败
\item
  数据重置与复现流程固定,详见 \texttt{../90-运维与靶场/95-重置复现与排障.md}
\end{enumerate}

\subsubsection{失败场景与处理策略}\label{failure-scenarios-strategies}

{\def\LTcaptype{none} % do not increment counter
\small
\begin{longtable}[]{@{}p{0.18\textwidth}p{0.18\textwidth}p{0.28\textwidth}p{0.26\textwidth}@{}}
\toprule\noalign{}
失败场景 & 检测方式 & 恢复策略 & 用户影响 \\
\midrule\noalign{}
\endhead
\bottomrule\noalign{}
\endlastfoot
客户机进程崩溃 & 中心机轮询超时 & 客户机自动重启(Docker restart policy) & 数据缺失,可从历史日志恢复 \\
RabbitMQ 消息队列满 & 队列写入失败 & 告警 + 手动扩容 & 实时性下降,需人工介入 \\
OpenSearch 写入失败 & 写入错误响应 & 重试 3 次 → 失败则丢弃 + 告警 & 事件丢失 \\
Neo4j 连接断开 & Cypher 查询异常 & 自动重连 + 重试 & 图谱更新延迟 \\
前端 API 调用超时 & HTTP timeout & 前端重试 + 错误提示 & 页面功能暂时不可用 \\
LLM 服务不可用 & API 调用失败 & 走固定回退路径(基于规则的解释生成) & 解释文本质量下降,功能仍可用 \\
\end{longtable}
}

%
}{%
  \begin{SideBar}
    \textbf{尚未生成正文内容。}
  \end{SideBar}
}%

\end{document}
