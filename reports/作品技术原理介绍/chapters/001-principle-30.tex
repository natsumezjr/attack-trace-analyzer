\section{作品技术原理介绍}\label{technical-principle}

\subsection{文档目的}\label{document-purpose}

本文件用于说明ATA系统从``多源采集''到``检测融合''再到``图谱溯源''的端到端技术原理。

\subsection{读者对象}\label{target-audience}

\begin{itemize}
\tightlist
\item
  评审人员
\item
  技术方案审核人员
\end{itemize}

\subsection{引用关系}\label{references}

\begin{itemize}
\tightlist
\item
  字段规范与示例:\texttt{../80-规范/81-ECS字段规范.md}
\item
  图谱规范:\texttt{../80-规范/84-Neo4j实体图谱规范.md}
\item
  接口规范:\texttt{../80-规范/86-接口全局约定.md}、\texttt{../80-规范/88-前端与中心机接口.md}
\item
  概要设计:\texttt{../40-概要设计/40-概要设计报告.md}
\item
  详细设计:\texttt{../50-详细设计/}
\end{itemize}

\subsection{技术路线概览}\label{tech-overview}

ATA系统采用``多源遥测 -\textgreater{} 统一字段 -\textgreater{} Store-first 检测 -\textgreater{} 图谱化建模 -\textgreater{} 溯源算法 -\textgreater{} 可视化与报告''的技术路线,核心目标是让每条结论都能追溯到可验证的 \texttt{event.id} 证据链,同时支持主机行为、主机日志、网络流量三类输入的统一分析 {[}R1{]}{[}R3{]}{[}R4{]}{[}R5{]}{[}R6{]}。

图 0-1 展示系统端到端技术路线:

\begin{figure}[H]
\centering
\includegraphics[width=0.95\textwidth]{../figures/architecture_overview.pdf}
\caption{ATA End-to-End Architecture Overview}
\label{fig:architecture-overview}
\end{figure}

\subsection{多源采集与 ECS 归一化}\label{multi-source-ecs}

主机行为、主机日志、网络流量是常见的三类安全遥测数据,具有覆盖面互补、证据链可交叉验证的优势 {[}R4{]}{[}R5{]}{[}R6{]}{[}R7{]}。ATA系统在客户机侧分别接入 Falco、Filebeat(+Sigma)、Suricata,并统一转换为 ECS 子集字段(权威口径见 \texttt{../80-规范/81-ECS字段规范.md})。

表 1 给出三类数据源的对比:

{\def\LTcaptype{none} % do not increment counter
\begin{longtable}[]{@{}lllll@{}}
\toprule\noalign{}
数据源类别 & 典型数据 & 采集组件 & 关键字段(示例) & 主要价值 \\
\midrule\noalign{}
\endhead
\bottomrule\noalign{}
\endlastfoot
主机行为 & 进程创建、文件访问、网络 syscall & Falco & \texttt{process.*} \texttt{file.*} \texttt{network.*} & 细粒度行为链路、进程关系 \\
主机日志 & 认证日志、系统日志、审计日志 & Filebeat + Sigma & \texttt{event.*} \texttt{user.*} \texttt{log.*} & 账号与权限变更、持久化证据 \\
网络流量 & DNS/HTTP/Flow/IDS 告警 & Suricata & \texttt{source.*} \texttt{destination.*} \texttt{network.*} & 对外通信链路、C2 线索 \\
\end{longtable}
}

\subsubsection{Falco(主机行为)}\label{falco-host-behavior}

Falco 基于内核事件或 eBPF 采集 syscall,适合捕获进程创建、文件访问、网络连接等主机行为 {[}R4{]}。ATA系统在 \texttt{client/sensor/falco/ecs-converter/} 将 Falco JSONL 输出转换为 ECS 字段,并投递至 RabbitMQ 的 \texttt{data.falco} 队列。

关键点:

\begin{itemize}
\tightlist
\item
  以规则名和事件上下文生成 \texttt{event.id},保证幂等写入
\item
  保留 \texttt{process.parent.*}、\texttt{process.executable} 等字段,支持进程链分析
\end{itemize}

\subsubsection{Filebeat + Sigma(主机日志)}\label{filebeat-sigma-host-logs}

主机日志类遥测在业界常由 Wazuh/OSSEC 等 HIDS 处理 {[}R12{]},ATA系统实现上采用 Filebeat 收集日志并结合 Sigma 规则做检测,达成等价的``主机日志 + 规则检测''能力 {[}R6{]}{[}R7{]}。对应实现位于 \texttt{client/sensor/filebeat/}。

关键点:

\begin{itemize}
\tightlist
\item
  \texttt{event.dataset} 用于标识日志来源与规则集
\item
  Sigma 告警转为 ECS Finding,统一进入检测融合流程
\end{itemize}

\subsubsection{Suricata(网络流量)}\label{suricata-network-traffic}

Suricata 通过抓包与 IDS 规则生成 EVE JSON,覆盖 DNS/HTTP/Flow 等网络事件与告警 {[}R5{]}。ATA系统在 \texttt{client/sensor/suricata/exporter/} 中将 EVE JSON 转换为 ECS 并投递 \texttt{data.suricata} 队列。

关键点:

\begin{itemize}
\tightlist
\item
  统一对 \texttt{source.*} / \texttt{destination.*} / \texttt{network.*} 建模
\item
  重点保留域名、协议、端口与告警签名
\end{itemize}

\subsubsection{\texorpdfstring{1.4 ECS 归一与 \texttt{event.id}}{1.4 ECS 归一与 event.id}}\label{ecs-normalization-event-id}

为保证多源可融合、可检索、可溯源,ATA系统统一遵循 ECS 子集字段规范 {[}R3{]}。关键字段如下:

{\def\LTcaptype{none} % do not increment counter
\begin{longtable}[]{@{}lll@{}}
\toprule\noalign{}
字段 & 说明 & 用途 \\
\midrule\noalign{}
\endhead
\bottomrule\noalign{}
\endlastfoot
\texttt{event.id} & 全局去重标识 & 幂等写入与证据引用 \\
\texttt{event.kind} / \texttt{event.category} & 事件类型 & 路由与检测 \\
\texttt{event.dataset} & 数据源标识 & 区分 Falco/Filebeat/Suricata \\
\texttt{host.id} / \texttt{host.name} & 主机身份 & 资产与图谱归并 \\
\texttt{process.*} / \texttt{file.*} & 主机行为实体 & 进程树与文件链 \\
\texttt{source.*} / \texttt{destination.*} & 网络通信 & C2 与横向移动 \\
\end{longtable}
}

\subsection{传输缓冲与中心机轮询}\label{buffer-and-polling}

客户机采用 RabbitMQ 作为本地缓冲层,中心机以固定周期轮询拉取增量数据,避免中心机与客户机强耦合 {[}R10{]}。队列消息在 \texttt{basic.get\ +\ ack} 后即被消费,保证``增量语义''。

表 2 为队列与接口对照:

{\def\LTcaptype{none} % do not increment counter
\begin{longtable}[]{@{}lll@{}}
\toprule\noalign{}
队列 & 拉取接口 & 数据源 \\
\midrule\noalign{}
\endhead
\bottomrule\noalign{}
\endlastfoot
\texttt{data.falco} & \texttt{GET\ /falco} & Falco 主机行为 \\
\texttt{data.filebeat} & \texttt{GET\ /filebeat} & 主机日志 \\
\texttt{data.suricata} & \texttt{GET\ /suricata} & 网络流量 \\
\end{longtable}
}

\subsection{Store-first 检测与告警融合(OpenSearch)}\label{store-first-detection-opensearch}

中心机采用 OpenSearch 作为事件与告警的统一存储与检索引擎 {[}R8{]}。流程遵循 Store-first:先入库 Telemetry,再触发检测与融合生成 Findings,确保所有检测结论都有可回放的原始事件。

图 3-1 展示检测融合流程:

\begin{figure}[H]
\centering
\includegraphics[width=0.95\textwidth]{../figures/store_first_pipeline.pdf}
\caption{Store-first Detection \& Fusion Pipeline}
\label{fig:store-first-pipeline}
\end{figure}

表 3 为索引与数据对象对应关系:

{\def\LTcaptype{none} % do not increment counter
\begin{longtable}[]{@{}lll@{}}
\toprule\noalign{}
数据对象 & OpenSearch 索引(示例) & 说明 \\
\midrule\noalign{}
\endhead
\bottomrule\noalign{}
\endlastfoot
Telemetry & \texttt{ecs-events-*} & 统一 ECS 事件 \\
Raw Finding & \texttt{findings-raw-*} & 检测原始告警 \\
Canonical Finding & \texttt{findings-canonical-*} & 融合后告警 \\
Analysis Task & \texttt{analysis-tasks-*} & 溯源任务状态 \\
\end{longtable}
}

\subsection{图谱建模与查询(Neo4j)}\label{graph-modeling-query-neo4j}

OpenSearch 擅长检索与统计,但溯源分析需要更强的``关系遍历与路径解释''。因此ATA系统将 Canonical Findings 与必要的 Telemetry 建模为实体关系图写入 Neo4j {[}R9{]}。

表 4 为 ECS 字段到图谱实体/关系的映射:

{\def\LTcaptype{none} % do not increment counter
\begin{longtable}[]{@{}lll@{}}
\toprule\noalign{}
ECS 字段/对象 & 图谱实体 & 关系类型(示例) \\
\midrule\noalign{}
\endhead
\bottomrule\noalign{}
\endlastfoot
\texttt{host.*} & Host & Host -\textgreater{} Process(运行) \\
\texttt{process.*} & Process & Process -\textgreater{} File(访问) \\
\texttt{file.*} & File & Process -\textgreater{} File(创建/修改) \\
\texttt{source.*} / \texttt{destination.*} & IP/Port & Process -\textgreater{} IP(连接) \\
\texttt{dns.*} / \texttt{url.*} & Domain/URL & Process -\textgreater{} Domain(解析/访问) \\
\end{longtable}
}

关键原则:

\begin{itemize}
\tightlist
\item
  以 \texttt{event.id} 回写边属性,保证溯源证据可追溯
\item
  通过 \texttt{analysis.*} 字段写回溯源结果(高亮边、任务 ID)
\end{itemize}

\subsection{关联溯源算法(KillChain)}\label{correlation-trace-algorithm-killchain}

溯源任务由前端触发,中心机异步执行并写回结果。核心算法遵循``FSA 战术分段 -\textgreater{} 候选路径 -\textgreater{} LLM 选择 -\textgreater{} 写回解释''的流程 {[}R1{]}{[}R2{]}。

图 5-1 展示 KillChain 分析流水线:

\begin{figure}[H]
\centering
\includegraphics[width=0.85\textwidth]{../figures/killchain_pipeline.pdf}
\caption{KillChain Analysis Pipeline (Phases A--D)}
\label{fig:killchain-pipeline}
\end{figure}

表 5 为算法阶段与输出:

{\def\LTcaptype{none} % do not increment counter
\begin{longtable}[]{@{}llll@{}}
\toprule\noalign{}
阶段 & 输入 & 输出 & 目的 \\
\midrule\noalign{}
\endhead
\bottomrule\noalign{}
\endlastfoot
Phase A & 告警边集合 & 战术段(Segment) & 识别攻击阶段 \\
Phase B & 段锚点 & 候选路径集合 & 连接阶段路径 \\
Phase C & 候选路径 & KillChain 结果 & 选择最合理链路 \\
Phase D & KillChain & 图谱写回/任务文档 & 可视化与报告 \\
\end{longtable}
}

ATA系统以 MITRE ATT\&CK 的战术维度作为分段依据,并引入回退机制保障在无 LLM 或低置信场景下仍可产出稳定结果。

\subsection{TTP 相似度匹配(组织归因)}\label{ttp-similarity-attribution}

系统从 Canonical Findings 提取 \texttt{threat.tactic.id} 与 \texttt{threat.technique.id},与 MITRE ATT\&CK CTI 离线数据进行相似度匹配 {[}R1{]}{[}R2{]}。实现上采用 TF-IDF 向量化与余弦相似度:

\begin{equation}
\text{sim}(A, B) = \frac{A \cdot B}{|A| \times |B|}
\end{equation}

输出 Top-3 相似组织及关键 techniques,作为溯源解释的补充证据。

\subsection{前端可视化与报告导出}\label{frontend-visualization-report-export}

前端采用 Next.js App Router,实现图谱渲染、任务管理、进度轮询与报告导出(接口定义见 \texttt{../80-规范/88-前端与中心机接口.md})。

主要能力:

\begin{itemize}
\tightlist
\item
  图谱按时间窗拉取,突出告警边与溯源路径
\item
  任务面板展示任务状态与溯源结论
\item
  报告导出输出时间线摘要、关键证据与 TTP 归因
\end{itemize}

\subsection{质量与一致性保障}\label{quality-consistency}

表 6 总结关键工程策略:

{\def\LTcaptype{none} % do not increment counter
\begin{longtable}[]{@{}lll@{}}
\toprule\noalign{}
策略 & 说明 & 目的 \\
\midrule\noalign{}
\endhead
\bottomrule\noalign{}
\endlastfoot
幂等写入 & \texttt{event.id} 去重 & 防止重复入库 \\
单定时器轮询 & 严格顺序执行 & 避免数据漂移 \\
固定时间窗 & 统一查询边界 & 结果可复现 \\
统一字段规范 & ECS 子集 & 跨源融合一致性 \\
写回边属性 & \texttt{analysis.*} & 前后端一致展示 \\
\end{longtable}
}

\subsection{项目实现绑定点}\label{implementation-bindings}

\begin{itemize}
\tightlist
\item
  客户机采集:\texttt{client/sensor/falco/} \texttt{client/sensor/filebeat/} \texttt{client/sensor/suricata/}
\item
  队列与拉取 API:\texttt{client/backend/} \texttt{docs/50-详细设计/客户机/55-拉取接口.md}
\item
  中心机轮询与存储:\texttt{backend/app/services/client\_poller.py} \texttt{backend/app/services/opensearch/}
\item
  图谱建模与查询:\texttt{backend/app/services/neo4j/}
\item
  KillChain/TTP:\texttt{backend/app/services/analyze/}
\item
  前端展示与导出:\texttt{frontend/app/} \texttt{frontend/components/}
\end{itemize}

\subsection{参考资料与链接}\label{references-links}

\begin{itemize}
\tightlist
\item
  {[}R1{]} MITRE ATT\&CK: \url{https://attack.mitre.org/}
\item
  {[}R2{]} MITRE CTI (STIX Bundle): \url{https://github.com/mitre/cti}
\item
  {[}R3{]} Elastic Common Schema (ECS): \url{https://www.elastic.co/guide/en/ecs/current/index.html}
\item
  {[}R4{]} Falco Documentation: \url{https://falco.org/docs/}
\item
  {[}R5{]} Suricata Documentation: \url{https://docs.suricata.io/}
\item
  {[}R6{]} Filebeat Documentation: \url{https://www.elastic.co/guide/en/beats/filebeat/current/index.html}
\item
  {[}R7{]} Sigma Rules: \url{https://sigmahq.io/}
\item
  {[}R8{]} OpenSearch Security Analytics: \url{https://opensearch.org/docs/latest/security-analytics/}
\item
  {[}R9{]} Neo4j Documentation: \url{https://neo4j.com/docs/}
\item
  {[}R10{]} RabbitMQ Documentation: \url{https://www.rabbitmq.com/documentation.html}
\item
  {[}R11{]} OpenSearch Project: \url{https://opensearch.org/docs/latest/}
\item
  {[}R12{]} Wazuh Documentation: \url{https://documentation.wazuh.com/}
\end{itemize}
