\section{概要设计报告}\label{outline-design-report}

本文件定义系统的\textbf{总体架构与关键机制},重点包括:

\begin{itemize}
\tightlist
\item
  中心机单定时器顺序流水线架构
\item
  前端可视化查询链路
\item
  面向教师交互的异步溯源任务模型
\item
  三大模块(OpenSearch、Neo4j、Analysis)的边界与协作方式
\end{itemize}

本文件不重复以下文档的细节:

\begin{itemize}
\tightlist
\item
  字段口径:\texttt{../80-规范/81-ECS字段规范.md}
\item
  图谱口径:\texttt{../80-规范/84-Neo4j实体图谱规范.md}
\item
  环境变量:\texttt{../80-规范/89-环境变量与配置规范.md}
\item
  客户机与中心机接口:\texttt{../80-规范/87-客户机与中心机接口.md}
\item
  OpenSearch 存储:\texttt{../50-详细设计/中心机/}\\ \texttt{62-OpenSearch存储与索引治理.md}
\item
  Neo4j 入图与图查询:\texttt{../50-详细设计/中心机/}\\ \texttt{64-Neo4j入图与图查询.md}
\item
  Analysis 任务模型:\texttt{../50-详细设计/分析/}\\ \texttt{70-任务模型与状态机.md}
\end{itemize}

\subsection{总体架构}\label{overall-architecture}

\subsubsection{术语与缩略语}\label{terminology}

为保证文档表述的一致性,本节定义核心术语:

{\def\LTcaptype{none} % do not increment counter
\small
\begin{longtable}[]{@{}p{0.2\textwidth}p{0.25\textwidth}p{0.45\textwidth}@{}}
\toprule\noalign{}
术语 & 英文 & 定义 \\
\midrule\noalign{}
\endhead
\bottomrule\noalign{}
\endlastfoot
\textbf{客户机(Client)} & Client & 部署在被监控主机上的数据采集与转换节点,采集 Falco、Filebeat、Suricata 三类数据源,转换为 ECS 格式并缓冲在本地队列 \\
\textbf{中心机(Center)} & Center Server & 汇聚数据、执行检测与分析的核心服务节点,承担轮询拉取、入库、检测融合、入图、对外 API 与异步溯源任务 \\
\textbf{Telemetry} & Telemetry & 遥测数据,指未经检测规则触发的原始观测事件,\texttt{event.kind="event"} \\
\textbf{Raw Finding} & Raw Finding & 由单一检测规则(如 Sigma 规则、Security Analytics 检测器)产生的初步告警,\texttt{event.kind="alert"} 且 \texttt{dataset!="canonical"} \\
\textbf{Canonical Finding} & Canonical Finding & 经融合去重后的标准化告警,\texttt{event.kind="alert"} 且 \texttt{dataset="finding.canonical"},用于图谱入图与溯源分析主输入 \\
\textbf{溯源任务(Trace Task)} & Trace Task & 以目标节点为起点、时间窗为边界的溯源分析任务,异步执行并在完成后将结果写回 Neo4j 边属性 \\
\textbf{TTP} & Tactics, Techniques, and Procedures & 战术、技术与过程,指 MITRE ATT\&CK 框架中对攻击行为的分类与描述 \\
\textbf{ECS} & Elastic Common Schema & Elastic 公司定义的通用事件字段规范,用于统一不同数据源的字段命名 \\
\textbf{实体图谱} & Entity Graph & 以 Host、User、Process、File、IP、Domain 等实体为节点、以事件为边构成的属性图,存储于 Neo4j \\
\textbf{关键路径} & Critical Path & 溯源任务计算出的攻击传播路径,由边属性 \texttt{analysis.is\_path\_edge=true} 标记 \\
\textbf{顺序流水线} & Sequential Pipeline & 中心机使用单个定时器串行执行轮询、存储、检测、入图四个步骤的架构模式 \\
\end{longtable}
}

\subsubsection{系统组成}\label{system-components}

系统由客户机侧和中心机侧组成:

\begin{itemize}
\tightlist
\item
  客户机侧:采集、转换、本地缓冲、对外提供拉取接口
\item
  中心机侧:定时拉取、入库、检测、融合、入图、对外 API、异步溯源与报告
\end{itemize}

\subsubsection{系统架构图(逻辑视图)}\label{architecture-logic-view}

\begin{figure}[htbp]
\centering
\includegraphics[width=\textwidth,height=0.5\textheight,keepaspectratio]{figures/outline-40-01.pdf}
\caption{系统架构图}
\label{fig:outline-40-01}
\end{figure}

\subsection{核心数据分层(中心机视角)}\label{data-layers}

\subsubsection{数据分层架构}\label{data-layer-architecture}

系统采用以下数据层次(从底到顶):

\begin{figure}[htbp]
\centering
\includegraphics[width=\textwidth,height=0.5\textheight,keepaspectratio]{figures/outline-40-02.pdf}
\caption{数据分层架构}
\label{fig:outline-40-02}
\end{figure}

\subsubsection{数据层次说明}\label{data-layer-description}

{\def\LTcaptype{none} % do not increment counter
\small
\begin{longtable}[]{@{}p{0.12\textwidth}p{0.22\textwidth}p{0.25\textwidth}p{0.2\textwidth}p{0.15\textwidth}@{}}
\toprule\noalign{}
层次 & 名称 & 判定条件 & 存储位置 & 用途 \\
\midrule\noalign{}
\endhead
\bottomrule\noalign{}
\endlastfoot
\textbf{Layer 1} & Telemetry(事实层) & \texttt{event.kind="event"} & OpenSearch \texttt{ecs-events-*} & 原始事件检索、检测输入 \\
\textbf{Layer 2} & Raw Findings(原始告警层) & \texttt{event.kind="alert"} 且 \texttt{dataset!="canonical"} & OpenSearch \texttt{raw-findings-*} & 原始告警审计、融合输入 \\
\textbf{Layer 3} & Canonical Findings(规范告警层) & \texttt{event.kind="alert"} 且 \texttt{dataset="canonical"} & OpenSearch \texttt{canonical-findings-*} & 图谱与溯源主输入 \\
\textbf{Layer 4} & Entity Graph(图谱层) & Telemetry + Canonical 转换 & Neo4j 节点/边 & 图查询、路径分析 \\
\textbf{Layer 5} & Trace Task(溯源任务层) & 异步任务创建 & OpenSearch \texttt{analysis-tasks-*} + Neo4j 边属性 & 任务状态、溯源结果 \\
\end{longtable}
}

\begin{quote}
说明:攻击链作为展示结构,由图谱与任务结果共同承载。前端展示时以图谱边序列及边属性为主,不依赖单独的链路索引。
\end{quote}

\subsection{中心机定时流水线(单定时器顺序流水线)}\label{sequential-pipeline}

\subsubsection{调度原则}\label{scheduling-principles}

\begin{itemize}
\tightlist
\item
  中心机必须使用\textbf{单一}定时器触发流水线
\item
  定时器周期由环境变量 \texttt{CENTER\_POLL\_INTERVAL\_SECONDS} 控制,默认值为 \texttt{5} 秒
\item
  同一时间只允许存在一个流水线实例运行:当上一次流水线未结束时,新的定时触发必须被跳过,避免并发写入导致的数据漂移与重复工作
\end{itemize}

\subsubsection{设计决策:为什么采用单定时器顺序流水线架构}\label{sequential-pipeline-decision}

\textbf{决策内容}:采用单个定时器串行执行轮询、存储、检测、入图四个步骤,而非并发或事件驱动架构

\textbf{决策依据}:

\begin{enumerate}
\def\labelenumi{\arabic{enumi}.}
\tightlist
\item
  \textbf{数据一致性保证}:串行执行避免了并发写入导致的事件顺序错乱,确保 Telemetry → Raw → Canonical → Graph 的严格因果顺序
\item
  \textbf{资源可控性}:单线程模型使 CPU、内存、网络带宽消耗可预测,避免高并发场景下的资源竞争
\item
  \textbf{故障排查简化}:串行流水线的调用链清晰,便于通过日志追踪问题根因
\item
  \textbf{实现简洁性}:避免引入分布式锁、消息队列去重等复杂机制
\end{enumerate}

\textbf{代价与权衡}:

\begin{itemize}
\tightlist
\item
  吞吐量上限受限于单线程处理速度,预计可支撑约 10 个客户机规模
\item
  单步骤故障会导致后续步骤阻塞,需通过快速失败与错误隔离缓解
\end{itemize}

\subsubsection{流水线输入与输出}\label{pipeline-inputs-outputs}

\begin{itemize}
\tightlist
\item
  输入:所有已注册客户机的新数据(由轮询拉取接口返回,格式为 ECS 文档)
\item
  输出:

  \begin{itemize}
  \tightlist
  \item
    OpenSearch:Telemetry、Raw Findings、Canonical Findings、Tasks
  \item
    Neo4j:Entity Graph(含溯源任务写回的边属性)
  \end{itemize}
\end{itemize}

\subsubsection{四步顺序与职责(每次 tick 内严格顺序执行)}\label{four-steps-sequential}

\paragraph{Step 1:从客户机拉取数据}\label{step-1-poll-data}

对每个已注册客户机:

\begin{enumerate}
\def\labelenumi{\arabic{enumi})}
\tightlist
\item
  中心机读取注册表条目(含 \texttt{listen\_url}、\texttt{client\_token\_hash} 等元数据)
\item
  调用客户机拉取接口(见 \texttt{../80-规范/87-客户机与中心机接口.md})
\item
  客户机从 RabbitMQ 队列中取出消息并返回 ECS 文档列表:

  \begin{itemize}
  \tightlist
  \item
    队列为空则返回空数组
  \item
    拉取行为会消费队列中的消息(队列语义保证增量)
  \end{itemize}
\end{enumerate}

\paragraph{Step 2:写入 OpenSearch(字段处理)}\label{step-2-write-opensearch}

中心机必须对每条 ECS 文档执行字段处理(细节见 \texttt{../50-详细设计/中心机/62-OpenSearch存储与索引治理.md} 与 \texttt{../80-规范/81-ECS字段规范.md}):

\begin{itemize}
\tightlist
\item
  三时间字段补齐与覆盖规则
\item
  \texttt{event.kind} 与 \texttt{event.dataset} 校验
\item
  \texttt{event.id} 补齐与幂等去重
\item
  按 \texttt{event.kind/event.dataset} 路由写入对应索引
\end{itemize}

\paragraph{Step 3:Store-first 检测 + Raw→Canonical 融合}\label{step-3-detection-fusion}

中心机必须在每个 tick 内执行:

\begin{enumerate}
\def\labelenumi{\arabic{enumi})}
\tightlist
\item
  触发 Store-first 检测:由 OpenSearch Security Analytics 对 Telemetry 索引执行扫描
\item
  读取检测产生的 Findings,并转换为 ECS Finding 文档,写入 \texttt{raw-findings-*}
\item
  对指定时间窗内的 Raw Findings 执行融合去重,生成 Canonical Findings,写入 \texttt{canonical-findings-*}
\end{enumerate}

融合去重的指纹规则、provider 合并规则与幂等规则由 \texttt{../50-详细设计/中心机/62-OpenSearch存储与索引治理.md} 定义。

\begin{quote}
实现口径(与当前代码对齐):Step 3 在每个 tick 内固定执行(不提供开关),用于持续产出 Raw→Canonical 的规范告警层
\end{quote}

\paragraph{Step 4:触发 ECS→Graph 写入 Neo4j}\label{step-4-write-neo4j}

中心机必须在每个 tick 内执行入图:

\begin{enumerate}
\def\labelenumi{\arabic{enumi})}
\tightlist
\item
  以 Canonical Findings 为主,补充必要的 Telemetry(用于补边与证据回溯)
\item
  对每条输入文档执行 ECS→Graph 转换
\item
  将节点/边写入 Neo4j,并满足 \texttt{../80-规范/84-Neo4j实体图谱规范.md} 的唯一键与边属性规范
\item
  边必须写入 \texttt{ts\_float}(数值时间戳),支撑时间窗查询与图算法投影
\end{enumerate}

\begin{quote}
实现口径(与当前代码对齐):Step 4 在每个 tick 内固定执行;入图以 \texttt{event.ingested}(中心机入库时间)为窗口边界,确保本 tick 生成的数据不会遗漏
\end{quote}

\subsubsection{流水线时序图}\label{pipeline-timing-diagram}

\begin{figure}[htbp]
\centering
\includegraphics[width=\textwidth,height=0.5\textheight,keepaspectratio]{figures/outline-40-03.pdf}
\caption{流水线时序图}
\label{fig:outline-40-03}
\end{figure}

流水线按照以下严格顺序执行(每个 tick 内):

\begin{enumerate}
\def\labelenumi{\arabic{enumi})}
\item \textbf{Step 1: 拉取数据} - 轮询服务从客户机集群拉取 ECS 事件
\item \textbf{Step 2: 写入 OpenSearch} - 批量存储事件,字段处理与去重
\item \textbf{Step 3: 检测与融合} - Store-first 检测生成 Raw Findings,融合去重生成 Canonical Findings
\item \textbf{Step 4: 入图 Neo4j} - 将 Canonical Findings 和 Telemetry 批量写入图谱
\end{enumerate}

整个流程由定时器(5s 周期)触发,单线程串行执行,确保数据一致性。

\subsection{前端可视化查询链路}\label{frontend-visualization}

\subsubsection{查询边界}\label{query-boundaries}

\begin{itemize}
\tightlist
\item
  Telemetry 与 Findings 的检索由 OpenSearch 提供
\item
  图谱可视化(节点/边)与路径查询由 Neo4j 提供
\item
  前端不得直连数据库,只通过后端 API 访问
\end{itemize}

\subsubsection{图可视化最小闭环}\label{graph-visualization-loop}

\begin{enumerate}
\def\labelenumi{\arabic{enumi})}
\tightlist
\item
  前端请求后端图查询接口(动作包含:时间窗边查询、告警边查询、时间窗最短路)
\item
  后端调用 Neo4j 模块,返回图数据(nodes/edges)
\item
  前端渲染图,允许教师点击节点查看其属性与相关边的证据引用
\end{enumerate}

对应 API 的对外形状由后端路由定义(实现已存在:\texttt{/api/v1/graph/query}),模块内部规则见 \texttt{../50-详细设计/中心机/64-Neo4j入图与图查询.md}。

\subsection{异步溯源任务(create\_task)}\label{async-trace-tasks}

\subsubsection{为什么必须异步}\label{why-async}

节点溯源可能涉及:

\begin{itemize}
\tightlist
\item
  子图扩展与多跳查询
\item
  图算法(最短路、风险权重路径等)
\item
  解释性生成(TTP 解释与文本说明)
\end{itemize}

这些步骤耗时不稳定,必须以异步任务执行,并对前端提供可轮询的进度状态。

\subsubsection{任务模型(状态机)}\label{task-state-machine}

每个溯源任务在 OpenSearch \texttt{analysis-tasks-*} 里保存一条任务文档,任务状态只能取以下值之一:

\begin{figure}[htbp]
\centering
\includegraphics[width=\textwidth,height=0.5\textheight,keepaspectratio]{figures/outline-40-04.pdf}
\caption{任务状态机}
\label{fig:outline-40-04}
\end{figure}

\textbf{状态说明}:

\begin{itemize}
\tightlist
\item
  \texttt{queued}:已创建,等待执行
\item
  \texttt{running}:执行中
\item
  \texttt{succeeded}:已完成
\item
  \texttt{failed}:失败(含错误信息)
\end{itemize}

\textbf{状态转移约束}:

\begin{itemize}
\tightlist
\item
  只允许 \texttt{queued\ →\ running\ →\ succeeded/failed} 单向转移
\item
  禁止回退与跳转(如 \texttt{running\ →\ queued}、\texttt{queued\ →\ succeeded})
\end{itemize}

\subsubsection{创建、执行与结果写回}\label{task-execution-writeback}

\begin{enumerate}
\def\labelenumi{\arabic{enumi})}
\tightlist
\item
  前端在图上选定目标节点后,请求后端创建任务
\item
  后端立即返回 \texttt{task\_id}
\item
  Analysis 模块异步执行任务:从 Neo4j 读取数据、执行算法、得到关键边集合、风险评分、Technique/Tactic 摘要及解释文本
\item
  Analysis 模块把结果写回 Neo4j 边属性,形成被标注的图
\item
  前端轮询任务状态,任务完成后再次请求图查询接口,读取边属性并展示溯源结果
\end{enumerate}

对应后端 API(固定):

\begin{itemize}
\tightlist
\item
  创建任务:\texttt{POST\ /api/v1/analysis/tasks}
\item
  查询任务状态:\texttt{GET\ /api/v1/analysis/tasks/\{task\_id\}}
\item
  拉取写回边:\texttt{POST /api/v1/graph/query}\\ \hspace*{1.5em}(\texttt{analysis\_edges\_by\_task} 或 \texttt{edges\_in\_window} + 前端过滤)
\end{itemize}

写回字段命名、覆盖规则与数据类型由 \texttt{../50-详细设计/分析/70-任务模型与状态机.md} 与 \texttt{../50-详细设计/中心机/64-Neo4j入图与图查询.md} 定义。

\subsubsection{KillChain 分析(内部实现)}\label{killchain-analysis}

Analysis 模块的溯源任务内部使用 KillChain 算法重建攻击路径。

KillChain 算法基于 MITRE ATT\&CK 战术分段,通过有限状态自动机 (FSA) 识别攻击阶段,并使用大语言模型 (LLM) 选择最合理的段间连接路径,生成完整攻击链解释。

\textbf{核心能力}:

\begin{enumerate}
\def\labelenumi{\arabic{enumi}.}
\tightlist
\item
  \textbf{战术分段}:基于 MITRE ATT\&CK 战术将攻击事件自动分段(Initial Access → Execution → Privilege Escalation → Lateral Movement → C2 → Impact)
\item
  \textbf{路径重建}:识别各战术阶段之间的连接路径
\item
  \textbf{智能选择}:使用 LLM 从候选路径中选择最合理的攻击链
\item
  \textbf{可解释性}:生成 10-20 句中文的全链路解释,包含主谓宾结构
\item
  \textbf{置信度评估}:输出可信度评分(0.0-1.0)
\end{enumerate}

\textbf{输出结果}:

\begin{itemize}
\tightlist
\item
  \texttt{segments{[}{]}}:战术分段列表,每个段包含状态、时间范围、锚点、异常边摘要
\item
  \texttt{selected\_paths{[}{]}}:段间连接路径列表
\item
  \texttt{explanation}:LLM 生成的完整攻击链解释
\item
  \texttt{confidence}:可信度评分
\end{itemize}

\textbf{持久化方式}:

\begin{itemize}
\tightlist
\item
  Neo4j 边属性:\texttt{custom.killchain.uuid}(ECS 合规字段)
\item
  任务文档:\texttt{task.result.killchain\_uuid} 和 \texttt{task.result.killchain}
\end{itemize}

详细设计见:\texttt{50-详细设计/分析/69-KillChain概览设计.md} 与 \texttt{74-KillChain结果展示规范.md}

\subsection{安全与审计边界(演示)}\label{security-audit}

\begin{itemize}
\tightlist
\item
  运行环境:靶场内网。中心机所有服务与数据库端口只对靶场内可达网络开放
\item
  CTI 数据:使用离线 ATT\&CK Enterprise CTI 数据包;运行时不依赖外网
\item
  数据留痕:所有结论必须能回溯到 \texttt{event.id},并能定位到数据源类型与时间窗范围(证据链)
\end{itemize}
