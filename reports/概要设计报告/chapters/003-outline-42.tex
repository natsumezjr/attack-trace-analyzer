\section{部署拓扑与网络规划}\label{deployment-network-planning}

本文档定义项目靶场的节点拓扑、网络规划、端口分配与访问边界,确保现场演示的部署一致性与可复现性。

\subsection{读者对象}\label{outline-42-audience}

\begin{itemize}
\tightlist
\item
  靶场负责人
\item
  部署与编排人员
\item
  演示与验收人员
\end{itemize}

\subsection{引用关系}\label{outline-42-references}

\begin{itemize}
\tightlist
\item
  详细部署步骤:\texttt{../90-运维与靶场/91-靶场部署.md}
\item
  一键编排说明:\texttt{../90-运维与靶场/92-一键编排.md}
\item
  接口与端口规范:\texttt{../80-规范/87-客户机与中心机接口.md}
\end{itemize}

\subsection{节点与角色}\label{nodes-roles}

靶场节点角色划分如下:

\begin{itemize}
\tightlist
\item
  \texttt{center}:中心机(部署 OpenSearch、Neo4j、后端 API 与前端服务)
\item
  \texttt{client-01} 至 \texttt{client-04}:客户机节点(负责数据采集与缓冲)
\item
  \texttt{c2}:命令与控制服务器(用于生成可观测通信证据)
\end{itemize}

\subsection{网络规划}\label{network-planning}

网络规划遵循以下原则:

\begin{enumerate}
\def\labelenumi{\arabic{enumi}.}
\tightlist
\item
  网络隔离:靶场网络与宿主机外网隔离,所有演示数据在靶场内闭环流转。
\item
  端口暴露:中心机对外端口仅面向靶场网络可达。
\item
  服务隔离:客户机仅向中心机开放数据拉取接口,不对外提供其他服务端口。
\end{enumerate}

\subsection{端口规划}\label{port-planning}

端口分配与用途如下:

{\def\LTcaptype{none} % do not increment counter
\small
\begin{longtable}[]{@{}lrll@{}}
\toprule\noalign{}
组件 & 端口 & 协议 & 访问控制 \\
\midrule\noalign{}
\endhead
\bottomrule\noalign{}
\endlastfoot
中心机后端 & 8001 & HTTP & 仅内网 \\
中心机前端 & 3000 & HTTP & 公网(可选) \\
OpenSearch & 9200 & HTTP & 仅本地 \\
Neo4j Bolt & 7687 & Bolt & 仅本地 \\
Neo4j Browser & 7474 & HTTP & 仅本地 \\
客户机拉取接口 & 8888 & HTTP & 仅内网 \\
\end{longtable}
}

\subsection{部署拓扑}\label{deployment-topology}

部署拓扑遵循"中心机集中、客户机分布"的架构:

\begin{enumerate}
\def\labelenumi{\arabic{enumi}.}
\tightlist
\item
  客户机采集数据并缓冲至本地队列
\item
  中心机周期性轮询拉取客户机数据
\item
  中心机将数据写入 OpenSearch,并根据需要触发检测与告警融合
\item
  中心机将数据入图至 Neo4j,前端通过中心机 API 查询与展示
\end{enumerate}

\subsection{安全边界与隔离}\label{security-boundaries-isolation}

\begin{enumerate}
\def\labelenumi{\arabic{enumi}.}
\tightlist
\item
  代码隔离:中心机不运行未知样本,不执行来自外部的不可信代码。
\item
  访问控制:靶场网络中所有组件的访问鉴权与令牌由接口规范统一定义。
\item
  运维保障:现场演示以稳定性为首要目标,任何需要人工介入的步骤必须写入运维文档。
\end{enumerate}
