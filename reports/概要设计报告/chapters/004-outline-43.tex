\section{非功能设计}\label{non-functional-design}

本文档将需求中的非功能要求转化为确定的工程设计约束,涵盖幂等性、一致性、可解释性、性能、稳定性与可复现性六个方面。

\subsection{读者对象}\label{outline-43-audience}

\begin{itemize}
\tightlist
\item
  系统架构与后端实现人员
\item
  测试与验收人员
\end{itemize}

\subsection{引用关系}\label{outline-43-references}

\begin{itemize}
\tightlist
\item
  需求分析:\texttt{../20-需求与验收/20-需求分析报告.md}
\item
  数据与规范:\texttt{../80-规范/}
\item
  测试报告:\texttt{../96-测试/96-测试分析报告.md}
\end{itemize}

\subsection{幂等与去重}\label{idempotency-deduplication}

\subsubsection{OpenSearch 幂等}\label{opensearch-idempotency}

系统以 \texttt{event.id} 作为全局幂等键:

\begin{itemize}
\tightlist
\item
  同一 \texttt{event.id} 重复写入 OpenSearch 不产生重复文档
\item
  \texttt{event.id} 的生成规则由 \texttt{../80-规范/81-ECS字段规范.md} 定义
\end{itemize}

\subsubsection{图谱幂等}\label{graph-idempotency}

Neo4j 节点写入通过唯一键约束保证幂等;边写入携带证据引用并提供可过滤的去重机制,具体规范见:

\begin{itemize}
\tightlist
\item
  \texttt{../80-规范/84-Neo4j实体图谱规范.md}
\item
  \texttt{../80-规范/85-溯源结果写回规范.md}
\end{itemize}

\subsection{一致性与可回放}\label{consistency-replayability}

\subsubsection{数据规模假设}\label{data-scale-assumptions}

系统设计与性能目标基于以下数据规模假设制定:

{\def\LTcaptype{none} % do not increment counter
\begin{longtable}[]{@{}lll@{}}
\toprule\noalign{}
指标 & 规模 & 说明 \\
\midrule\noalign{}
\endhead
\bottomrule\noalign{}
\endlastfoot
单客户机每分钟事件量 & \textasciitilde1000 条 & 三传感器(Falco/Filebeat/Suricata)合计 \\
单客户机每天事件量 & \textasciitilde144 万条 & 按 24 小时连续运行计算 \\
10 客户机集群每天事件量 & \textasciitilde1440 万条 & 典型部署规模 \\
图谱节点规模(30天窗口) & \textasciitilde10 万节点 & Host/User/Process/File/IP/Domain 等实体 \\
图谱边规模(30天窗口) & \textasciitilde50 万边 & 包含时间属性的关系边 \\
Raw Findings 每天生成量 & \textasciitilde1000 条 & 基于 Sigma 与 Security Analytics 检测 \\
Canonical Findings 每天生成量 & \textasciitilde100 条 & 经融合去重后的规范告警 \\
单次溯源任务查询时间窗 & 1-60 分钟 & 用户可选的时间窗范围 \\
\end{longtable}
}

\subsubsection{一致性目标}\label{consistency-objectives}

系统需满足以下一致性要求:

\begin{enumerate}
\def\labelenumi{\arabic{enumi}.}
\tightlist
\item
  同一批输入事件在重复拉取与重复执行下,OpenSearch 与 Neo4j 的关键输出保持一致。
\item
  任意告警与溯源结论均可回溯到 Telemetry 的 \texttt{event.id} 证据。
\end{enumerate}

\subsection{可解释与证据链}\label{explainability-evidence}

系统需满足以下可解释性要求:

\begin{itemize}
\tightlist
\item
  前端展示的告警、关系边与溯源结果必须包含证据引用
\item
  溯源写回字段使用统一前缀 \texttt{analysis.},\\
  字段集合与覆盖规则由 \texttt{../80-规范/85-溯源结果写回规范.md} 定义
\end{itemize}

\subsection{性能目标与容量边界}\label{performance-capacity}

系统需满足以下性能要求:

\begin{itemize}
\tightlist
\item
  图查询应在秒级内返回满足可视化展示所需的数据量
\item
  溯源任务允许长耗时,但必须提供可轮询的状态与进度信息
\end{itemize}

容量边界与数据保留策略详见:

\begin{itemize}
\tightlist
\item
  \texttt{../80-规范/80-数据对象与生命周期.md}
\end{itemize}

\subsubsection{前端查询性能}\label{frontend-query-performance}

{\def\LTcaptype{none} % do not increment counter
\begin{longtable}[]{@{}lll@{}}
\toprule\noalign{}
指标 & 目标值 & 测量方法 \\
\midrule\noalign{}
\endhead
\bottomrule\noalign{}
\endlastfoot
图查询响应时间 & \textless3 秒 & 实际负载测试 \\
事件检索响应时间 & \textless1 秒 & 实际负载测试 \\
页面首次渲染 & \textless2 秒 & 浏览器性能测试 \\
\end{longtable}
}

\subsubsection{中心机轮询性能}\label{center-polling-performance}

{\def\LTcaptype{none} % do not increment counter
\begin{longtable}[]{@{}lll@{}}
\toprule\noalign{}
指标 & 目标值 & 测量方法 \\
\midrule\noalign{}
\endhead
\bottomrule\noalign{}
\endlastfoot
单次轮询耗时 & \textless10 秒 & 轮询日志统计 \\
单轮拉取吞吐量 & \textgreater1000 evt/s & 轮询日志统计 \\
数据积压恢复时间 & \textless5 分钟 & 积压场景测试 \\
\end{longtable}
}

\subsubsection{Neo4j 入图性能}\label{neo4j-ingestion-performance}

{\def\LTcaptype{none} % do not increment counter
\begin{longtable}[]{@{}lll@{}}
\toprule\noalign{}
指标 & 目标值 & 测量方法 \\
\midrule\noalign{}
\endhead
\bottomrule\noalign{}
\endlastfoot
100 事件入图耗时 & \textless1 秒 & 性能测试 \texttt{test\_batch\_ingest\_performance} \\
1000 事件入图耗时 & \textless5 秒 & 性能测试 \texttt{test\_batch\_ingest\_performance} \\
单事件平均网络往返 & \textless2 次 & 通过批量 UNWIND MERGE 实现 \\
批量写入吞吐量 & \textgreater100 evt/s & 实际负载测试 \\
\end{longtable}
}

\subsection{稳定性与故障处理}\label{stability-fault-handling}

\subsubsection{故障处理原则}\label{fault-handling-principles}

系统故障处理遵循以下原则:

\begin{enumerate}
\def\labelenumi{\arabic{enumi}.}
\tightlist
\item
  单次轮询失败不影响后续轮询继续执行
\item
  LLM 调用失败必须走固定回退路径,确保演示不因外部服务中断而失败
\item
  数据重置与复现流程固定,详见 \texttt{../90-运维与靶场/95-重置复现与排障.md}
\end{enumerate}

\subsubsection{失败场景与处理策略}\label{failure-scenarios-strategies}

{\def\LTcaptype{none} % do not increment counter
\small
\begin{longtable}[]{@{}p{0.18\textwidth}p{0.18\textwidth}p{0.28\textwidth}p{0.26\textwidth}@{}}
\toprule\noalign{}
失败场景 & 检测方式 & 恢复策略 & 用户影响 \\
\midrule\noalign{}
\endhead
\bottomrule\noalign{}
\endlastfoot
客户机进程崩溃 & 中心机轮询超时 & 客户机自动重启(Docker restart policy) & 数据缺失,可从历史日志恢复 \\
RabbitMQ 消息队列满 & 队列写入失败 & 告警 + 手动扩容 & 实时性下降,需人工介入 \\
OpenSearch 写入失败 & 写入错误响应 & 重试 3 次 → 失败则丢弃 + 告警 & 事件丢失 \\
Neo4j 连接断开 & Cypher 查询异常 & 自动重连 + 重试 & 图谱更新延迟 \\
前端 API 调用超时 & HTTP timeout & 前端重试 + 错误提示 & 页面功能暂时不可用 \\
LLM 服务不可用 & API 调用失败 & 走固定回退路径(基于规则的解释生成) & 解释文本质量下降,功能仍可用 \\
\end{longtable}
}
