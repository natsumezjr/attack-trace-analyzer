\section*{概要设计报告}
\addcontentsline{toc}{section}{概要设计报告}

\begin{SideBar}
\textbf{文档定位:}本报告用于描述 Attack Trace Analyzer(ATA)的总体架构与关键机制,回答"系统由哪些模块构成、模块如何协作、数据如何流转、关键设计为何这样取舍"。\\
\textbf{与《详细设计报告》的关系:}本报告强调"模块边界 + 数据/控制流";细节(字段、索引治理、图谱 schema、任务状态机实现)请参阅《详细设计报告》与对应代码目录。
\end{SideBar}

\begin{KeyBox}
\textbf{一页总结(系统核心要点):}
\begin{itemize}
  \item \textbf{多源采集(客户端):}Falco(主机行为)、Filebeat(+Sigma)(主机日志)、Suricata(网络流量)统一转换为 ECS 子集字段。
  \item \textbf{中心机流水线:}采用"单定时器顺序流水线"模型:轮询拉取增量数据 $\rightarrow$ Store-first 入 OpenSearch $\rightarrow$ 检测融合(Raw $\rightarrow$ Canonical)$\rightarrow$ ECS 入图写入 Neo4j。
  \item \textbf{双存储分工:}OpenSearch 负责检索/检测/去重/索引治理;Neo4j 负责实体关系图、时间窗内关系遍历与溯源结果写回(边属性 \texttt{analysis.*})。
  \item \textbf{溯源任务模型:}前端/接口触发异步 Trace Task(目标节点 + 时间窗),后台计算后写回图谱,并提供任务状态与结果查询。
  \item \textbf{可解释性原则:}每条溯源结论必须可回链到证据事件(如 \texttt{event.id}、\texttt{custom.evidence.event\_ids}),保证可复核。
\end{itemize}
\end{KeyBox}

\clearpage
