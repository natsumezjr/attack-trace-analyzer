\section{数据流与时序}\label{data-flow-timing}

系统涉及五类核心数据对象:

\begin{enumerate}
\def\labelenumi{\arabic{enumi}.}
\tightlist
\item
  \textbf{Telemetry(遥测数据)}:事实事件(ECS 格式,\texttt{event.kind="event"}),\\
  \hspace*{1.5em}存储于 OpenSearch \texttt{ecs-events-*} 索引。
\item
  \textbf{Raw Finding(原始告警)}:原始告警(ECS 格式,\texttt{event.kind="alert"} 且非 canonical),\\
  \hspace*{1.5em}存储于 OpenSearch \texttt{raw-findings-*} 索引。
\item
  \textbf{Canonical Finding(规范告警)}:规范告警(ECS 格式,\texttt{event.kind="alert"} 且\\
  \hspace*{1.5em}\texttt{event.dataset="finding.canonical"}),存储于 OpenSearch \texttt{canonical-findings-*} 索引。
\item
  \textbf{Entity Graph(实体关系图)}:实体关系图(Neo4j),输入数据为 Telemetry 与 Canonical Finding。
\item
  \textbf{Trace Task(溯源任务)}:溯源任务(OpenSearch 任务索引 + Neo4j 边属性写回)。
\end{enumerate}

\subsection{端到端数据流(中心机单定时器流水线)}\label{end-to-end-data-flow}

\subsubsection{客户机侧(每台主机)}\label{client-side-flow}

\begin{enumerate}
\def\labelenumi{\arabic{enumi})}
\tightlist
\item
  \textbf{数据采集}:传感器(Filebeat / Falco / Suricata)采集并输出原始数据。
\item
  \textbf{格式转换}:将采集结果转换为 ECS 标准文档(字段规范见 \texttt{../80-规范/81-ECS字段规范.md})。
\item
  \textbf{队列缓冲}:将 ECS 文档写入本地 RabbitMQ 队列进行缓冲。
\item
  \textbf{接口暴露}:对外提供数据拉取接口,从队列中取出数据并返回,供中心机轮询调用(接口规范见 \texttt{../80-规范/87-客户机与中心机接口.md})。
\end{enumerate}

\subsubsection{中心机侧(每次 tick,严格顺序)}\label{center-side-flow}

中心机定时器每次触发时,均按以下顺序执行:

\begin{enumerate}
\def\labelenumi{\arabic{enumi})}
\tightlist
\item
  \textbf{数据拉取}:从所有已注册客户机拉取新增数据。
\item
  \textbf{数据入库}:将数据写入 OpenSearch(包括 Telemetry/Raw Findings 路由、三时间字段处理及幂等去重)。
\item
  \textbf{检测与融合}:通过 Store-first 检测机制产出 Raw Findings,并融合生成 Canonical Findings 后写回 OpenSearch。
\item
  \textbf{图谱构建}:以 Canonical Findings 为主体,辅以必要的 Telemetry 数据,转换为图结构并写入 Neo4j。
\end{enumerate}

上述四个步骤由同一定时器驱动,形成"顺序流水线"架构,详细系统规格见 \texttt{40-概要设计报告.md}。

\begin{quote}
\textbf{实现细节}(与当前代码对齐):

\begin{itemize}
\tightlist
\item
  后端在每个 tick 固定执行 Step 1/2/3/4,构成"单定时器顺序流水线"架构;
\item
  Step 4 入图时以本 tick 的 \texttt{event.ingested} 时间窗为边界,确保"本 tick 生成的 Canonical Finding"不会因 \texttt{@timestamp} 较早而遗漏。
\end{itemize}
\end{quote}

\subsection{前端可视化数据流}\label{frontend-visualization-flow}

\begin{enumerate}
\def\labelenumi{\arabic{enumi})}
\tightlist
\item
  \textbf{页面访问}:用户打开前端页面。
\item
  \textbf{数据请求}:前端向后端发起请求:

  \begin{itemize}
  \tightlist
  \item
    查询事件/告警:后端查询 OpenSearch;
  \item
    查询图谱:后端查询 Neo4j。
  \end{itemize}
\item
  \textbf{页面渲染}:前端进行可视化渲染:

  \begin{itemize}
  \tightlist
  \item
    事件时间线(数据来源:OpenSearch);
  \item
    实体关系图谱(数据来源:Neo4j 返回的 nodes/edges)。
  \end{itemize}
\end{enumerate}

\subsection{溯源任务数据流}\label{trace-task-flow}

\begin{enumerate}
\def\labelenumi{\arabic{enumi})}
\tightlist
\item
  \textbf{节点选择}:用户在图谱上选定一个节点(node uid)。
\item
  \textbf{任务创建}:前端请求后端创建溯源任务,后端立即返回 \texttt{task\_id}。
\item
  \textbf{异步分析}:Analysis 模块异步执行:读取 Neo4j 子图 → 执行算法分析 → 生成关键路径与解释。
\item
  \textbf{结果写回}:Analysis 模块将分析结果写回 Neo4j 边属性(字段规范见 \texttt{32/33})。
\item
  \textbf{结果展示}:前端轮询任务状态,任务完成后再次请求图查询接口,读取边属性并展示溯源结果。
\item
  \textbf{报告导出}:前端导出报告,包含告警信息、图谱与溯源结果。
\end{enumerate}
