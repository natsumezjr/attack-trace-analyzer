\section{任务分工说明}\label{assignment-description}

\subsection{文档目的}\label{document-purpose}

本文件用于记录团队成员信息、任务分解、责任人、工作量占比与产出物映射。

\subsection{读者对象}\label{target-audience}

\begin{itemize}
\tightlist
\item
  老师与助教(用于评估分工合理性与工作量)
\item
  团队内部成员(用于对齐责任边界)
\end{itemize}

\subsection{引用关系}\label{references}

\begin{itemize}
\tightlist
\item
  交付物打包结构:\texttt{11-交付物清单与打包结构.md}
\end{itemize}

\subsection{团队与角色}\label{team-and-roles}

{\footnotesize
\def\LTcaptype{none} % do not increment counter
\begin{longtable}[]{@{}p{0.15\textwidth}p{0.25\textwidth}p{0.60\textwidth}@{}}
\toprule\noalign{}
姓名 & 角色 & 负责模块 \\
\midrule\noalign{}
\endhead
\bottomrule\noalign{}
\endlastfoot
董容嘉 & 数据获取(主机行为) & \texttt{client/sensor/falco/}(Falco 采集与 ECS 转换) \\
马民泽 & 数据获取(网络流量) & \texttt{client/sensor/suricata/}(Suricata 采集与 ECS 转换) \\
姜乐 & 数据获取(主机日志) & \texttt{client/sensor/filebeat/}(Filebeat + Sigma 规则) \\
田炎烁 & 数据湖构建 & \texttt{backend/app/services/opensearch/}(Security Analytics) \\
吕梓杰 & 图谱构建 & \texttt{backend/app/services/neo4j/}(OpenSearch → Neo4j 入图) \\
伍万圣 & 客户机/中心机全栈 & \texttt{client/} \texttt{backend/app/api/} \texttt{frontend/}(接口联调与前端展示) \\
范玉彬 & 关联溯源算法 & \texttt{backend/app/services/analyze/}(KillChain/FSA/候选路径) \\
曾靖然 & 关联溯源算法 & \texttt{backend/app/services/analyze/}(LLM 选择器/TTP 相似度) \\
王一安 & 靶场环境 & \texttt{docs/90-运维与靶场/} \texttt{scripts/靶场编排/} \\
张天华 & 项目架构师 & 系统架构设计、后端核心模块开发与技术难点攻关 \\
\end{longtable}
}

\begin{quote}
说明:主机日志采集在实现中采用 \texttt{Filebeat\ +\ Sigma}(见 \texttt{client/sensor/filebeat/}),与``Wazuh 能力''对应;如需对外说明为 Wazuh,请在报告中说明替代关系。
\end{quote}

\subsection{任务分解与责任人}\label{task-breakdown}

{\tiny
\def\LTcaptype{none} % do not increment counter
\begin{longtable}[]{@{}p{0.08\textwidth}p{0.18\textwidth}p{0.22\textwidth}p{0.10\textwidth}p{0.32\textwidth}@{}}
\toprule\noalign{}
任务编号 & 任务名称 & 任务说明 & 责任人 & 产出物 \\
\midrule\noalign{}
\endhead
\bottomrule\noalign{}
\endlastfoot
T-001 & Falco 采集与 ECS 转换 & 主机行为采集、ECS 字段归一与队列投递 & 董容嘉 & \texttt{client/sensor/falco/} \texttt{docs/50-详细设计/客户机/51-Falco采集与ECS转换.md} \\
T-002 & Suricata 采集与 ECS 转换 & 网络流量/IDS 输出、ECS 归一与队列投递 & 马民泽 & \texttt{client/sensor/suricata/} \texttt{docs/50-详细设计/客户机/52-Suricata采集与ECS转换.md} \\
T-003 & 主机日志采集与检测 & Filebeat 采集 + Sigma 检测、ECS 归一与队列投递 & 姜乐 & \texttt{client/sensor/filebeat/} \texttt{docs/50-详细设计/客户机/53-Filebeat采集与ECS转换.md} \\
T-004 & OpenSearch 数据湖与检测融合 & 索引治理、事件入库、Security Analytics 检测/融合 & 田炎烁 & \texttt{backend/app/services/opensearch/} \texttt{docs/50-详细设计/中心机/62-OpenSearch存储与索引治理.md} \texttt{docs/50-详细设计/中心机/63-检测与告警融合.md} \\
T-005 & OpenSearch → Neo4j 入图 & ECS 事件入图、实体/边建模、图查询与回标 & 吕梓杰 & \texttt{backend/app/services/neo4j/} \texttt{docs/50-详细设计/中心机/64-Neo4j入图与图查询.md} \texttt{docs/50-详细设计/中心机/65-图谱回标与边属性.md} \\
T-006 & 全栈集成与前端展示 & 客户机拉取接口联调、中心机 API、前端可视化 & 伍万圣 & \texttt{client/backend/} \texttt{backend/app/api/} \texttt{frontend/} \texttt{docs/50-详细设计/前端/} \\
T-007 & KillChain/FSA 关联溯源 & 状态机分段、候选路径构造与评分 & 范玉彬 & \texttt{backend/app/services/analyze/killchain.py} \texttt{backend/app/services/analyze/attack\_fsa.py} \texttt{docs/50-详细设计/分析/69-KillChain概览设计.md} \texttt{docs/50-详细设计/分析/70-任务模型与状态机.md} \texttt{docs/50-详细设计/分析/71-候选路径构造与评分.md} \\
T-008 & LLM 选择器与 TTP 相似度 & LLM 选择器与回退、TTP 相似度匹配与结果写回 & 曾靖然 & \texttt{backend/app/services/analyze/killchain\_llm.py} \texttt{backend/app/services/analyze/ttp\_similarity/} \texttt{docs/50-详细设计/分析/72-LLM选择器与回退机制.md} \texttt{docs/50-详细设计/分析/73-TTP相似度匹配.md} \\
T-009 & 靶场部署与复现 & 靶场编排、C2/攻击脚本、验证与排障 & 王一安 & \texttt{docs/90-运维与靶场/} \texttt{scripts/靶场编排/} \\
T-010 & 系统架构与核心模块 & ATA系统总体架构设计、后端异步任务模型、客户机轮询机制、技术难点攻关与集成排障 & 张天华 & \texttt{backend/app/services/client\_poller.py} \texttt{backend/app/models/} \texttt{docs/40-概要设计/} \texttt{docs/50-详细设计/} \\
\end{longtable}
}

\subsection{工作量占比与依据}\label{workload-distribution}

{\small
\def\LTcaptype{none} % do not increment counter
\begin{longtable}[]{@{}p{0.12\textwidth}p{0.15\textwidth}p{0.63\textwidth}@{}}
\toprule\noalign{}
姓名 & 工作量占比 & 依据说明 \\
\midrule\noalign{}
\endhead
\bottomrule\noalign{}
\endlastfoot
董容嘉 & 10\% & Falco 采集链路开发与联调 \\
马民泽 & 10\% & Suricata 采集链路开发与联调 \\
姜乐 & 10\% & Filebeat + Sigma 规则检测与联调 \\
田炎烁 & 10\% & OpenSearch 索引治理与检测融合 \\
吕梓杰 & 10\% & OpenSearch → Neo4j 入图与图查询 \\
伍万圣 & 10\% & 客户机/中心机联调与前端展示 \\
范玉彬 & 10\% & KillChain/FSA 关联溯源 \\
曾靖然 & 10\% & LLM 选择器与 TTP 相似度 \\
王一安 & 10\% & 靶场部署、复现与排障 \\
张天华 & 10\% & 系统架构设计、后端异步任务模型开发(client\_poller、Trace Task)、客户机轮询机制、集成排障与技术难点攻关 \\
\end{longtable}
}

\subsection{产出物映射}\label{deliverable-mapping}

ATA系统文档与代码的主要产出物路径如下:

\begin{itemize}
\tightlist
\item
  文档:\texttt{docs/}(详细设计、规范、运维、测试)
\item
  后端:\texttt{backend/}(中心机服务与分析)
\item
  客户机:\texttt{client/}(采集、转换、队列与拉取接口)
\item
  前端:\texttt{frontend/}(图谱可视化与任务管理)
\item
  部署与靶场:\texttt{scripts/靶场编排/}
\item
  测试与脚本:\texttt{backend/tests/} \texttt{tests/e2e/} \texttt{backend/scripts/}
\item
  交付材料:\texttt{report/}
\end{itemize}
