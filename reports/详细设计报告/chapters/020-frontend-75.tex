\subsection{图谱可视化与交互}\label{subsec:graph-visualization-interaction}

\subsection{图查询请求与时间窗}\label{subsec:graph-query-time-window}

图谱页的数据源为中心机图查询接口:\texttt{POST\ /api/v1/graph/query}。

\subsection{视图模式(固定配置)}\label{subsec:view-modes}

溯源分析页面提供三种视图模式:

\begin{enumerate}
\def\labelenumi{\arabic{enumi}.}
\tightlist
\item
  告警视图:\texttt{action="alarm\_edges"}
\item
  时间窗视图:\texttt{action="edges\_in\_window"}
\item
  任务视图:\texttt{action="analysis\_edges\_by\_task"}
\end{enumerate}

各视图功能说明:

\begin{itemize}
\tightlist
\item
  告警视图:展示 \texttt{is\_alarm=true} 的边集合
\item
  时间窗视图:展示用户选定时间窗内的边集合
\item
  任务视图:展示溯源任务写回的边集合
\end{itemize}

\subsection{时间窗输入(固定配置)}\label{subsec:time-window-input}

后端接口的时间窗字段定义如下(详见 \texttt{88-前端与中心机接口.md}):

\begin{itemize}
\tightlist
\item
  \texttt{start\_ts}:ISO 8601 格式(UTC 时区)
\item
  \texttt{end\_ts}:ISO 8601 格式(UTC 时区)
\end{itemize}

前端页面\textbf{不提供高级时间选择器},仅保留"最近 N 分钟"的快捷方式:

\begin{enumerate}
\def\labelenumi{\arabic{enumi}.}
\tightlist
\item
  用户选择 \texttt{N}(分钟);
\item
  在触发"刷新图谱"或"创建溯源任务"时,前端以当前时刻 \texttt{now} 计算:

  \begin{itemize}
  \tightlist
  \item
    \texttt{end\_ts\ =\ now}
  \item
    \texttt{start\_ts\ =\ now\ -\ N\ minutes}
  \end{itemize}
\item
  将计算得到的 \texttt{start\_ts/end\_ts} 同时用于:

  \begin{itemize}
  \tightlist
  \item
    时间窗视图(\texttt{action="edges\_in\_window"})
  \item
    溯源任务创建(\texttt{POST\ /api/v1/analysis/tasks})
  \end{itemize}
\end{enumerate}

该规则旨在限制图查询返回规模、确保任务输入与图展示口径一致,并保障报告导出的可复现性。

\textbf{时间窗计算示例}:

\begin{verbatim}
// 时间窗计算函数
function calculateTimeWindow(minutes) {
  const now = new Date();
  const endTs = now.toISOString(); // 当前时间 UTC

  const startTime = new Date(now.getTime() - minutes * 60 * 1000);
  const startTs = startTime.toISOString(); // N 分钟前 UTC

  return {
    start_ts: startTs,
    end_ts: endTs,
  };
}

// 使用示例
const window = calculateTimeWindow(15);
console.log(window);
// 输出示例:
// {
//   start_ts: "2025-01-16T10:30:00.000Z",
//   end_ts: "2025-01-16T10:45:00.000Z"
// }
\end{verbatim}

\textbf{时间窗快捷选项}:

{\def\LTcaptype{none} % do not increment counter
\footnotesize
\begin{longtable}[]{@{}p{0.25\textwidth}p{0.12\textwidth}p{0.53\textwidth}@{}}
\toprule\noalign{}
选项名称 & 时间范围 & 典型场景 \\
\midrule\noalign{}
\endhead
\bottomrule\noalign{}
\endlastfoot
最近 5 分钟 & N=5 & 实时监控 \\
最近 15 分钟 & N=15 & 即时分析 \\
最近 30 分钟 & N=30 & 短期溯源 \\
最近 1 小时 & N=60 & 常规分析 \\
最近 3 小时 & N=180 & 长周期溯源 \\
\end{longtable}
}

\subsection{请求字段固定值(固定配置)}\label{subsec:request-field-fixed-values}

为确保前端渲染所需字段完整,图查询请求固定携带:

\begin{itemize}
\tightlist
\item
  \texttt{allowed\_reltypes=null}(不做关系类型裁剪)
\end{itemize}

时间窗视图筛选告警边时固定使用:

\begin{itemize}
\tightlist
\item
  \texttt{only\_alarm=true}
\end{itemize}

接口字段的完整定义见:\texttt{../../80-规范/88-前端与中心机接口.md}。

\subsection{渲染模型}\label{subsec:rendering-model}

前端图渲染使用 AntV G6(\texttt{@antv/g6}),图数据模型与样式绑定关系保持固定。

\subsection{图谱渲染数据流(固定配置)}\label{subsec:graph-rendering-data-flow}

\begin{figure}[htbp]
\centering
\includegraphics[width=\textwidth,height=0.5\textheight,keepaspectratio]{figures/frontend-75/frontend-75-01.pdf}
\caption{图谱渲染数据流}
\label{fig:frontend-75-01}
\end{figure}

\subsection{后端数据模型到 G6 的映射(固定配置)}\label{subsec:backend-to-g6-mapping}

后端接口 \texttt{graph/query} 返回数据包含:

\begin{itemize}
\tightlist
\item
  \texttt{nodes{[}{]}}:节点对象,包含 \texttt{uid/ntype/key/props} 字段
\item
  \texttt{edges{[}{]}}:边对象,包含 \texttt{src\_uid/dst\_uid/rtype/props} 字段
\end{itemize}

G6 映射规则如下:

\begin{enumerate}
\def\labelenumi{\arabic{enumi}.}
\tightlist
\item
  G6 Node

  \begin{itemize}
  \tightlist
  \item
    \texttt{id\ =\ node.uid}
  \item
    \texttt{data\ =\ node}(完整保留)
  \item
    \texttt{label}:从 \texttt{node.props} 与 \texttt{node.uid} 生成(见 2.2)
  \end{itemize}
\item
  G6 Edge

  \begin{itemize}
  \tightlist
  \item
    \texttt{source\ =\ edge.src\_uid}
  \item
    \texttt{target\ =\ edge.dst\_uid}
  \item
    \texttt{data\ =\ edge}(完整保留)
  \item
    \texttt{id}:按固定规则生成(见 2.3)
  \end{itemize}
\end{enumerate}

\textbf{数据映射示例}:

{\def\LTcaptype{none} % do not increment counter
\footnotesize
\begin{longtable}[]{@{}p{0.20\textwidth}p{0.12\textwidth}p{0.28\textwidth}p{0.30\textwidth}@{}}
\toprule\noalign{}
数据层级 & 后端字段 & G6 字段 & 示例值 \\
\midrule\noalign{}
\endhead
\bottomrule\noalign{}
\endlastfoot
\textbf{节点标识} & \texttt{uid} & \texttt{id} & \texttt{"host-12345"} \\
\textbf{节点类型} & \texttt{ntype} & \texttt{data.ntype} & \texttt{"Host"} \\
\textbf{节点键} & \texttt{key} & \texttt{data.key} & \texttt{"/proc/1234"} \\
\textbf{节点属性} & \texttt{props} & \texttt{data.props} & \texttt{\{host.name:\ "web-01"\}} \\
\textbf{节点标签} & - & \texttt{label} & \texttt{"web-01"} (从 props 生成) \\
\textbf{边源节点} & \texttt{src\_uid} & \texttt{source} & \texttt{"host-12345"} \\
\textbf{边目标节点} & \texttt{dst\_uid} & \texttt{target} & \texttt{"proc-67890"} \\
\textbf{边关系类型} & \texttt{rtype} & \texttt{data.rtype} & \texttt{"spawned"} \\
\textbf{边属性} & \texttt{props} & \texttt{data.props} & \texttt{\{is\_alarm:\ true\}} \\
\textbf{边标识} & - & \texttt{id} & \texttt{"e-a3f5c9d2b1e4..."} (SHA1 生成) \\
\end{longtable}
}

\subsection{节点标签生成规则(固定配置)}\label{subsec:node-label-generation-rules}

节点展示标签按以下规则生成(从上到下依次匹配):

\begin{enumerate}
\def\labelenumi{\arabic{enumi}.}
\tightlist
\item
  Host 节点:

  \begin{itemize}
  \tightlist
  \item
    \texttt{props{[}"host.name"{]}} 非空则用该值
  \item
    否则使用 \texttt{props{[}"host.id"{]}}
  \end{itemize}
\item
  Process 节点:

  \begin{itemize}
  \tightlist
  \item
    \texttt{props{[}"process.name"{]}} 非空则用该值
  \item
    否则使用 \texttt{props{[}"process.entity\_id"{]}}
  \end{itemize}
\item
  User 节点:

  \begin{itemize}
  \tightlist
  \item
    \texttt{props{[}"user.name"{]}} 非空则用该值
  \item
    否则使用 \texttt{props{[}"user.id"{]}}
  \end{itemize}
\item
  IP 节点:\texttt{props{[}"ip"{]}}
\item
  Domain 节点:\texttt{props{[}"domain.name"{]}}
\item
  File 节点:\texttt{props{[}"file.path"{]}}
\item
  兜底:\texttt{node.uid}
\end{enumerate}

\subsection{边 ID 生成规则(固定配置)}\label{subsec:edge-id-generation-rules}

后端边对象未提供独立稳定 ID,前端需为 G6 边生成稳定 \texttt{id},生成规则如下:

\begin{enumerate}
\def\labelenumi{\arabic{enumi}.}
\tightlist
\item
  构造原始串:\texttt{raw\ =\ src\_uid\ +\ "\textbar{}"\ +\ rtype\ +\ "\textbar{}"\ +\ dst\_uid\ +\ "\textbar{}"\ +\ ts\_float\ +\ "\textbar{}"\ +\ event\_id}
\item
  计算 \texttt{sha1(raw)} 的前 16 位
\item
  拼接为:\texttt{e-\textless{}sha1\_16\textgreater{}}
\end{enumerate}

其中:

\begin{itemize}
\tightlist
\item
  \texttt{ts\_float} 取 \texttt{edge.props{[}"ts\_float"{]}}(数值,秒)
\item
  \texttt{event\_id} 取 \texttt{edge.props{[}"event.id"{]}}(字符串)
\end{itemize}

\textbf{代码示例}:

\begin{verbatim}
// 生成边 ID(前端实现)
function generateEdgeId(edge) {
  const raw = `${edge.src_uid}|${edge.rtype}|${edge.dst_uid}|` +
              `${edge.props["ts_float"]}|${edge.props["event.id"]}`;
  const sha1 = crypto.subtle.digest('SHA-1', raw);
  const hash16 = sha1.substring(0, 16);
  return `e-${hash16}`;
}
\end{verbatim}

\subsection{布局与交互能力(固定配置)}\label{subsec:layout-interaction-capabilities}

图布局采用有向层次布局(Dagre),参数配置如下:

\begin{itemize}
\tightlist
\item
  \texttt{rankdir="LR"}
\item
  \texttt{nodesep=30}
\item
  \texttt{ranksep=60}
\end{itemize}

\textbf{配置示例}:

\begin{verbatim}
// G6 布局配置
const layout = {
  type: 'dagre',
  rankdir: 'LR',
  nodesep: 30,
  ranksep: 60,
};

// 交互模式配置
const modes = {
  default: [
    'drag-canvas',
    'zoom-canvas',
    'drag-node',
  ],
};
\end{verbatim}

交互能力配置如下:

\begin{itemize}
\tightlist
\item
  拖动画布
\item
  缩放画布
\item
  拖动节点
\end{itemize}

\subsection{交互规则}\label{subsec:interaction-rules}

\subsection{选中与详情面板(固定配置)}\label{subsec:selection-detail-panel}

用户点选节点或边后,页面展示详情面板:

\begin{itemize}
\tightlist
\item
  Node 详情:展示 \texttt{uid/ntype/key/props}(JSON 格式)
\item
  Edge 详情:展示 \texttt{src\_uid/dst\_uid/rtype/props}(JSON 格式)
\end{itemize}

详情面板用于现场演示的证据解释,需重点展示 \texttt{event.id} 与 \texttt{custom.evidence.event\_ids{[}{]}}。

\subsection{高亮规则(固定配置)}\label{subsec:highlighting-rules}

边的颜色与线型高亮规则如下:

\begin{enumerate}
\def\labelenumi{\arabic{enumi}.}
\tightlist
\item
  告警边高亮:当 \texttt{edge.props.is\_alarm=true} 时,边样式为红色加粗
\item
  关键路径边高亮:当 \texttt{edge.props{[}"analysis.is\_path\_edge"{]}=true} 时,边样式为蓝色加粗
\item
  普通边:灰色细线
\end{enumerate}

当同一条边同时满足条件 1 与 2 时,关键路径边样式优先于告警边样式。

\textbf{高亮效果示意}:

{\def\LTcaptype{none} % do not increment counter
\footnotesize
\begin{longtable}[]{@{}p{0.18\textwidth}p{0.10\textwidth}p{0.20\textwidth}p{0.20\textwidth}p{0.22\textwidth}@{}}
\toprule\noalign{}
边类型 & 判断条件 & 样式配置 & 颜色值 & 线宽 \\
\midrule\noalign{}
\endhead
\bottomrule\noalign{}
\endlastfoot
告警边 & \texttt{is\_alarm=true} & 红色加粗 & \texttt{\#ff4d4f} & 3px \\
关键路径边 & \texttt{analysis.is\_path\_edge=true} & 蓝色加粗 & \texttt{\#1890ff} & 3px \\
普通边 & 默认 & 灰色细线 & \texttt{\#d9d9d9} & 1px \\
\end{longtable}
}

\textbf{代码示例}:

\begin{verbatim}
// 边样式映射函数
function getEdgeStyle(edge) {
  const props = edge.props || {};

  // 优先级:关键路径 > 告警 > 普通
  if (props["analysis.is_path_edge"] === true) {
    return {
      stroke: '#1890ff',
      lineWidth: 3,
      lineAppendWidth: 3,
    };
  }

  if (props.is_alarm === true) {
    return {
      stroke: '#ff4d4f',
      lineWidth: 3,
      lineAppendWidth: 3,
    };
  }

  // 普通边
  return {
    stroke: '#d9d9d9',
    lineWidth: 1,
    lineAppendWidth: 1,
  };
}
\end{verbatim}

\textbf{样式优先级流程图}:

\begin{figure}[htbp]
\centering
\includegraphics[width=\textwidth,height=0.5\textheight,keepaspectratio]{figures/frontend-75/frontend-75-02.pdf}
\caption{边样式优先级流程}
\label{fig:frontend-75-02}
\end{figure}

\subsection{任务触发与轮询}\label{subsec:task-trigger-polling}

\subsection{交互流程图(完整闭环)}\label{subsec:interaction-flow-complete-loop}

\begin{figure}[htbp]
\centering
\includegraphics[width=\textwidth,height=0.5\textheight,keepaspectratio]{figures/frontend-75/frontend-75-03.pdf}
\caption{前端交互流程完整闭环}
\label{fig:frontend-75-03}
\end{figure}

\subsection{任务触发(固定配置)}\label{subsec:task-trigger}

溯源任务触发流程如下:

\begin{enumerate}
\def\labelenumi{\arabic{enumi}.}
\tightlist
\item
  用户在时间窗视图中选中一个节点
\item
  页面使用该节点的 \texttt{uid} 作为 \texttt{target\_node\_uid}
\item
  使用当前页面时间窗(由``最近 N 分钟''计算得到)作为 \texttt{start\_ts/end\_ts}
\item
  调用 \texttt{POST\ /api/v1/analysis/tasks} 创建任务
\end{enumerate}

\subsection{轮询策略(固定配置)}\label{subsec:polling-strategy}

任务创建成功后,前端启动轮询机制,策略如下:

\begin{itemize}
\tightlist
\item
  轮询接口:\texttt{GET\ /api/v1/analysis/tasks/\{task\_id\}}
\item
  轮询间隔:\texttt{1s}
\item
  停止条件:\texttt{task.status} 为 \texttt{succeeded} 或 \texttt{failed}
\end{itemize}

\textbf{轮询状态机}:

{\def\LTcaptype{none} % do not increment counter
\footnotesize
\begin{longtable}[]{@{}p{0.20\textwidth}p{0.12\textwidth}p{0.28\textwidth}p{0.30\textwidth}@{}}
\toprule\noalign{}
任务状态 & 含义 & 前端行为 & 后续动作 \\
\midrule\noalign{}
\endhead
\bottomrule\noalign{}
\endlastfoot
\texttt{queued} & 任务已创建,等待执行 & 显示"等待中" (0\%) & 继续轮询 \\
\texttt{running} & 任务执行中 & 显示进度条 (5-95\%) & 继续轮询 \\
\texttt{succeeded} & 任务成功完成 & 停止轮询,显示成功 & 查询写回边并高亮 \\
\texttt{failed} & 任务失败 & 停止轮询,显示错误 & 展示 error.message \\
\end{longtable}
}

\textbf{轮询实现示例}:

\begin{verbatim}
// 前端轮询实现
async function pollTaskStatus(taskId) {
  const pollInterval = 1000; // 1s
  let shouldStop = false;

  const timer = setInterval(async () => {
    try {
      const response = await fetch(`/api/v1/analysis/tasks/${taskId}`);
      const task = await response.json();

      // 更新进度 UI
      updateProgressBar(task.progress);

      // 检查终止条件
      if (task.status === 'succeeded') {
        clearInterval(timer);
        await loadAnalysisEdges(taskId);
        highlightCriticalPath();
      } else if (task.status === 'failed') {
        clearInterval(timer);
        showErrorMessage(task.error?.message || '任务失败');
      }
    } catch (error) {
      clearInterval(timer);
      showErrorMessage('网络请求失败');
    }
  }, pollInterval);
}
\end{verbatim}

\subsection{写回边读取与渲染(固定配置)}\label{subsec:writeback-reading-rendering}

当 \texttt{task.status="succeeded"} 时,页面执行以下操作:

\begin{enumerate}
\def\labelenumi{\arabic{enumi}.}
\tightlist
\item
  调用 \texttt{POST\ /api/v1/graph/query},参数配置如下:

  \begin{itemize}
  \tightlist
  \item
    \texttt{action="analysis\_edges\_by\_task"}
  \item
    \texttt{task\_id=\textless{}task\_id\textgreater{}}
  \item
    \texttt{only\_path=true}
  \end{itemize}
\item
  将返回的边集合叠加到当前图中,并触发关键路径高亮(参见 3.2 节)。
\end{enumerate}

当 \texttt{task.status="failed"} 时,页面展示失败原因(后端返回的 \texttt{error.message})。

\textbf{API 响应示例}:

\begin{verbatim}
// POST /api/v1/graph/query
// action=analysis_edges_by_task
{
  "nodes": [
    {
      "uid": "host-1921681110",
      "ntype": "Host",
      "key": null,
      "props": {
        "host.name": "web-01",
        "host.id": "192.168.1.110"
      }
    },
    {
      "uid": "proc-12345",
      "ntype": "Process",
      "key": "/proc/12345",
      "props": {
        "process.name": "nginx",
        "process.pid": 12345,
        "process.entity_id": "abcdef123456"
      }
    }
  ],
  "edges": [
    {
      "src_uid": "host-1921681110",
      "dst_uid": "proc-12345",
      "rtype": "spawned",
      "props": {
        "ts_float": 1737033600.123,
        "event.id": "event-67890",
        "is_alarm": false,
        "analysis.is_path_edge": true,
        "analysis.path_score": 0.95,
        "analysis.path_position": 3
      }
    }
  ]
}
\end{verbatim}
