\subsection{Falco 采集与 ECS 转换}\label{21-falco-collection-ecs-conversion}

\subsubsection{组件与文件位置}\label{1-components-file-locations}

{\def\LTcaptype{none} % do not increment counter
\footnotesize
\begin{longtable}[]{@{}p{0.25\textwidth}p{0.12\textwidth}p{0.53\textwidth}@{}}
\toprule\noalign{}
组件 & 位置 & 说明 \\
\midrule\noalign{}
\endhead
\bottomrule\noalign{}
\endlastfoot
Falco 引擎 & \texttt{client/docker-compose.yml} 的 \texttt{falco} 服务 & Falco 输出 JSONL 到共享卷 \\
ECS 转换器 & \texttt{client/sensor/falco/ecs-converter/falco\_json\_to\_ecs.py} & 读取 JSONL,转换为 ECS JSON,发布到 RabbitMQ \\
\end{longtable}
}

\subsubsection{Falco 组件架构}\label{11-falco-component-architecture}

\begin{figure}[htbp]
\centering
\includegraphics[width=\textwidth,height=0.5\textheight,keepaspectratio]{figures/client-51/client-51-01.pdf}
\caption{Falco 组件架构}
\label{fig:client-51-01}
\end{figure}

\subsubsection{采集输入}\label{2-collection-input}

\subsubsection{Falco 输出文件}\label{21-falco-output-file}

Falco 将事件以 JSONL 格式写入共享卷:

\begin{itemize}
\tightlist
\item
  路径:\texttt{/data/falco.jsonl}
\item
  格式:每行一个 JSON 对象(JSONL)
\end{itemize}

该文件由 \texttt{falco} 容器生成,由 \texttt{falco-ecs} 容器消费。

\subsubsection{Falco 容器关键配置}\label{22-falco-container-config}

Falco 以特权模式运行并挂载宿主机关键目录,配置固定为:

\begin{verbatim}
image: falcosecurity/falco:0.42.1
privileged: true
pid: host
volumes:
  - /var/run/docker.sock:/host/var/run/docker.sock
  - /dev:/host/dev
  - /proc:/host/proc:ro
  - /boot:/host/boot:ro
  - /lib/modules:/host/lib/modules:ro
  - /usr:/host/usr:ro
  - /etc:/host/etc:ro
  - /sys/kernel/debug:/sys/kernel/debug
  - /sys/kernel/tracing:/sys/kernel/tracing
  - ./data:/data
command:
  - falco
  - -o
  - json_output=true
  - -o
  - file_output.enabled=true
  - -o
  - file_output.filename=/data/falco.jsonl
\end{verbatim}

\subsubsection{Telemetry 规则(用于``能入图'')}\label{23-telemetry-rules-graph}

Falco 默认规则集主要用于``检测/告警'',而图谱重建需要一定比例的``事实 Telemetry''。

ATA系统通过在 Falco 容器中加载额外的 Telemetry 规则文件补充以下能力:

\begin{itemize}
\tightlist
\item
  \texttt{execve/execveat}:产生 \texttt{hostlog.process}(用于进程树 / 父子进程)
\item
  \texttt{open/openat/...}:产生 \texttt{hostbehavior.file} / \texttt{hostlog.file\_registry}(用于文件访问链路)
\item
  \texttt{connect}:产生 \texttt{hostbehavior.syscall}(用于 \texttt{Process\ →\ IP} 的网络连接边)
\end{itemize}

规则文件位置:

\begin{itemize}
\tightlist
\item
  \texttt{client/sensor/falco/rules.d/ata\_telemetry\_rules.yaml}
\end{itemize}

部署时通过容器挂载方式加载至 Falco 默认规则目录:

\begin{itemize}
\tightlist
\item
  \texttt{/etc/falco/rules.d}
\end{itemize}

\subsubsection{ECS 转换规则}\label{3-ecs-conversion-rules}

\subsubsection{转换器运行方式}\label{31-converter-operation}

\texttt{falco-ecs} 容器通过脚本 \texttt{falco\_json\_to\_ecs.py} 启动,参数配置如下:

\begin{verbatim}
--input /data/falco.jsonl
--follow
--queue data.falco
\end{verbatim}

\subsubsection{输出字段形态}\label{32-output-field-format}

Falco 转换器采用\textbf{嵌套对象形态}输出(而非点号扁平键)。

示例(节选):

\begin{verbatim}
{
  "@timestamp": "2026-01-14T12:00:00.000Z",
  "ecs": {"version": "9.2.0"},
  "event": {"kind": "event", "dataset": "hostlog.process"},
  "host": {"name": "client-01", "id": "h-1111111111111111"},
  "process": {"pid": 1234, "executable": "/usr/bin/bash"},
  "message": "...",
  "falco": { "...": "raw event payload" }
}
\end{verbatim}

说明:

\begin{enumerate}
\def\labelenumi{\arabic{enumi}.}
\tightlist
\item
  Falco 转换器输出 \texttt{ecs.version="9.2.0"};中心机入库前会执行最终规范化,确保权威口径一致(参见 \texttt{../../80-规范/81-ECS字段规范.md})。
\item
  Falco 原始事件完整保存在 \texttt{falco} 字段中,用于审计与回放。
\item
  Telemetry 的 \texttt{event.dataset} 根据事件类型映射为 \texttt{hostlog.process} / \texttt{hostbehavior.file} / \texttt{hostlog.file\_registry} / \texttt{hostbehavior.syscall},以满足 \texttt{../../80-规范/84-Neo4j实体图谱规范.md} 的抽取条件。
\item
  为确保三传感器数据在同一主机上的可关联性,转换器支持:

  \begin{itemize}
  \tightlist
  \item
    通过环境变量 \texttt{HOST\_NAME} 覆盖 \texttt{host.name}(权威值参见 \texttt{89-环境变量与配置规范.md})
  \item
    优先使用环境变量 \texttt{HOST\_ID} 覆盖 \texttt{host.id};当 \texttt{HOST\_ID} 缺失时,按 \texttt{81-ECS字段规范.md} 规则回退生成 \texttt{host.id}(\texttt{h-} + sha1(host.name){[}:16{]})。
  \end{itemize}
\end{enumerate}

\subsubsection{eventkind 规则}\label{33-event-kind-rules}

Falco 转换器依据 \texttt{priority} 与阈值判断事件性质:

\begin{itemize}
\tightlist
\item
  正常行为:\texttt{event.kind="event"}
\item
  异常行为:\texttt{event.kind="alert"}
\end{itemize}

阈值参数由转换器启动参数 \texttt{-\/-abnormal-priority} 指定,默认值为 \texttt{WARNING}。

\subsubsection{转换逻辑示例}\label{34-conversion-logic-example}

以下代码展示 Falco 事件到 ECS 的核心转换逻辑:

\begin{verbatim}
def falco_to_ecs(falco_event: dict) -> dict:
    """将 Falco JSONL 事件转换为 ECS 格式"""
    ecs_event = {
        "ecs": {"version": "9.2.0"},
        "event": {
            "kind": determine_event_kind(falco_event),  # 基于 priority 判断
            "dataset": map_dataset(falco_event),       # hostlog.process 等
        },
        "host": {
            "name": os.getenv("HOST_NAME", falco_event.get("host_name")),
            "id": get_or_generate_host_id()
        },
        "process": {
            "pid": falco_event.get("proc.pid"),
            "executable": falco_event.get("proc.exe"),
            "command": falco_event.get("proc.cmdline")
        },
        "message": falco_event.get("output", ""),
        "falco": falco_event  # 保留原始事件用于审计
    }
    return ecs_event

def determine_event_kind(falco_event: dict) -> str:
    """基于 priority 判断事件类型"""
    priority = falco_event.get("priority", "")
    if priority in ["WARNING", "ERROR", "CRITICAL"]:
        return "alert"
    return "event"

def map_dataset(falco_event: dict) -> str:
    """映射 event.dataset"""
    evt_type = falco_event.get("event_type", "")
    if evt_type == "execve":
        return "hostlog.process"
    elif evt_type in ["open", "openat"]:
        return "hostlog.file_registry"
    elif evt_type == "connect":
        return "hostbehavior.syscall"
    return "hostbehavior.default"
\end{verbatim}

\subsubsection{eventid 生成与幂等}\label{4-event-id-generation-idempotency}

\subsubsection{eventid 的来源}\label{41-event-id-source}

Falco 转换器不强制生成 \texttt{event.id}。客户机对外拉取接口在返回数据前会补全稳定的 \texttt{event.id}:

\begin{itemize}
\tightlist
\item
  当消息体已包含 \texttt{event.id}(扁平键或嵌套键)时直接透传;
\item
  当缺失 \texttt{event.id} 时,对消息体做 \texttt{sha1},取前 16 位生成:\texttt{evt-\textless{}sha1{[}:16{]}\textgreater{}}。
\end{itemize}

该补齐逻辑位于:\texttt{client/backend/queue/client.go} 的 \texttt{ensureEventID}。

权威规则参见:

\begin{itemize}
\tightlist
\item
  \texttt{../../80-规范/81-ECS字段规范.md}
\end{itemize}

\subsubsection{幂等与重复处理}\label{42-idempotency-duplicate-handling}

\begin{itemize}
\tightlist
\item
  RabbitMQ 队列层面保证增量拉取:消息被 \texttt{ack} 后不再重复返回。
\item
  OpenSearch 写入以 \texttt{event.id} 为幂等键,重复写入不会产生重复文档。
\end{itemize}

\subsubsection{队列投递}\label{5-queue-delivery}

Falco 转换器将数据投递至 RabbitMQ:

\begin{itemize}
\tightlist
\item
  AMQP:由环境变量 \texttt{RABBITMQ\_URL} 指定
\item
  队列:由环境变量 \texttt{RABBITMQ\_QUEUE} 指定,默认为 \texttt{data.falco}
\end{itemize}

队列语义参见:\texttt{54-RabbitMQ缓冲与队列语义.md}。

\subsubsection{故障处理}\label{6-error-handling}

\begin{enumerate}
\def\labelenumi{\arabic{enumi}.}
\tightlist
\item
  Falco 未产生日志文件时,转换器会持续等待输入文件出现后开始处理。
\item
  RabbitMQ 连接断开时,转换器会重连后继续发布。
\item
  任何无法解析的行会被跳过,转换器继续处理后续行以保证持续运行。
\end{enumerate}
