\section{前端模块}\label{sec:frontend-module}

\subsection{页面结构}\label{subsec:page-structure}

前端采用 Next.js App Router(\texttt{frontend/app/}),页面与代码文件的绑定关系固定如下。

\subsubsection{页面结构图}\label{subsubsec:page-structure-diagram}

\begin{figure}[htbp]
\centering
\includegraphics[width=\textwidth,height=0.5\textheight,keepaspectratio]{figures/frontend-74/frontend-74-01.pdf}
\caption{Next.js 前端页面结构}
\label{fig:frontend-74-01}
\end{figure}

\subsubsection{页面路由与文件映射}\label{subsubsec:page-routing-file-mapping}

{\def\LTcaptype{none} % do not increment counter
\footnotesize
\begin{longtable}[]{@{}p{0.20\textwidth}p{0.25\textwidth}p{0.45\textwidth}@{}}
\toprule\noalign{}
路由 & 页面用途 & 代码位置 \\
\midrule\noalign{}
\endhead
\bottomrule\noalign{}
\endlastfoot
{\scriptsize\texttt{/}} & 项目首页与入口 & {\scriptsize\texttt{frontend/app/page.tsx}} \\
{\scriptsize\texttt{/dashboard}} & 总览大盘 & {\scriptsize\texttt{frontend/app/dashboard/page.tsx}} \\
{\scriptsize\texttt{/trace}} & 溯源分析主界面 & {\scriptsize\texttt{frontend/app/trace/page.tsx}} \\
\end{longtable}
}

\subsubsection{布局与导航}\label{subsubsec:layout-navigation}

布局与导航组件的绑定关系如下:

\begin{enumerate}
\def\labelenumi{\arabic{enumi}.}
\tightlist
\item
  全站根布局:\texttt{frontend/app/layout.tsx}

  \begin{itemize}
  \tightlist
  \item
    集成主题切换功能(\texttt{ThemeProvider} + \texttt{ModeToggle})
  \end{itemize}
\item
  侧边栏组件:\texttt{frontend/components/sidebar/app-sidebar.tsx}

  \begin{itemize}
  \tightlist
  \item
    固定菜单项:主页、总览、溯源分析
  \end{itemize}
\item
  \texttt{/dashboard} 与 \texttt{/trace} 共享侧边栏布局:

  \begin{itemize}
  \tightlist
  \item
    \texttt{frontend/app/dashboard/layout.tsx}
  \item
    \texttt{frontend/app/trace/layout.tsx}
  \end{itemize}
\end{enumerate}

\subsubsection{组件关系图}\label{subsubsec:component-relationship-diagram}

\begin{figure}[htbp]
\centering
\includegraphics[width=\textwidth,height=0.5\textheight,keepaspectratio]{figures/frontend-74/frontend-74-02.pdf}
\caption{前端组件关系}
\label{fig:frontend-74-02}
\end{figure}

\subsection{数据流与状态管理}\label{subsec:data-flow-state-management}

前端采用``页面内状态 + 明确的请求边界''的状态管理策略:

\begin{enumerate}
\def\labelenumi{\arabic{enumi}.}
\tightlist
\item
  页面状态仅保存渲染所需的最小数据集
\item
  所有后端数据通过 \texttt{88-前端与中心机接口.md} 统一获取
\item
  状态变更由用户操作或轮询驱动,避免隐式副作用
\end{enumerate}

\subsubsection{Dashboard 数据流}\label{subsubsec:dashboard-data-flow}

Dashboard 页面展示四类概览卡片,当前实现以静态组件渲染为主。数据填充采用以下数据源:

\begin{itemize}
\tightlist
\item
  Telemetry 统计:\texttt{POST /api/v1/events/search}
\item
  Raw/Canonical 告警统计:\texttt{POST /api/v1/findings/search}
\item
  任务统计:\texttt{GET /api/v1/analysis/tasks}
\end{itemize}

Dashboard 页面仅读取聚合统计数据,不触发溯源任务与图查询。

\subsubsection{Trace 数据流}\label{subsubsec:trace-data-flow}

Trace 页面的数据流由以下 4 个阶段组成:

\begin{enumerate}
\def\labelenumi{\arabic{enumi}.}
\tightlist
\item
  时间窗输入(最近 N 分钟)→ 图数据拉取
\item
  图交互 → 创建溯源任务
\item
  任务轮询 → 读取写回边并高亮
\item
  TTP 相似度分析 → 输出 Top-3 组织与解释字段
\end{enumerate}

所有 HTTP 请求通过中心机后端完成,前端不直连 OpenSearch 与 Neo4j。

\subsubsection{完整数据流图}\label{subsubsec:complete-data-flow-diagram}

\begin{figure}[htbp]
\centering
\includegraphics[width=\textwidth,height=0.5\textheight,keepaspectratio]{figures/frontend-74/frontend-74-03.pdf}
\caption{前端完整数据流}
\label{fig:frontend-74-03}
\end{figure}

\subsubsection{数据流阶段说明}\label{subsubsec:data-flow-stage-description}

{\def\LTcaptype{none} % do not increment counter
\begin{longtable}[]{@{}lllll@{}}
\toprule\noalign{}
阶段 & 触发条件 & 数据流向 & 涉及组件 & 数据存储 \\
\midrule\noalign{}
\endhead
\bottomrule\noalign{}
\endlastfoot
\textbf{初始化} & 页面加载 & 前端 → 后端 → Neo4j & 图谱组件 & 图节点与边 \\
\textbf{图查询} & 时间窗变更 & 前端 → 后端 → Neo4j & 图谱组件 & 时间窗内子图 \\
\textbf{任务创建} & 节点点击 & 前端 → 后端 → Neo4j & 任务面板 & 任务文档 \\
\textbf{进度轮询} & 自动触发 & 前端 → 后端 → Neo4j & 任务列表 & 任务状态字段 \\
\textbf{写回读取} & 任务完成 & 前端 → 后端 → Neo4j & 图谱组件 & analysis\_edges\_by\_task \\
\textbf{TTP 分析} & 任务完成 & 前端 → 后端(缓存) & 报告查看器 & ttp\_similarity 字段 \\
\end{longtable}
}

\subsection{接口调用边界}\label{subsec:api-call-boundaries}

\subsubsection{网络边界约束}\label{subsubsec:network-boundary-constraints}

前端的网络边界约束如下:

\begin{itemize}
\tightlist
\item
  仅调用中心机 HTTP API(\texttt{/api/v1/*} 与 \texttt{/health})
\item
  不直连 OpenSearch(\texttt{/9200})与 Neo4j(\texttt{/7687})
\end{itemize}

\subsubsection{Trace 页面接口绑定}\label{subsubsec:trace-page-api-binding}

Trace 页面在不同阶段调用接口的绑定关系如下(字段结构以 \texttt{88-前端与中心机接口.md} 为准):

{\def\LTcaptype{none} % do not increment counter
\begin{longtable}[]{@{}lllll@{}}
\toprule\noalign{}
阶段 & 触发动作 & 方法 & 路径 & 目的 \\
\midrule\noalign{}
\endhead
\bottomrule\noalign{}
\endlastfoot
图拉取 & 设置时间窗并刷新 & POST & \texttt{/api/v1/graph/query} & 获取时间窗图边与节点 \\
创建任务 & 点选目标节点 & POST & \texttt{/api/v1/analysis/tasks} & 创建溯源任务并得到 \texttt{task\_id} \\
轮询任务 & 创建任务后自动开始 & GET & \texttt{/api/v1/analysis/tasks/\{task\_id\}} & 获取任务状态与进度 \\
读取写回 & 任务完成后自动触发 & POST & \texttt{/api/v1/graph/query} & 读取 \texttt{analysis\_edges\_by\_task} 并高亮 \\
相似度 & 任务完成后自动展示 & - & - & 优先从任务文档 \texttt{task.result.ttp\_similarity.*} 读取并展示 Top-3;必要时回退调用 \texttt{/api/v1/analysis/ttp-similarity} \\
\end{longtable}
}

\subsubsection{前端代理规则}\label{subsubsec:frontend-proxy-rules}

ATA系统运行时前端端口为 \texttt{3000},后端端口为 \texttt{8001}。为满足浏览器同源策略,前端通过 Next.js rewrite 规则将以下路径转发到本地后端:

\begin{itemize}
\tightlist
\item
  \texttt{/api/*} → \texttt{http://localhost:8001/api/*}
\item
  \texttt{/health} → \texttt{http://localhost:8001/health}
\end{itemize}

配置文件位置:\texttt{frontend/next.config.ts}。

\paragraph{代理配置示例}\label{para:proxy-config-example}

\begin{verbatim}
// frontend/next.config.ts
import type { NextConfig } from "next";

const nextConfig: NextConfig = {
  async rewrites() {
    return [
      {
        source: "/api/:path*",
        destination: "http://localhost:8001/api/:path*",
      },
      {
        source: "/health",
        destination: "http://localhost:8001/health",
      },
    ];
  },
};

export default nextConfig;
\end{verbatim}

\paragraph{端口映射表}\label{para:port-mapping-table}

{\def\LTcaptype{none} % do not increment counter
\begin{longtable}[]{@{}llll@{}}
\toprule\noalign{}
源端 & 目标端口 & 协议 & 说明 \\
\midrule\noalign{}
\endhead
\bottomrule\noalign{}
\endlastfoot
\texttt{:3000} & \texttt{:8001} & HTTP & 前端 → 后端 API 请求 \\
\texttt{:8001} & \texttt{:7687} & Bolt & 后端 → Neo4j 图数据库 \\
\texttt{:8001} & \texttt{:9200} & HTTP & 后端 → OpenSearch 搜索引擎 \\
\end{longtable}
}
