\subsection{RabbitMQ 缓冲与队列语义}\label{24-rabbitmq-buffer-queue-semantics}

\subsubsection{队列命名}\label{1-queue-naming}

客户机侧使用 3 个固定队列分别接收三类数据源:

{\def\LTcaptype{none} % do not increment counter
\footnotesize
\begin{longtable}[]{@{}p{0.25\textwidth}p{0.12\textwidth}p{0.53\textwidth}@{}}
\toprule\noalign{}
数据源 & 默认队列名 & 环境变量 \\
\midrule\noalign{}
\endhead
\bottomrule\noalign{}
\endlastfoot
Falco & \texttt{data.falco} & \texttt{FALCO\_QUEUE} 或 \texttt{RABBITMQ\_QUEUE} \\
Filebeat & \texttt{data.filebeat} & \texttt{FILEBEAT\_QUEUE} 或 \texttt{RABBITMQ\_QUEUE} \\
Suricata & \texttt{data.suricata} & \texttt{SURICATA\_QUEUE} 或 \texttt{RABBITMQ\_QUEUE} \\
\end{longtable}
}

队列名称的实际取值以 \texttt{client/docker-compose.yml} 配置为准。

\subsubsection{消息流转架构}\label{11-message-flow-architecture}

\begin{figure}[htbp]
\centering
\includegraphics[width=\textwidth,height=0.5\textheight,keepaspectratio]{figures/client-54/client-54-01.pdf}
\caption{消息流转架构}
\label{fig:client-54-01}
\end{figure}

\subsubsection{消费语义与增量语义}\label{2-consumption-incremental-semantics}

中心机通过客户机拉取接口获取数据时,客户机 Go 后端从队列中逐条读取消息并确认:

\begin{itemize}
\tightlist
\item
  使用 \texttt{basic.get} 方法从队列拉取消息
\item
  拉取到的消息在返回前执行 \texttt{ack} 确认
\item
  队列为空时,接口返回空数组
\end{itemize}

基于上述机制,增量语义由 RabbitMQ 队列自身保证,无需使用游标机制。

\subsubsection{队列消费时序}\label{21-queue-consumption-timing}

\begin{figure}[htbp]
\centering
\includegraphics[width=\textwidth,height=0.5\textheight,keepaspectratio]{figures/client-54/client-54-02.pdf}
\caption{队列消费时序}
\label{fig:client-54-02}
\end{figure}

\subsubsection{幂等与重复处理}\label{3-idempotency-duplicate-handling}

\paragraph{拉取层幂等}\label{31-pull-layer-idempotency}

同一条消息被 \texttt{ack} 确认后,后续拉取操作将不再返回该消息,从而避免重复处理。

\paragraph{入库层幂等}\label{32-storage-layer-idempotency}

中心机入库时使用 \texttt{event.id} 字段去重,重复写入不会产生重复文档(详见 \texttt{../../80-规范/81-ECS字段规范.md})。

\subsubsection{断连与恢复}\label{4-disconnection-recovery}

当 RabbitMQ 连接断开时,系统按以下流程处理:

\begin{enumerate}
\def\labelenumi{\arabic{enumi}.}
\tightlist
\item
  客户机 Go 后端在下次拉取请求时自动重连
\item
  若重连失败,接口返回 500 错误并在响应体中携带 \texttt{error} 字段
\item
  中心机在下一轮轮询时继续重试,并同步更新注册表中的错误信息
\end{enumerate}

\subsubsection{容量边界}\label{5-capacity-boundaries}

RabbitMQ 的容量边界取决于宿主机磁盘空间与 RabbitMQ 默认策略。

为确保演示环境稳定运行:

\begin{enumerate}
\def\labelenumi{\arabic{enumi}.}
\tightlist
\item
  靶场运行前必须清理历史队列数据
\item
  复现与演示过程中若出现磁盘空间不足,按 \texttt{../../90-运维与靶场/95-重置复现与排障.md} 执行清理后重新运行
\end{enumerate}
