\subsection{OpenSearch 存储与索引治理}\label{center-62-opensearch-storage}

\subsection{模块职责与边界}\label{center-62-module-boundaries}

OpenSearch 模块承担中心机侧事实、告警与任务元数据的权威存储与检索,具体职责包括:

\begin{enumerate}
\def\labelenumi{\arabic{enumi}.}
\tightlist
\item
  \textbf{索引体系}:定义索引命名规范、按日滚动策略与生命周期保留策略;
\item
  \textbf{入库路由}:依据 \texttt{event.kind} 与 \texttt{event.dataset} 将文档分发至目标索引;
\item
  \textbf{字段处理}:对 ECS 文档执行三时间字段处理、扁平键兼容及基础校验;
\item
  \textbf{Store-first 检测}:触发 OpenSearch Security Analytics 扫描 Telemetry 并生成 Findings;
\item
  \textbf{融合去重}:将 Raw Findings 融合为 Canonical Findings 并写回 OpenSearch;
\item
  \textbf{任务存储}:持久化溯源任务的状态与元数据,支持前端轮询。
\end{enumerate}

本模块不负责:

\begin{itemize}
\tightlist
\item
  Neo4j 图谱建模与查询(见 \texttt{64-Neo4j入图与图查询.md} 与 \texttt{../../80-规范/84-Neo4j实体图谱规范.md});
\item
  溯源算法与结果写回(见 \texttt{../../50-详细设计/分析/} 与 \texttt{../../80-规范/85-溯源结果写回规范.md});
\item
  客户机采集与接口(见 \texttt{../../50-详细设计/客户机/} 与 \texttt{../../80-规范/87-客户机与中心机接口.md})。
\end{itemize}

\subsection{索引体系与命名}\label{center-62-index-system}

\subsection{索引清单(系统运行时必须存在)}\label{center-62-index-list}

{\def\LTcaptype{none} % do not increment counter
\footnotesize
\begin{longtable}[]{@{}p{0.20\textwidth}p{0.12\textwidth}p{0.28\textwidth}p{0.30\textwidth}@{}}
\toprule\noalign{}
索引模式 & 数据对象 & 写入方 & 用途 \\
\midrule\noalign{}
\endhead
\bottomrule\noalign{}
\endlastfoot
\texttt{ecs-events-YYYY-MM-DD} & Telemetry & 中心机流水线 Step 2 & 事实事件检索、Store-first 检测输入 \\
\texttt{raw-findings-YYYY-MM-DD} & Raw Findings & 中心机流水线 Step 3 & 原始告警审计、融合输入 \\
\texttt{canonical-findings-YYYY-MM-DD} & Canonical Findings & 中心机流水线 Step 3 & 图谱与溯源的主输入 \\
\texttt{client-registry} & 客户机注册表 & 客户机注册 + 流水线更新 & 客户机列表、游标、在线状态 \\
\texttt{analysis-tasks-YYYY-MM-DD} & Trace Task & Analysis 模块 & 异步任务状态与进度轮询 \\
\end{longtable}
}

\begin{quote}
说明:索引保留策略见第 6 节;ECS 字段规范详见 \texttt{../../80-规范/81-ECS字段规范.md}。
\end{quote}

\subsection{索引命名规则(必须遵守)}\label{center-62-index-naming}

\begin{enumerate}
\def\labelenumi{\arabic{enumi}.}
\tightlist
\item
  所有按日滚动的索引必须使用连字符日期:\texttt{YYYY-MM-DD}。
\item
  索引名不得出现点号日期(例如 \texttt{2026.01.13}),避免 Security Analytics 的 pattern 解析问题。
\item
  \texttt{client-registry} 不按日滚动,索引名固定为 \texttt{client-registry}。
\end{enumerate}

\subsection{索引体系架构}\label{center-62-index-architecture}

\begin{figure}[htbp]
\centering
\includegraphics[width=\textwidth,height=0.5\textheight,keepaspectratio]{figures/center-62/center-62-01.pdf}
\caption{索引体系架构}
\label{fig:center-62-01}
\end{figure}

\textbf{架构说明}:

\begin{itemize}
\tightlist
\item
  \textbf{Telemetry 索引}(\texttt{ecs-events-*}):接收 Falco、Filebeat、Suricata 三类传感器的原始事件
\item
  \textbf{Raw Findings 索引}(\texttt{findings-raw-*}):存储 Security Analytics 生成的原始告警
\item
  \textbf{Canonical Findings 索引}(\texttt{findings-canonical-*}):存储融合去重后的规范化告警
\item
  \textbf{Tasks 索引}(\texttt{analysis-tasks-*}):存储异步溯源任务的状态与结果
\end{itemize}

\subsection{入库路由与字段处理}\label{center-62-routing-processing}

\subsection{路由规则(权威)}\label{center-62-routing-rules}

对每条输入文档,中心机须按以下规则路由(伪代码表达):

\begin{itemize}
\tightlist
\item
  当 \texttt{event.kind\ ==\ "event"}:写入 \texttt{ecs-events-*}
\item
  当 \texttt{event.kind\ ==\ "alert"} 且 \texttt{event.dataset\ ==\ "finding.canonical"}:写入 \texttt{canonical-findings-*}
\item
  当 \texttt{event.kind\ ==\ "alert"} 且 \texttt{event.dataset\ !=\ "finding.canonical"}:写入 \texttt{raw-findings-*}
\end{itemize}

\subsection{数据流向与写入流程}\label{center-62-data-flow}

\begin{figure}[htbp]
\centering
\includegraphics[width=\textwidth,height=0.5\textheight,keepaspectratio]{figures/center-62/center-62-02.pdf}
\caption{数据流向与写入流程}
\label{fig:center-62-02}
\end{figure}

\textbf{流程说明}:

\begin{enumerate}
\def\labelenumi{\arabic{enumi}.}
\tightlist
\item
  \textbf{客户机采集}:Falco/Filebeat/Suricata 采集数据并写入本地缓冲区;
\item
  \textbf{中心机拉取}:Step1 通过 cursor 机制批量拉取客户机数据;
\item
  \textbf{字段处理}:Step2 执行 ECS 校验、三时间字段处理及路由判断;
\item
  \textbf{OpenSearch 写入}:按路由规则写入对应索引,利用 \texttt{event.id} 实现幂等去重;
\item
  \textbf{检测与融合}:Step3 触发 Security Analytics 检测并融合告警;
\item
  \textbf{任务管理}:Step4 创建溯源任务并持续更新任务状态。
\end{enumerate}

\subsection{三时间字段处理(必须执行)}\label{center-62-time-fields}

中心机写入 OpenSearch 前必须保证三时间字段满足 \texttt{../../80-规范/81-ECS字段规范.md}:

\begin{itemize}
\tightlist
\item
  \texttt{@timestamp}:主时间轴。若缺失,必须从 \texttt{event.created} 推导;若仍无法得到,中心机必须丢弃该文档。
\item
  \texttt{event.created}:观察时间。若缺失,中心机必须回填为 \texttt{@timestamp}。
\item
  \texttt{event.ingested}:入库时间。中心机必须覆盖为"当前入库时间",不得使用上游携带值。
\end{itemize}

\subsection{ECS 字段映射与处理}\label{center-62-ecs-mapping}

中心机写入 OpenSearch 前,须对关键字段进行映射和转换以确保符合 ECS 规范:

{\def\LTcaptype{none} % do not increment counter
\footnotesize
\begin{longtable}[]{@{}p{0.18\textwidth}p{0.10\textwidth}p{0.20\textwidth}p{0.20\textwidth}p{0.22\textwidth}@{}}
\toprule\noalign{}
ECS 字段 & 数据类型 & 必需 & 处理规则 & 说明 \\
\midrule\noalign{}
\endhead
\bottomrule\noalign{}
\endlastfoot
\texttt{@timestamp} & date & ✅ & 优先使用原始值;缺失时从 \texttt{event.created} 推导;仍缺失则丢弃文档 & 主时间轴,用于日志检索和时序分析 \\
\texttt{event.id} & keyword & ✅ & 若缺失则生成 UUID:\texttt{\{client\_id\}-\{sensor\_type\}-\{timestamp\}-\{seq\}} & 幂等去重的唯一标识 \\
\texttt{event.kind} & keyword & ✅ & 枚举值:\texttt{event} / \texttt{alert} / \texttt{state} & 路由判断的核心字段 \\
\texttt{event.category} & keyword & ❌ & 映射规则:\texttt{file} → \texttt{file},\texttt{network} → \texttt{network},\texttt{process} → \texttt{process} & 用于前端分类展示 \\
\texttt{event.dataset} & keyword & ✅ & 格式:\texttt{\{sensor\}.\{type\}},如 \texttt{falco.syscall}、\texttt{finding.canonical} & 区分数据来源和告警类型 \\
\texttt{event.created} & date & ✅ & 缺失时回填为 \texttt{@timestamp} & 事件首次被观察到的时间 \\
\texttt{event.ingested} & date & ✅ & 覆盖为当前入库时间(UTC 毫秒时间戳) & 中心机接收时间,用于监控延迟 \\
\texttt{source.ip} & ip & ❌ & 保持原值,支持 IPv4/IPv6 & 源地址,用于关联分析 \\
\texttt{source.port} & long & ❌ & 范围校验:0-65535 & 源端口 \\
\texttt{destination.ip} & ip & ❌ & 保持原值,支持 IPv4/IPv6 & 目标地址 \\
\texttt{destination.port} & long & ❌ & 范围校验:0-65535 & 目标端口 \\
\texttt{process.pid} & long & ❌ & 保持原值 & 进程 ID \\
\texttt{process.executable} & keyword & ❌ & 规范化绝对路径,统一使用 \texttt{/} 分隔符 & 进程可执行文件路径 \\
\texttt{file.path} & keyword & ❌ & 规范化绝对路径,统一使用 \texttt{/} 分隔符 & 文件路径 \\
\texttt{file.name} & keyword & ❌ & 从 \texttt{file.path} 中提取文件名 & 文件名 \\
\texttt{user.name} & keyword & ❌ & 保持原值 & 用户名 \\
\texttt{host.name} & keyword & ✅ & 使用客户机注册表中的 \texttt{hostname} & 主机名 \\
\texttt{host.ip} & ip & ❌ & 使用客户机注册表中的 \texttt{ip} & 主机 IP \\
\texttt{agent.type} & keyword & ✅ & 固定值:\texttt{falco} / \texttt{filebeat} / \texttt{suricata} & 传感器类型标识 \\
\texttt{agent.ephemeral\_id} & keyword & ❌ & 保持原值 & 传感器实例 ID \\
\texttt{alert.severity} & long & ❌ & 范围映射:1-21 低 / 22-59 中 / 60-100 高 & 告警严重级别 \\
\texttt{alert.status} & keyword & ❌ & 枚举值:\texttt{active} / \texttt{resolved} / \texttt{suppressed} & 告警状态 \\
\texttt{threat.framework} & keyword & ❌ & 固定值:\texttt{MITRE\ ATT\&CK} & 威胁框架标识 \\
\texttt{threat.tactic.id} & keyword & ❌ & 格式:\texttt{TAxxxx},如 \texttt{TA0001} & 战术 ID \\
\texttt{threat.technique.id} & keyword & ❌ & 格式:\texttt{Txxxx},如 \texttt{T1059} & 技术 ID \\
\texttt{rule.name} & keyword & ❌ & 保持原值 & 规则名称 \\
\texttt{rule.category} & keyword & ❌ & 映射到 MITRE 战术,如 \texttt{Execution} / \texttt{Persistence} & 规则分类 \\
\texttt{tags} & keyword & ❌ & 数组格式,自动去重 & 标签数组,用于快速过滤 \\
\end{longtable}
}

\textbf{字段处理注意事项}:

\begin{enumerate}
\def\labelenumi{\arabic{enumi}.}
\tightlist
\item
  \textbf{扁平键兼容}:保留 \texttt{\_source} 中的扁平字段(如 \texttt{proc\_name}),但查询时优先使用嵌套 ECS 字段;
\item
  \textbf{类型校验}:写入前校验字段类型,类型不匹配时记录警告并使用默认值或丢弃;
\item
  \textbf{缺失字段}:必需字段缺失时拒绝入库,非必需字段缺失时使用默认值或留空;
\item
  \textbf{数组字段}:\texttt{tags}、\texttt{threat.tactic.id} 等数组字段须去重,避免重复标签;
\item
  \textbf{IP 地址}:支持 IPv4 和 IPv6,自动识别并设置正确的 \texttt{ip} 类型。
\end{enumerate}

\subsection{幂等与去重(必须满足)}\label{center-62-idempotency}

\begin{enumerate}
\def\labelenumi{\arabic{enumi}.}
\tightlist
\item
  每条文档必须具备 \texttt{event.id};
\item
  中心机写入须按 \texttt{event.id} 幂等:同一 \texttt{event.id} 重复写入不得产生重复文档;
\item
  对于无法保证 \texttt{event.id} 稳定的上游输入,须在进入中心机前补齐稳定 \texttt{event.id}。
\end{enumerate}

\subsection{检测与融合}\label{center-62-detection}

OpenSearch 的检测触发、Raw Finding 生成与 Canonical Finding 融合去重规则详见独立设计章节:

\begin{itemize}
\tightlist
\item
  \texttt{63-检测与告警融合.md}。
\end{itemize}

\subsection{保留策略与脚本}\label{center-62-retention}

\subsection{数据保留周期(固定)}\label{center-62-retention-period}

数据保留周期的权威定义见:\texttt{../../80-规范/80-数据对象与生命周期.md}。

\subsection{初始化与运维脚本(入口)}\label{center-62-scripts}

OpenSearch 侧的初始化流程包括:

\begin{enumerate}
\def\labelenumi{\arabic{enumi})}
\tightlist
\item
  启动 OpenSearch 容器;
\item
  启动中心机后端,后端在启动阶段调用 \texttt{initialize\_indices()} 创建或验证索引存在性(含 mapping);
\item
  配置 Security Analytics detector 并导入 Sigma 规则。
\end{enumerate}

脚本入口文件如下:

\begin{itemize}
\tightlist
\item
  Sigma 规则导入:\texttt{backend/app/services/opensearch/scripts/import\_sigma\_rules.py};
\item
  Security Analytics detector 配置:\texttt{backend/app/services/opensearch/scripts/setup\_security\_analytics.py}。
\end{itemize}
