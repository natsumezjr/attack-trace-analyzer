\subsection{Suricata 采集与 ECS 转换}\label{22-suricata-collection-ecs-conversion}

\subsection{采集输入}\label{client-52-1-collection-input}

\subsection{Suricata 引擎}\label{11-suricata-engine}

Suricata 引擎运行于容器 \texttt{suricata} 中,通过以下脚本启动:

\begin{itemize}
\tightlist
\item
  \texttt{client/sensor/suricata/engine/run-suricata.sh}
\end{itemize}

引擎将检测结果输出为 EVE 日志文件,供导出器读取处理:

\begin{itemize}
\tightlist
\item
  默认路径:\texttt{/data/eve.json}
\end{itemize}

\subsection{关键运行参数}\label{12-key-runtime-parameters}

Suricata 的运行行为由环境变量控制:

\begin{itemize}
\tightlist
\item
  \texttt{SURICATA\_MODE}:运行模式(\texttt{live} 或 \texttt{pcap})
\item
  \texttt{SURICATA\_INTERFACE}:抓包网卡名称
\item
  \texttt{SURICATA\_HOME\_NET}:HOME\_NET 地址组
\end{itemize}

参数取值与运行逻辑在 \texttt{client/sensor/suricata/engine/run-suricata.sh} 中实现。

\subsection{EVE JSON 示例}\label{13-eve-json-example}

以下是一个 DNS 查询事件的 EVE JSON 格式示例:

\begin{verbatim}
{
  "timestamp": "2026-01-14T12:00:00.000Z",
  "event_type": "dns",
  "src_ip": "10.0.0.1",
  "src_port": 12345,
  "dest_ip": "8.8.8.8",
  "dest_port": 53,
  "proto": "UDP",
  "dns": {
    "type": "QUERY",
    "id": 12345,
    "query": "example.com",
    "query_type": "A",
    "query_class": "IN",
    "answers": [
      {"rrname": "example.com", "rrtype": "A", "rdata": "93.184.216.34"}
    ]
  }
}
\end{verbatim}

\subsection{转换规则}\label{2-conversion-rules}

\subsection{导出器位置}\label{21-exporter-location}

Suricata EVE 导出器位于:

\begin{itemize}
\tightlist
\item
  \texttt{client/sensor/suricata/exporter/app.py}
\end{itemize}

导出器持续监听 EVE 文件的新增内容,将其解析为 ECS 格式并发布至 RabbitMQ 队列。

\subsection{输出字段形态}\label{22-output-field-format}

Suricata 导出器采用\textbf{点号扁平键形态}输出(例如 \texttt{event.dataset}、\texttt{source.ip}),中心机在入库前将其转换为嵌套对象形态。

\subsection{主机身份(跨传感器关联)}\label{221-host-identity-cross-sensor}

为实现 Suricata 网络遥测数据与 Falco/Filebeat 主机行为/日志在中心机的 \texttt{Host} 节点汇聚,导出器遵循以下规则:

\begin{itemize}
\tightlist
\item
  \texttt{host.name}:默认取自环境变量 \texttt{HOST\_NAME}(见 \texttt{../../80-规范/89-环境变量与配置规范.md})
\item
  \texttt{host.id}:优先使用环境变量 \texttt{HOST\_ID};缺失时回退为 \texttt{h-} + sha1(host.name){[}:16{]}(见 \texttt{../../80-规范/81-ECS字段规范.md})
\end{itemize}

\subsection{dataset 取值范围}\label{23-dataset-value-range}

导出器依据 \texttt{event\_type} 映射 dataset,取值限定为以下类型:

\begin{itemize}
\tightlist
\item
  \texttt{netflow.flow}
\item
  \texttt{netflow.dns}
\item
  \texttt{netflow.http}
\item
  \texttt{netflow.tls}
\item
  \texttt{netflow.icmp}
\end{itemize}

对于 Suricata IDS 告警,导出器以 \texttt{event.kind="alert"} 标记(\texttt{event.dataset="netflow.alert}),中心机入库时将其规范化为 Raw Finding:

\begin{itemize}
\tightlist
\item
  \texttt{finding.raw.suricata}
\end{itemize}

\subsection{event\_type 与 dataset 映射表}\label{24-event_type-ux4e0e-dataset-ux6620ux5c04ux8868}

{\def\LTcaptype{none} % do not increment counter
\footnotesize
\begin{longtable}[]{@{}p{0.20\textwidth}p{0.12\textwidth}p{0.28\textwidth}p{0.30\textwidth}@{}}
\toprule\noalign{}
event\_type & event.dataset & event.kind & 说明 \\
\midrule\noalign{}
\endhead
\bottomrule\noalign{}
\endlastfoot
dns & netflow.dns & event & DNS 查询 Telemetry \\
http & netflow.http & event & HTTP 请求 Telemetry \\
flow & netflow.flow & event & 网络流 Telemetry \\
tls & netflow.tls & event & TLS 握手 Telemetry \\
icmp & netflow.icmp & event & ICMP 流量 Telemetry \\
alert & finding.raw.suricata & alert & IDS 告警(Raw Finding) \\
\end{longtable}
}

\subsection{EVE JSON 到 ECS 转换流程}\label{25-eve-json-ecs-conversion-flow}

\begin{figure}[htbp]
\centering
\includegraphics[width=\textwidth,height=0.5\textheight,keepaspectratio]{figures/client-52/client-52-01.pdf}
\caption{EVE JSON 到 ECS 转换流程}
\label{fig:client-52-01}
\end{figure}

\subsection{网络字段与证据引用}\label{3-network-fields-evidence-reference}

导出器按以下规则填充网络相关字段:

\begin{itemize}
\tightlist
\item
  \texttt{source.ip}、\texttt{source.port}
\item
  \texttt{destination.ip}、\texttt{destination.port}
\item
  \texttt{network.transport}、\texttt{network.protocol}
\item
  \texttt{flow.id}、\texttt{network.community\_id}
\item
  DNS 查询与解析(当 \texttt{event\_type=dns} 时):

  \begin{itemize}
  \tightlist
  \item
    \texttt{dns.question.name}
  \item
    \texttt{dns.answers{[}{]}}(用于构建 \texttt{Domain\ →\ IP} 的 \texttt{RESOLVES\_TO} 边)
  \end{itemize}
\end{itemize}

字段定义详见 \texttt{../../80-规范/81-ECS字段规范.md}。

\subsection{队列投递}\label{client-52-4-queue-delivery}

导出器将数据投递至 RabbitMQ:

\begin{itemize}
\tightlist
\item
  队列:\texttt{data.suricata}
\item
  \texttt{RABBITMQ\_URL} 与 \texttt{RABBITMQ\_QUEUE} 由容器环境变量指定
\end{itemize}

拉取接口在返回前会补齐稳定的 \texttt{event.id},规则见 \texttt{../../80-规范/87-客户机与中心机接口.md}。

\subsection{故障处理}\label{client-52-5-error-handling}

\begin{enumerate}
\def\labelenumi{\arabic{enumi}.}
\tightlist
\item
  EVE 文件不存在时,导出器自动创建目录并等待文件生成。
\item
  RabbitMQ 发布失败时,导出器重新连接并重试发布。
\item
  单条 EVE 行解析失败时,导出器跳过该行并继续处理后续数据。
\end{enumerate}
