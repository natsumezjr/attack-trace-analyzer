\subsection{Neo4j 入图与图查询}\label{center-64-graph-query}

\subsection{模块职责与边界}\label{center-64-module-boundaries}

Neo4j 模块承担``实体关系图(Entity Graph)''的权威存储与图查询职责,具体包括:

\begin{enumerate}
\def\labelenumi{\arabic{enumi}.}
\tightlist
\item
  \textbf{Schema 管理}:创建并维护节点唯一约束与常用索引;
\item
  \textbf{入图写入}:将输入的 ECS 文档转换为节点或边,并写入 Neo4j;
\item
  \textbf{时间窗查询}:支持按时间窗查询边集合,供图可视化与算法使用;
\item
  \textbf{图算法查询}:支持基于时间窗投影图的最短路计算(Neo4j GDS);
\item
  \textbf{结果承载}:存储溯源任务写回的边属性,支持``按节点查询溯源结果''。
\end{enumerate}

本模块不负责:

\begin{itemize}
\tightlist
\item
  OpenSearch 的检测与融合(见 \texttt{63-检测与告警融合.md});
\item
  溯源算法的具体执行(见 \texttt{../../50-详细设计/分析/});
\item
  ECS 字段口径(见 \texttt{../../80-规范/81-ECS字段规范.md})。
\end{itemize}

\subsection{Schema 与约束}\label{center-64-schema}

\subsection{节点唯一约束(必须存在)}\label{center-64-constraints}

节点类型与唯一键定义遵循 \texttt{../../80-规范/84-Neo4j实体图谱规范.md}。Neo4j 必须建立以下唯一约束(表达为``Label + 属性键''):

{\def\LTcaptype{none} % do not increment counter
\footnotesize
\begin{longtable}[]{@{}p{0.30\textwidth}p{0.60\textwidth}@{}}
\toprule\noalign{}
Label & 唯一键 \\
\midrule\noalign{}
\endhead
\bottomrule\noalign{}
\endlastfoot
{\scriptsize\texttt{Host}} & {\scriptsize\texttt{host.id}} \\
{\scriptsize\texttt{User}} & {\scriptsize\texttt{user.id}} \\
{\scriptsize\texttt{User}} & {\scriptsize\texttt{host.id + user.name}} \\
{\scriptsize\texttt{Process}} & {\scriptsize\texttt{process.entity\_id}} \\
{\scriptsize\texttt{File}} & {\scriptsize\texttt{host.id + file.path}} \\
{\scriptsize\texttt{Domain}} & {\scriptsize\texttt{domain.name}} \\
{\scriptsize\texttt{IP}} & {\scriptsize\texttt{ip}} \\
\end{longtable}
}

\begin{quote}
说明:\texttt{User} 与 \texttt{File} 的复合键用于避免跨主机误合并。当事件包含 \texttt{user.id} 时,使用 \texttt{user.id} 作为唯一键;否则使用 \texttt{host.id\ +\ user.name} 作为唯一键。
\end{quote}

\subsection{图数据模型}\label{center-64-data-model}

\begin{figure}[htbp]
\centering
\includegraphics[width=\textwidth,height=0.5\textheight,keepaspectratio]{figures/center-64/center-64-01.pdf}
\caption{图数据模型}
\label{fig:center-64-01}
\end{figure}

\subsection{索引(必须存在)}\label{center-64-indexes}

为支撑展示与排障需求,Neo4j 必须为以下属性建立索引:

\begin{itemize}
\tightlist
\item
  \texttt{Host.host.name}
\item
  \texttt{User.user.name}
\item
  \texttt{Process.process.executable}
\item
  \texttt{File.file.path}
\item
  \texttt{Domain.domain.name}
\item
  \texttt{IP.ip}
\end{itemize}

\subsection{写入:ECS → Graph}\label{center-64-write-ecs}

\subsection{入图输入范围(严格)}\label{center-64-input-scope}

Neo4j 入图仅接受两类 ECS 文档:

\begin{enumerate}
\def\labelenumi{\arabic{enumi}.}
\tightlist
\item
  Telemetry:\texttt{event.kind="event"}
\item
  Canonical Findings:\texttt{event.kind="alert"} 且 \texttt{event.dataset="finding.canonical"}
\end{enumerate}

Raw Findings(包括传感器原始告警与 Security Analytics 原始 finding)不得直接入图。

\subsection{入图边属性(必须写入)}\label{center-64-edge-props}

每条入图边必须写入以下属性(字段名与来源遵循 \texttt{../../80-规范/81-ECS字段规范.md} 与 \texttt{../../80-规范/84-Neo4j实体图谱规范.md}):

\begin{itemize}
\tightlist
\item
  \texttt{ts} 或 \texttt{@timestamp}:边的事件时间(字符串时间戳)
\item
  \texttt{ts\_float}:数值时间戳(秒,float),用于时间窗过滤与 GDS 投影
\item
  \texttt{custom.evidence.event\_ids{[}{]}}:证据事件引用列表
\item
  \texttt{event.kind} / \texttt{event.dataset} / \texttt{event.id}:用于回溯与区分来源
\end{itemize}

当边来自 Canonical Finding 时,边必须额外写入:

\begin{itemize}
\tightlist
\item
  \texttt{is\_alarm=true}
\item
  \texttt{rule.*}、\texttt{threat.*}、\texttt{event.severity}、\texttt{custom.finding.*} 等字段(用于解释与可视化)
\end{itemize}

\subsection{写入幂等边界}\label{center-64-idempotency}

\begin{itemize}
\tightlist
\item
  节点写入必须幂等(MERGE),通过唯一键去重;
\item
  边写入采用``按证据追加''语义:允许存在多条相同类型的关系,但每条边必须携带其证据 \texttt{event.id} 与证据列表,便于后续去重或回放。
\end{itemize}

\begin{quote}
边去重属于 Analysis 的工作范围,在``展示层''与``任务结果写回层''通过属性过滤实现干净展示。
\end{quote}

\subsection{批量写入优化(v2.1+)}\label{center-64-batch-write}

自 v2.1 起,Neo4j 模块支持批量写入 API 以提升入图性能:

\textbf{批量 API}:

\begin{itemize}
\tightlist
\item
  \texttt{add\_nodes\_and\_edges(nodes,\ edges)} —— 在单个事务中批量写入节点和边
\end{itemize}

\textbf{性能优化策略}:

\begin{itemize}
\tightlist
\item
  \textbf{节点批量写入}:按节点类型分组,使用 UNWIND 批量 MERGE
\item
  \textbf{边批量写入}:在单个事务中逐个 MERGE(需匹配起终点)
\item
  \textbf{性能提升}:处理 1000 条事件的网络往返从 \textasciitilde16000 次降至 \textasciitilde160 次(提升 100 倍)
\end{itemize}

\textbf{使用示例}:

\begin{verbatim}
from app.services.neo4j import db

## 批量写入
nodes = [host_node(host_id="h-001"), user_node(user_id="u-001")]
edges = [logon_edge(user, host)]
db.add_nodes_and_edges(nodes, edges)
\end{verbatim}

\textbf{向后兼容性}:

\begin{itemize}
\tightlist
\item
  保留 \texttt{add\_node()} 和 \texttt{add\_edge()} 单条 API,确保旧代码无需修改
\item
  批量 API 与单条 API 具有相同的幂等性保证
\end{itemize}

\textbf{实施位置}:

\begin{itemize}
\tightlist
\item
  代码:\texttt{backend/app/services/neo4j/db.py:200-400}
\item
  测试:\texttt{backend/tests/unit/test\_services\_neo4j/test\_db\_batch.py}
\end{itemize}

\begin{figure}[htbp]
\centering
\includegraphics[width=\textwidth,height=0.5\textheight,keepaspectratio]{figures/center-64/center-64-02.pdf}
\caption{批量写入优化}
\label{fig:center-64-02}
\end{figure}

\subsection{查询:可视化与算法的图查询}\label{center-64-query}

\subsection{图查询能力清单(必须支持)}\label{center-64-query-capabilities}

Neo4j 模块必须提供以下查询能力:

\begin{enumerate}
\def\labelenumi{\arabic{enumi}.}
\tightlist
\item
  \textbf{告警边查询}:返回所有 \texttt{is\_alarm=true} 的边集合;
\item
  \textbf{时间窗边查询}:给定 \texttt{{[}t\_min,\ t\_max{]}}(秒),返回时间窗内的边集合,并支持按关系类型过滤;
\item
  \textbf{时间窗最短路}:给定 \texttt{src\_uid}、\texttt{dst\_uid}、\texttt{{[}t\_min,\ t\_max{]}} 与风险权重表,返回时间窗内的加权最短路边序列。
\end{enumerate}

\subsection{后端对外 API 绑定(固定)}\label{center-64-api-binding}

后端对外提供统一的图查询接口:

\begin{itemize}
\tightlist
\item
  \texttt{POST\ /api/v1/graph/query}
\end{itemize}

该接口支持以下动作:

\begin{itemize}
\tightlist
\item
  \texttt{alarm\_edges}
\item
  \texttt{edges\_in\_window}
\item
  \texttt{shortest\_path\_in\_window}
\item
  \texttt{analysis\_edges\_by\_task}
\end{itemize}

接口的请求/响应字段由后端实现定义,Neo4j 模块负责提供稳定的查询语义与返回结构(包括 nodes/edges 的 uid、rtype、props)。

其中:

\begin{itemize}
\tightlist
\item
  \texttt{analysis\_edges\_by\_task}:按 \texttt{analysis.task\_id} 拉取该任务写回的边集合。当请求参数 \texttt{only\_path=true} 时,仅返回 \texttt{analysis.is\_path\_edge=true} 的关键路径边;当 \texttt{only\_path=false} 时,返回该任务写回的全部边。
\end{itemize}

\begin{figure}[htbp]
\centering
\includegraphics[width=\textwidth,height=0.5\textheight,keepaspectratio]{figures/center-64/center-64-03.pdf}
\caption{Neo4j 查询场景}
\label{fig:center-64-03}
\end{figure}

\subsection{结果写回:边属性规范}\label{center-64-writeback}

溯源结果写回属于``图谱回标与边属性''模块,其详细设计参见:

\begin{itemize}
\tightlist
\item
  写回数据结构(权威规范):\texttt{../../80-规范/85-溯源结果写回规范.md}
\item
  工程实现与读取规范:\texttt{65-图谱回标与边属性.md}
\end{itemize}
