\section{客户机模块}\label{client-module}

\subsection{客户机职责边界}\label{client-responsibility-boundary}

客户机承担以下核心职责:

\begin{enumerate}
\def\labelenumi{\arabic{enumi}.}
\tightlist
\item
  采集三类数据源:Falco、Filebeat、Suricata;
\item
  将各类输出转换为 ECS 子集字段,生成可供中心机入库的 JSON 事件;
\item
  将事件写入 RabbitMQ 队列实现本地缓冲;
\item
  暴露拉取接口,支持中心机轮询获取增量数据。
\end{enumerate}

以下功能不在客户机职责范围内:

\begin{itemize}
\tightlist
\item
  OpenSearch 存储与查询;
\item
  Neo4j 图谱建模与查询;
\item
  溯源算法执行与结果写回。
\end{itemize}

\subsection{组件清单与目录结构}\label{component-list-directory}

客户机包含以下容器(配置文件:\texttt{client/docker-compose.yml}):

{\def\LTcaptype{none} % do not increment counter
\footnotesize
\begin{longtable}[]{@{}p{0.25\textwidth}p{0.12\textwidth}p{0.53\textwidth}@{}}
\toprule\noalign{}
容器 & 作用 & 主要代码位置 \\
\midrule\noalign{}
\endhead
\bottomrule\noalign{}
\endlastfoot
\texttt{rabbitmq} & 本地缓冲区 & \texttt{client/docker-compose.yml} \\
\texttt{falco} & 主机行为采集 & \texttt{client/docker-compose.yml} \\
\texttt{falco-ecs} & Falco ECS 转换并投递队列 & \texttt{client/sensor/falco/ecs-converter/} \\
\texttt{filebeat} & 主机日志采集 + Sigma 检测并投递队列 & \texttt{client/sensor/filebeat/} \\
\texttt{suricata} & 网络 IDS 与 EVE 输出 & \texttt{client/sensor/suricata/engine/} \\
\texttt{suricata-exporter} & EVE ECS 转换并投递队列 & \texttt{client/sensor/suricata/exporter/} \\
\texttt{backend} & 拉取 API(Gin) & \texttt{client/backend/} \\
\end{longtable}
}

\subsection{端到端数据流}\label{end-to-end-data-flow}

\subsubsection{数据流图(完整视图)}\label{data-flow-diagram}

\begin{figure}[htbp]
\centering
\includegraphics[width=\textwidth,height=0.5\textheight,keepaspectratio]{figures/client-50/client-50-01.pdf}
\caption{客户机数据流图}
\label{fig:client-50-01}
\end{figure}

\subsubsection{三传感器数据流对比}\label{three-sensor-data-flow}

{\def\LTcaptype{none} % do not increment counter
\footnotesize
\begin{longtable}[]{@{}llllll@{}}
\toprule\noalign{}
传感器 & 采集方式 & 输出格式 & 转换器 & 队列名 & 特点 \\
\midrule\noalign{}
\endhead
\bottomrule\noalign{}
\endlastfoot
\textbf{Falco} & 内核钩子 & JSONL & \texttt{falco\_json\_to\_ecs.py} & \texttt{data.falco} & 进程/文件/网络细粒度行为 \\
\textbf{Filebeat} & 日志文件 & ECS JSON & \texttt{detector.py} & \texttt{data.filebeat} & 认证日志 + Sigma 规则告警 \\
\textbf{Suricata} & 网络抓包 & EVE JSON & \texttt{exporter/app.py} & \texttt{data.suricata} & DNS/HTTP/Flow/IDS 告警 \\
\end{longtable}
}

\subsubsection{关键设计约束}\label{key-design-constraints}

\begin{enumerate}
\def\labelenumi{\arabic{enumi}.}
\tightlist
\item
  \textbf{队列增量语义}: 消息被 \texttt{ack} 后不再返回,确保中心机拉取的增量性
\item
  \textbf{event.id 稳定性}: 由 backend 拉取接口统一补齐,避免上游缺失
\item
  \textbf{host.id 一致性}: 三传感器统一使用 \texttt{HOST\_ID} 环境变量,或通过 \texttt{host.name} hash 生成
\item
  \textbf{ECS 归一化}: 三种不同格式统一为 ECS 子集,便于中心机入库
\end{enumerate}

\subsubsection{三传感器数据量规模对比}\label{three-sensor-data-volume}

{\def\LTcaptype{none} % do not increment counter
\footnotesize
\begin{longtable}[]{@{}llllll@{}}
\toprule\noalign{}
数据源 & 事件速率 & 单事件大小 & 每分钟数据量 & 主要字段 & 用途 \\
\midrule\noalign{}
\endhead
\bottomrule\noalign{}
\endlastfoot
\textbf{Falco} & \textasciitilde100-500 events/min & \textasciitilde2KB & \textasciitilde0.2-1 MB & process、file、network & 进程树、文件访问链路 \\
\textbf{Filebeat} & \textasciitilde10-50 lines/min & \textasciitilde0.5KB & \textasciitilde0.005-0.025 MB & user、source、event.outcome & 认证链路、登录失败 \\
\textbf{Suricata} & 取决于网络流量 & \textasciitilde1-3KB & 变化大(典型:0.1-10 MB) & source/destination、dns、http & DNS解析、网络连接、IDS告警 \\
\end{longtable}
}

\textbf{说明}:

\begin{itemize}
\tightlist
\item
  Falco 事件速率取决于系统活跃度,正常运行时约 100-500 events/min;
\item
  Filebeat 仅处理认证相关日志,速率较低;
\item
  Suricata 数据量与网络流量成正比,演示环境建议控制在 \textless{} 1Gbps。
\end{itemize}

中心机轮询逻辑参考:

\begin{itemize}
\tightlist
\item
  \texttt{backend/app/services/client\_poller.py}
\end{itemize}

\subsection{分模块详细设计索引}\label{module-index}

客户机详细设计按模块拆分如下:

\begin{itemize}
\tightlist
\item
  Falco:\texttt{51-Falco采集与ECS转换.md}
\item
  Suricata:\texttt{52-Suricata采集与ECS转换.md}
\item
  Filebeat:\texttt{53-Filebeat采集与ECS转换.md}
\item
  RabbitMQ:\texttt{54-RabbitMQ缓冲与队列语义.md}
\item
  拉取接口:\texttt{55-拉取接口.md}
\end{itemize}

\subsection{与中心机的交互边界}\label{interaction-boundary}

\begin{enumerate}
\def\labelenumi{\arabic{enumi}.}
\tightlist
\item
  客户机仅暴露拉取接口(HTTP GET),接口规范由 \texttt{../../80-规范/87-客户机与中心机接口.md} 定义;
\item
  客户机不保存中心机状态,不维护游标,增量语义由 RabbitMQ 队列保证;
\item
  客户机返回数据前需确保事件包含稳定的 \texttt{event.id},补齐逻辑在拉取接口中实现。
\end{enumerate}

\subsection{运维与排障入口}\label{maintenance-troubleshooting}

客户机部署、启动、验证、重置与排障详见运维文档:

\begin{itemize}
\tightlist
\item
  编译安装与使用:\texttt{../../90-运维与靶场/90-编译安装与使用.md}
\item
  靶场部署:\texttt{../../90-运维与靶场/91-靶场部署.md}
\item
  一键编排:\texttt{../../90-运维与靶场/92-一键编排.md}
\item
  验证清单:\texttt{../../90-运维与靶场/94-验证清单.md}
\item
  重置复现与排障:\texttt{../../90-运维与靶场/95-重置复现与排障.md}
\end{itemize}
