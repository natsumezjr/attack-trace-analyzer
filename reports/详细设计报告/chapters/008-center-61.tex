\subsection{注册与轮询}\label{31-registration-polling}

\subsection{注册表结构}\label{registry-structure}

中心机使用 OpenSearch 索引 \texttt{client-registry} 作为客户机注册表的唯一权威数据源。该索引存储所有已注册客户机的核心信息,包括标识、接口地址、能力声明以及轮询状态。

注册表文档包含以下字段:

\begin{itemize}
\tightlist
\item
  \texttt{client.id}:客户机唯一标识符
\item
  \texttt{client.listen\_url}:客户机拉取接口基地址(如 \texttt{http://10.0.0.11:8888})
\item
  \texttt{client.capabilities}:数据源能力声明,支持三类(falco、suricata、filebeat)
\item
  \texttt{poll.last\_seen}:最近一次轮询时间戳
\item
  \texttt{poll.status}:最近一次轮询状态标记
\item
  \texttt{poll.last\_error}:最近一次轮询失败的错误信息
\end{itemize}

字段定义与索引约束详见:
\begin{itemize}
\tightlist
\item
  \texttt{../../80-规范/82-OpenSearch索引与Mapping规范.md}
\item
  \texttt{../../80-规范/87-客户机与中心机接口.md}
\end{itemize}

\subsection{注册流程}\label{registration-flow}

中心机通过 HTTP 接口接收客户机注册请求。注册成功后,客户机信息必须持久化到 \texttt{client-registry} 索引中。

\textbf{注册接口}:

\begin{itemize}
\tightlist
\item
  \texttt{POST /api/v1/clients/register}
\end{itemize}

接口的请求格式、响应结构以及错误码定义详见:
\begin{itemize}
\tightlist
\item
  \texttt{../../80-规范/87-客户机与中心机接口.md}
\end{itemize}

\textbf{容错要求}:注册失败时严禁降级为"仅内存登记",必须向调用方返回明确的错误信息,同时保持注册表数据的一致性。

\subsection{轮询调度}\label{polling-scheduling}

\subsection{实现位置}\label{implementation-location}

轮询服务的核心实现位于:

\begin{itemize}
\tightlist
\item
  \texttt{backend/app/services/client\_poller.py}
\end{itemize}

\subsection{轮询周期与超时}\label{polling-timeout}

轮询周期与超时参数通过环境变量配置,代码中内置默认值:

\begin{itemize}
\tightlist
\item
  \texttt{CENTER\_POLL\_INTERVAL\_SECONDS}:轮询间隔(默认 \texttt{5} 秒)
\item
  \texttt{CENTER\_POLL\_TIMEOUT\_SECONDS}:单次请求超时(默认 \texttt{5} 秒)
\end{itemize}

完整的环境变量清单与默认值详见:
\begin{itemize}
\tightlist
\item
  \texttt{../../80-规范/89-环境变量与配置规范.md}
\end{itemize}

\subsection{单轮询流程}\label{single-polling-flow}

每个定时器触发周期执行一次完整的轮询流程,如图 \ref{fig:center-61-01} 所示。

\begin{figure}[htbp]
\centering
\includegraphics[width=\textwidth,height=0.5\textheight,keepaspectratio]{figures/center-61/center-61-01.pdf}
\caption{单轮询流程}
\label{fig:center-61-01}
\end{figure}

\textbf{流程说明}:

\begin{enumerate}
\def\labelenumi{\arabic{enumi}.}
\tightlist
\item
  \textbf{注册表读取}:从 OpenSearch 索引 \texttt{client-registry} 读取所有已注册客户机列表;
\item
  \textbf{客户机遍历}:解析每个客户机的 \texttt{client.id}、\texttt{client.listen\_url}、\texttt{client.capabilities} 字段;
\item
  \textbf{路由选择}:根据客户机的 capabilities 确定需要拉取的数据路由,可选路由包括 \texttt{falco}、\texttt{suricata}、\texttt{filebeat};
\item
  \textbf{HTTP 拉取}:向每个客户机的各个路由发送 HTTP GET 请求,地址格式为 \texttt{\{listen\_url\}/\{route\}};
\item
  \textbf{事件提取}:从响应体中提取 \texttt{data[]} 事件数组,汇总为本周期的事件列表;
\item
  \textbf{状态更新}:将轮询结果写回注册表,更新 \texttt{poll.*} 相关字段;
\item
  \textbf{事件存储}(Step 2):调用 \texttt{store\_events()} 将事件列表写入 OpenSearch,根据路由分别存储到 Telemetry、Raw Findings 或 Canonical 索引;
\item
  \textbf{数据分析}(Step 3):调用 \texttt{run\_data\_analysis()} 执行检测算法与告警融合,将原始数据转换为 Canonical 格式并写回 OpenSearch;
\item
  \textbf{图谱入库}(Step 4):调用 Neo4j 入图流程,将 Telemetry 与 Canonical 数据导入图数据库。
\end{enumerate}

\textbf{相关模块}:

\begin{itemize}
\tightlist
\item
  Step 2 事件存储:\texttt{backend/app/services/opensearch/storage.py:store\_events()}
\item
  Step 3 数据分析:\texttt{backend/app/services/opensearch/analysis.py:run\_data\_analysis()}
\item
  Step 4 图谱入库:\texttt{backend/app/services/neo4j/ingest.py:ingest\_from\_opensearch\_ingested\_window()}
\end{itemize}

\subsection{状态更新与错误处理}\label{status-update-error-handling}

\subsection{轮询状态机}\label{polling-state-machine}

轮询服务通过状态机管理执行流程,状态转换逻辑如图 \ref{fig:center-61-02} 所示。

\begin{figure}[htbp]
\centering
\includegraphics[width=\textwidth,height=0.5\textheight,keepaspectratio]{figures/center-61/center-61-02.pdf}
\caption{轮询状态机}
\label{fig:center-61-02}
\end{figure}

\textbf{状态定义}:

\begin{itemize}
\tightlist
\item
  \textbf{Idle}:空闲等待状态,等待定时器触发
\item
  \textbf{Polling}:轮询进行中,正在从注册表读取客户机列表
\item
  \textbf{Fetching}:数据拉取阶段,向客户机接口发起 HTTP 请求
\item
  \textbf{Processing}:数据处理阶段,提取事件并更新状态
\item
  \textbf{Success}:单个客户机轮询成功
\item
  \textbf{Error}:单个客户机轮询失败
\end{itemize}

\subsection{故障处理策略}\label{error-handling-strategy}

轮询服务采用隔离与容错机制,确保单个客户机或数据源故障不影响整体服务可用性。

\textbf{容错原则}:

\begin{enumerate}
\def\labelenumi{\arabic{enumi}.}
\tightlist
\item
  \textbf{数据源隔离}:单个数据源拉取失败不影响其他数据源的拉取;
\item
  \textbf{客户机隔离}:单个客户机轮询失败不影响其他客户机的轮询;
\item
  \textbf{错误记录}:失败信息写入 \texttt{poll.last\_error},状态标记写入 \texttt{poll.status},下一轮次自动重试;
\item
  \textbf{持续运行}:轮询循环不因单个客户机异常而退出;
\item
  \textbf{必选依赖处理}:OpenSearch 与 Neo4j 属于中心机必选依赖。当 Step 3(数据分析)或 Step 4(图谱入库)失败时,\textbf{轮询任务应快速失败并暴露错误}(便于靶场联调与验收时及时发现问题),但\textbf{禁止在后台任务中直接调用 \texttt{os.\_exit(1)} 终止 FastAPI 进程}(避免服务"假死"或不可控退出,同时保留排障与恢复能力)。
\end{enumerate}

\subsection{故障处理时序图}\label{error-handling-sequence-diagram}

完整的超时、重试与错误记录流程如图 \ref{fig:center-61-03} 所示。

\begin{figure}[htbp]
\centering
\includegraphics[width=\textwidth,height=0.5\textheight,keepaspectratio]{figures/center-61/center-61-03.pdf}
\caption{故障处理时序图}
\label{fig:center-61-03}
\end{figure}

\textbf{时序说明}:

\begin{enumerate}
\def\labelenumi{\arabic{enumi}.}
\tightlist
\item
  \textbf{正常流程}:定时器触发轮询服务 → 拉取所有客户机的三类数据源 → 更新注册表状态 → 执行数据处理流水线(存储/分析/入图);
\item
  \textbf{超时处理}:单个数据源超时(如 falco)不阻塞其他数据源(suricata 与 filebeat 继续拉取)→ 错误信息记录到 \texttt{poll.last\_error} → 状态标记为 \texttt{partial\_success} → 继续处理已成功拉取的事件;
\item
  \textbf{流水线故障}:当 OpenSearch 或 Neo4j 等必选依赖失败时 → 抛出异常终止本轮轮询任务 → 记录完整错误堆栈 → 保持 FastAPI 进程运行(便于排障与恢复)。
\end{enumerate}
