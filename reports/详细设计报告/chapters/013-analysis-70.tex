\section{分析模块}\label{analysis-module}

\subsection{触发方式}\label{analysis-trigger}

溯源任务通过前端点选节点触发,执行流程如下:

\begin{enumerate}
\def\labelenumi{\arabic{enumi}.}
\tightlist
\item
  前端在图中选择目标节点;
\item
  调用中心机创建任务接口;
\item
  中心机立即返回 \texttt{task\_id};
\item
  中心机后台执行任务并持续更新状态;
\item
  任务完成后,前端通过图查询接口读取结果。
\end{enumerate}

\paragraph{任务执行时序图}\label{analysis-task-timing}

\begin{figure}[htbp]
\centering
\includegraphics[width=\textwidth,height=0.5\textheight,keepaspectratio]{figures/analysis-70/analysis-70-01.pdf}
\caption{任务执行时序图}
\label{fig:analysis-70-01}
\end{figure}

\subsection{task\_id 规则}\label{analysis-task-id}

\texttt{task\_id} 格式为:

\begin{verbatim}
trace-<uuid_v4>
\end{verbatim}

\textbf{示例}:

\begin{verbatim}
{
  "task_id": "trace-550e8400-e29b-41d4-a716-446655440000"
}
\end{verbatim}

实现位置:

\begin{itemize}
\tightlist
\item
  \texttt{backend/app/services/analyze/pipeline.py} 的 \texttt{new\_task\_id()}
\end{itemize}

\subsection{任务状态存储}\label{analysis-task-storage}

任务状态存储于 OpenSearch 的按日滚动索引:

\begin{verbatim}
analysis-tasks-YYYY-MM-DD
\end{verbatim}

\textbf{示例}:

\begin{verbatim}
analysis-tasks-2026-01-16
analysis-tasks-2026-01-17
\end{verbatim}

索引创建与 mapping 定义位置:

\begin{itemize}
\tightlist
\item
  \texttt{backend/app/services/opensearch/mappings.py} 的 \texttt{analysis\_tasks\_mapping}
\end{itemize}

\subsection{任务文档字段}\label{analysis-task-fields}

\subsubsection{任务文档结构图}\label{analysis-task-structure}

\begin{figure}[htbp]
\centering
\includegraphics[width=\textwidth,height=0.5\textheight,keepaspectratio]{figures/analysis-70/analysis-70-02.pdf}
\caption{任务文档结构图}
\label{fig:analysis-70-02}
\end{figure}

\subsubsection{字段定义表}\label{analysis-field-definitions}

任务文档字段定义如下:

{\def\LTcaptype{none} % do not increment counter
\footnotesize
\begin{longtable}[]{@{}p{0.28\textwidth}p{0.12\textwidth}p{0.28\textwidth}p{0.22\textwidth}@{}}
\toprule\noalign{}
字段路径 & 类型 & 说明 & 示例值 \\
\midrule\noalign{}
\endhead
\bottomrule\noalign{}
\endlastfoot
{\scriptsize\texttt{@timestamp}} & string & 任务创建时间(RFC3339) & {\scriptsize\texttt{2026-01-16T10:30:00Z}} \\
{\scriptsize\texttt{task.id}} & string & 任务唯一标识 & {\scriptsize\texttt{trace-550e8400-...}} \\
{\scriptsize\texttt{task.status}} & string & 任务状态 & {\scriptsize\texttt{queued} / \texttt{running} / \texttt{succeeded} / \texttt{failed}} \\
{\scriptsize\texttt{task.progress}} & integer & 任务进度(0-100) & {\scriptsize\texttt{85}} \\
{\scriptsize\texttt{task.target.node\_uid}} & string & 目标节点 UID & {\scriptsize\texttt{node-123}} \\
{\scriptsize\texttt{task.window.start\_ts}} & string & 分析时间窗起点 & {\scriptsize\texttt{2026-01-16T00:00:00Z}} \\
{\scriptsize\texttt{task.window.end\_ts}} & string & 分析时间窗终点 & {\scriptsize\texttt{2026-01-16T23:59:59Z}} \\
{\scriptsize\texttt{task.started\_at}} & string & 任务开始时间 & {\scriptsize\texttt{2026-01-16T10:30:05Z}} \\
{\scriptsize\texttt{task.finished\_at}} & string & 任务结束时间 & {\scriptsize\texttt{2026-01-16T10:35:10Z}} \\
{\scriptsize\texttt{task.error}} & string & 失败原因 & {\scriptsize\texttt{Neo4j connection timeout}} \\
{\scriptsize\texttt{task.result.summary}} & string & 任务结果摘要 & {\scriptsize\texttt{Found 3 attack paths}} \\
\end{longtable}
}

任务级结构化结果字段定义如下:

{\def\LTcaptype{none} % do not increment counter
\footnotesize
\begin{longtable}[]{@{}p{0.45\textwidth}p{0.12\textwidth}p{0.33\textwidth}@{}}
\toprule\noalign{}
字段路径 & 类型 & 说明 \\
\midrule\noalign{}
\endhead
\bottomrule\noalign{}
\endlastfoot
{\scriptsize\texttt{task.result.ttp\_similarity.attack\_tactics}} & array & 匹配的攻击战术列表 \\
{\scriptsize\texttt{task.result.ttp\_similarity.attack\_techniques}} & array & 匹配的攻击技术列表 \\
{\scriptsize\texttt{task.result.ttp\_similarity.similar\_apts}} & array & 相似 APT 组织列表 \\
{\scriptsize\texttt{task.result.trace.updated\_edges}} & array & 更新的边属性集合 \\
{\scriptsize\texttt{task.result.trace.path\_edges}} & array & 溯源路径边集合 \\
\end{longtable}
}

\subsubsection{任务文档示例}\label{analysis-task-example}

\begin{verbatim}
{
  "@timestamp": "2026-01-16T10:30:00Z",
  "task": {
    "id": "trace-550e8400-e29b-41d4-a716-446655440000",
    "status": "succeeded",
    "progress": 100,
    "target": {
      "node_uid": "node-malicious-ssh-123"
    },
    "window": {
      "start_ts": "2026-01-16T00:00:00Z",
      "end_ts": "2026-01-16T23:59:59Z"
    },
    "started_at": "2026-01-16T10:30:05Z",
    "finished_at": "2026-01-16T10:35:10Z",
    "error": null,
    "result": {
      "summary": "成功识别3条攻击路径,匹配2个APT组织",
      "ttp_similarity": {
        "attack_tactics": ["TA0001", "TA0002"],
        "attack_techniques": ["T1190", "T1021"],
        "similar_apts": ["APT28", "APT29"]
      },
      "trace": {
        "updated_edges": 15,
        "path_edges": ["edge-1", "edge-2", "edge-3"]
      }
    }
  }
}
\end{verbatim}

以上字段由任务执行流水线写入,实现位置:

\begin{itemize}
\tightlist
\item
  \texttt{backend/app/services/analyze/pipeline.py} 的 \texttt{run\_analysis\_task()}
\end{itemize}

\subsection{状态机与进度更新}\label{analysis-state-machine}

\subsubsection{状态机}\label{analysis-states}

状态转移规则定义如下:

\begin{figure}[htbp]
\centering
\includegraphics[width=\textwidth,height=0.5\textheight,keepaspectratio]{figures/analysis-70/analysis-70-03.pdf}
\caption{状态机}
\label{fig:analysis-70-03}
\end{figure}

\textbf{状态转移规则}:

{\def\LTcaptype{none} % do not increment counter
\begin{longtable}[]{@{}lll@{}}
\toprule\noalign{}
当前状态 & 可转移状态 & 触发条件 \\
\midrule\noalign{}
\endhead
\bottomrule\noalign{}
\endlastfoot
\texttt{queued} & \texttt{running} & 任务执行器开始处理 \\
\texttt{running} & \texttt{succeeded} & 分析完成且无错误 \\
\texttt{running} & \texttt{failed} & 分析过程出现异常 \\
\end{longtable}
}

\textbf{禁止操作}:

\begin{itemize}
\tightlist
\item
  ❌ \texttt{succeeded} → \texttt{running}(已完成任务不可重入)
\item
  ❌ \texttt{failed} → \texttt{running}(失败任务需重新创建)
\item
  ❌ \texttt{queued} → \texttt{succeeded}(跳过执行直接完成)
\item
  ❌ 任何状态的回退操作
\end{itemize}

\subsubsection{进度更新规则}\label{analysis-progress}

任务进度由中心机流水线按以下规则更新:

{\def\LTcaptype{none} % do not increment counter
\begin{longtable}[]{@{}llll@{}}
\toprule\noalign{}
进度值 & 状态 & 阶段描述 & 更新时机 \\
\midrule\noalign{}
\endhead
\bottomrule\noalign{}
\endlastfoot
0 & \texttt{queued} & 任务已创建 & 写入任务文档时 \\
5 & \texttt{running} & 开始执行 & 启动任务执行器时 \\
20 & \texttt{running} & 准备并行计算 & 数据加载完成后 \\
70 & \texttt{running} & 写入相似度结果 & TTP 分析完成时 \\
95 & \texttt{running} & 写回图边属性 & Neo4j 更新完成时 \\
100 & \texttt{succeeded} & 任务完成 & 所有步骤成功时 \\
\end{longtable}
}

\textbf{进度更新示例}:

\begin{verbatim}
// 阶段 1: 创建任务
{
  "task": {
    "status": "queued",
    "progress": 0,
    "started_at": null,
    "finished_at": null
  }
}

// 阶段 2: 开始执行
{
  "task": {
    "status": "running",
    "progress": 5,
    "started_at": "2026-01-16T10:30:05Z",
    "finished_at": null
  }
}

// 阶段 3: 计算中
{
  "task": {
    "status": "running",
    "progress": 70,
    "started_at": "2026-01-16T10:30:05Z",
    "finished_at": null
  }
}

// 阶段 4: 完成
{
  "task": {
    "status": "succeeded",
    "progress": 100,
    "started_at": "2026-01-16T10:30:05Z",
    "finished_at": "2026-01-16T10:35:10Z"
  }
}
\end{verbatim}

\textbf{失败处理}:

任务执行失败时,必须写入以下字段:

{\def\LTcaptype{none} % do not increment counter
\begin{longtable}[]{@{}lll@{}}
\toprule\noalign{}
字段 & 要求 & 说明 \\
\midrule\noalign{}
\endhead
\bottomrule\noalign{}
\endlastfoot
\texttt{status} & 必须为 \texttt{failed} & 标记任务失败 \\
\texttt{finished\_at} & 必须写入 & 记录失败时间 \\
\texttt{error} & 必须写入且非空 & 描述失败原因 \\
\end{longtable}
}

\textbf{失败示例}:

\begin{verbatim}
{
  "task": {
    "status": "failed",
    "progress": 35,
    "started_at": "2026-01-16T10:30:05Z",
    "finished_at": "2026-01-16T10:31:20Z",
    "error": "Neo4j query timeout: execution exceeded 30000ms"
  }
}
\end{verbatim}

\subsection{结果读取方式}\label{analysis-result-read}

任务完成后,前端通过图查询接口读取结果。

\subsubsection{查询接口}\label{analysis-query-api}

调用 \texttt{POST\ /api/v1/graph/query} 的 \texttt{analysis\_edges\_by\_task} 动作,按 \texttt{task\_id} 拉取结果边集合。

\textbf{请求示例}:

\begin{verbatim}
{
  "action": "analysis_edges_by_task",
  "params": {
    "task_id": "trace-550e8400-e29b-41d4-a716-446655440000"
  }
}
\end{verbatim}

\textbf{响应示例}:

\begin{verbatim}
{
  "status": "success",
  "data": {
    "edges": [
      {
        "edge_uid": "edge-1",
        "source": "node-ssh-server",
        "target": "node-malicious-ssh-123",
        "relationship": "CONNECTED_TO",
        "properties": {
          "task_id": "trace-550e8400-e29b-41d4-a716-446655440000",
          "path_order": 1,
          "is_attack_path": true
        }
      },
      {
        "edge_uid": "edge-2",
        "source": "node-malicious-ssh-123",
        "target": "node-lateral-move-456",
        "relationship": "EXECUTED_ON",
        "properties": {
          "task_id": "trace-550e8400-e29b-41d4-a716-446655440000",
          "path_order": 2,
          "is_attack_path": true
        }
      }
    ],
    "total": 2
  }
}
\end{verbatim}

\subsubsection{查询结果流程图}\label{analysis-query-flow}

\begin{figure}[htbp]
\centering
\includegraphics[width=\textwidth,height=0.5\textheight,keepaspectratio]{figures/analysis-70/analysis-70-04.pdf}
\caption{查询结果流程图}
\label{fig:analysis-70-04}
\end{figure}

接口权威定义见:

\begin{itemize}
\tightlist
\item
  \texttt{../../80-规范/88-前端与中心机接口.md}
\end{itemize}
