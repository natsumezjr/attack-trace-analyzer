\section{中心机模块}\label{3-center-module}

\section{中心机职责边界}\label{1-center-responsibilities}

中心机承担以下职责:

\begin{enumerate}
\def\labelenumi{\arabic{enumi}.}
\tightlist
\item
  接收并维护客户机注册信息(\texttt{client-registry});
\item
  周期性轮询客户机,拉取事件并存储至 OpenSearch;
\item
  执行检测与告警融合,生成 Canonical Findings;
\item
  对外提供前端 API,涵盖事件查询、告警查询、图查询、任务管理、报告导出等功能;
\item
  执行溯源任务并将结果写回图谱边属性。
\end{enumerate}

\section{FastAPI 应用生命周期}\label{2-fastapi-lifecycle}

中心机后端基于 FastAPI 框架构建,入口文件位于:

\begin{itemize}
\tightlist
\item
  \texttt{backend/main.py}
\end{itemize}

应用启动时将自动启动两个后台任务:

\begin{enumerate}
\def\labelenumi{\arabic{enumi}.}
\tightlist
\item
  客户机轮询服务(周期性拉取客户机事件并入库)
\item
  溯源任务执行器(异步执行溯源任务并写回结果)
\end{enumerate}

关于应用生命周期的详细说明,请参考相关模块代码,入口位于:

\section{中心机系统架构}\label{21-center-system-architecture}

\begin{figure}[htbp]
\centering
\includegraphics[width=\textwidth,height=0.5\textheight,keepaspectratio]{figures/center-60/center-60-01.pdf}
\caption{中心机系统架构}
\label{fig:center-60-01}
\end{figure}

\begin{itemize}
\tightlist
\item
  \texttt{backend/app/services/client\_poller.py}
\item
  \texttt{backend/app/services/analyze/runner.py}
\end{itemize}

\section{代码目录结构}\label{3-code-directory-structure}

中心机后端核心代码组织在 \texttt{backend/app/} 目录下:

\begin{verbatim}
backend/app/
  api/                 路由定义与请求响应模型
  core/                配置、日志、时间工具
  dto/                 DTO 定义
  schemas/             Pydantic schema
  services/
    client_poller.py   轮询服务
    opensearch/        OpenSearch 存储与分析
    neo4j/             Neo4j 入图与图查询
    analyze/           溯源算法与任务执行
\end{verbatim}

\section{模块依赖关系}\label{31-module-dependencies}

\begin{figure}[htbp]
\centering
\includegraphics[width=\textwidth,height=0.5\textheight,keepaspectratio]{figures/center-60/center-60-02.pdf}
\caption{模块依赖关系}
\label{fig:center-60-02}
\end{figure}

\section{API 路由组织}\label{4-api-routing}

路由统一汇总入口:

\begin{itemize}
\tightlist
\item
  \texttt{backend/app/api/router.py}
\end{itemize}

外部 API 的完整定义见:

\begin{itemize}
\tightlist
\item
  \texttt{../../80-规范/88-前端与中心机接口.md}
\end{itemize}

本节仅说明路由模块的组织方式,接口字段定义请参考上述文档。

\section{端到端数据流}\label{41-end-to-end-data-flow}

\begin{figure}[htbp]
\centering
\includegraphics[width=\textwidth,height=0.5\textheight,keepaspectratio]{figures/center-60/center-60-03.pdf}
\caption{端到端数据流}
\label{fig:center-60-03}
\end{figure}

\section{模块集成点}\label{5-module-integration-points}

\section{OpenSearch}\label{51-opensearch}

中心机通过以下模块统一处理 OpenSearch 的读写操作:

\begin{itemize}
\tightlist
\item
  \texttt{backend/app/services/opensearch/}
\end{itemize}

字段规范化与去重由 \texttt{store\_events()} 函数负责,详见 \texttt{62-OpenSearch存储与索引治理.md}。

\section{Neo4j}\label{52-neo4j}

中心机图查询与入图操作通过以下模块实现:

\begin{itemize}
\tightlist
\item
  \texttt{backend/app/services/neo4j/}
\end{itemize}

具体实现见 \texttt{64-Neo4j入图与图查询.md} 和 \texttt{../../80-规范/84-Neo4j实体图谱规范.md}。

\section{Analysis}\label{53-analysis}

中心机溯源任务与算法的核心入口:

\begin{itemize}
\tightlist
\item
  \texttt{backend/app/services/analyze/}
\end{itemize}

相关算法设计详见 \texttt{../../50-详细设计/分析/}。

\section{运维入口}\label{6-operations-entry}

中心机的启动流程、健康检查、数据清理以及演示准备步骤均在运维文档中说明:

\begin{itemize}
\tightlist
\item
  \texttt{../../90-运维与靶场/90-编译安装与使用.md}
\end{itemize}
