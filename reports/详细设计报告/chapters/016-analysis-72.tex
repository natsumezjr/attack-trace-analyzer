\subsection{LLM 选择器与回退机制}\label{sec:llm-selector-fallback}

\subsubsection{输入裁剪规则}\label{subsec:input-truncation-rules}

LLM 选择器的输入数据由 \texttt{killchain.py} 生成,经由 \texttt{killchain\_llm.py} 进行二次裁剪处理,确保满足以下要求:

\begin{enumerate}
\def\labelenumi{\arabic{enumi}.}
\tightlist
\item
  输入规模可控(字段数量与文本长度均受约束);
\item
  输入信息可完整回溯至图中的边与路径(保留 \texttt{path\_id} 及关键步骤信息);
\item
  输入数据在回退模式下具备完全可复现性(不依赖外部服务状态)。
\end{enumerate}

代码实现绑定点:

\begin{itemize}
\tightlist
\item
  原始输入生成:\texttt{backend/app/services/analyze/killchain.py:build\_llm\_payload()}
\item
  二次裁剪:\texttt{backend/app/services/analyze/killchain\_llm.py:PayloadReducer.reduce()}
\item
  启发式预筛选:\texttt{backend/app/services/analyze/killchain\_llm.py:HeuristicPreselector.preselect()}
\end{itemize}

\subsubsection{原始输入结构(固定)}\label{subsec:original-input-structure}

\texttt{build\_llm\_payload()} 生成的 payload 采用 JSON 对象结构,顶层字段定义如下:

\begin{itemize}
\tightlist
\item
  \texttt{constraints}:全局约束条件(包含时间窗、锚点等信息);
\item
  \texttt{segments\ 数组}:按 ATT\&CK 战术分段的异常摘要信息;
\item
  \texttt{pairs\ 数组}:相邻分段之间的"锚点对"及候选连接路径集合。
\end{itemize}

字段详细说明:

\begin{itemize}
\tightlist
\item
  \texttt{segments\ 数组.abnormal\_edge\_summaries\ 数组}:元素来源于边摘要(用于结果解释与路径选择);
\item
  \texttt{pairs\ 数组.candidates{[}{]}}:元素为候选路径(每条路径包含 \texttt{steps\ 数组})。
\end{itemize}

\subsubsection{二次裁剪(PayloadReducer,固定参数)}\label{subsec:payload-reducer}

二次裁剪模块将大规模 payload 压缩为 reduced payload,压缩规则与参数配置固定如下:

\begin{enumerate}
\def\labelenumi{\arabic{enumi}.}
\tightlist
\item
  文本截断长度阈值:\texttt{max\_str\_len\ 为\ 200};
\item
  单路径最大保留步骤数:\texttt{max\_steps\_per\_path\ 为\ 10};
\item
  每个步骤的 \texttt{key\_props} 仅保留固定字段集合(参见下表)。
\end{enumerate}

代码实现绑定点:\texttt{backend/app/services/analyze/killchain\_llm.py:LLMChooseConfig} 与 \texttt{DEFAULT\_EDGE\_KEYS\_KEEP}。

\paragraph{Reduced Payload 顶层结构(固定)}\label{subsec:reduced-payload-structure}

裁剪后的数据结构定义如下:

\begin{itemize}
\tightlist
\item
  \texttt{constraints}:原样保留(转换为普通 dict 格式);
\item
  \texttt{segments\ 数组}:保留分段元信息及裁剪后的异常摘要;
\item
  \texttt{pairs\ 数组}:保留段对元信息及裁剪后的候选路径步骤。
\end{itemize}

\paragraph{steps.key\_props 保留字段集合(固定)}\label{subsec:key-props-retention}

每个步骤的 \texttt{key\_props} 仅保留以下键值(其余键值被丢弃):

\begin{itemize}
\tightlist
\item
  \texttt{edge\_id}
\item
  \texttt{ts}
\item
  \texttt{src\_uid}
\item
  \texttt{dst\_uid}
\item
  \texttt{rel}
\item
  \texttt{event.id}
\item
  \texttt{event.dataset}
\item
  \texttt{event.action}
\item
  \texttt{rule.name}
\item
  \texttt{threat.tactic.name}
\item
  \texttt{threat.technique.name}
\item
  \texttt{host.id}
\item
  \texttt{host.name}
\item
  \texttt{user.name}
\item
  \texttt{process.entity\_id}
\item
  \texttt{process.name}
\item
  \texttt{process.command\_line}
\item
  \texttt{source.ip}
\item
  \texttt{destination.ip}
\item
  \texttt{dns.question.name}
\item
  \texttt{domain.name}
\end{itemize}

\paragraph{Payload 裁剪流程}\label{subsec:payload-truncation-flow}

\begin{figure}[htbp]
\centering
\includegraphics[width=\textwidth,height=0.5\textheight,keepaspectratio]{figures/analysis-72/analysis-72-01.pdf}
\caption{Payload 裁剪流程}
\label{fig:analysis-72-01}
\end{figure}

\subsubsection{启发式预筛(HeuristicPreselector,固定参数)}\label{subsec:heuristic-preselector}

基于 reduced payload,系统执行启发式预筛选操作,进一步削减每个段对的候选路径数量。

固定参数:

\begin{itemize}
\tightlist
\item
  每个段对最多保留候选数:\texttt{per\_pair\_keep\ 为\ 8}
\end{itemize}

启发式评分规则定义如下:

\begin{enumerate}
\def\labelenumi{\arabic{enumi}.}
\tightlist
\item
  路径跳数定义:\texttt{hop=len(steps)};
\item
  基础得分计算:\texttt{base\ =\ 10.0\ /\ (1.0\ +\ hop)};
\item
  连续性得分:统计当前候选与"全链滚动上下文 token 集合"的交集数量 \texttt{overlap},加权得分为 \texttt{0.5\ *\ overlap};
\item
  综合得分公式:\texttt{score\ =\ base\ +\ 0.5\ *\ overlap}。
\end{enumerate}

滚动上下文 token 集合包含以下固定维度:

\begin{itemize}
\tightlist
\item
  \texttt{proc}:\texttt{process.entity\_id}
\item
  \texttt{host}:\texttt{host.id}
\item
  \texttt{user}:\texttt{user.name}
\item
  \texttt{ip}:\texttt{source.ip} 与 \texttt{destination.ip}
\item
  \texttt{domain}:\texttt{dns.question.name} 与 \texttt{domain.name}
\end{itemize}

预筛选输出行为定义如下:

\begin{enumerate}
\def\labelenumi{\arabic{enumi}.}
\tightlist
\item
  每个 \texttt{pair.candidates{[}{]}} 按 score 降序排列;
\item
  截断保留 Top \texttt{per\_pair\_keep} 个候选;
\item
  在 \texttt{pair.heuristic\_ranking\ 数组} 中记录评分结果与依据,便于调试分析。
\end{enumerate}

\paragraph{启发式评分公式}\label{subsec:heuristic-scoring-formula}

\begin{figure}[htbp]
\centering
\includegraphics[width=\textwidth,height=0.5\textheight,keepaspectratio]{figures/analysis-72/analysis-72-02.pdf}
\caption{启发式评分公式}
\label{fig:analysis-72-02}
\end{figure}

\paragraph{评分示例}\label{subsec:scoring-examples}

{\def\LTcaptype{none} % do not increment counter
\footnotesize
\begin{longtable}[]{@{}p{0.08\textwidth}p{0.12\textwidth}p{0.12\textwidth}p{0.15\textwidth}p{0.12\textwidth}p{0.25\textwidth}@{}}
\toprule\noalign{}
hop & base & overlap & 连续性得分 & score & 说明 \\
\midrule\noalign{}
\endhead
\bottomrule\noalign{}
\endlastfoot
1 & 5.0 & 3 & 1.5 & 6.5 & 最短路径,连续性最高 \\
2 & 3.33 & 2 & 1.0 & 4.33 & 中等路径,连续性中等 \\
3 & 2.5 & 1 & 0.5 & 3.0 & 较长路径,连续性较低 \\
5 & 1.67 & 0 & 0.0 & 1.67 & 长路径,无连续性 \\
\end{longtable}
}

\subsubsection{输出结构与校验规则}\label{subsec:output-structure-validation}

LLM chooser 对外输出采用 JSON 对象格式(Python dict),字段集合与校验规则配置固定。

代码实现绑定点:

\begin{itemize}
\tightlist
\item
  输出校验:\texttt{backend/app/services/analyze/killchain\_llm.py:validate\_choose\_result()}
\item
  JSON 提取:\texttt{backend/app/services/analyze/killchain\_llm.py:\_extract\_json\_obj()}
\end{itemize}

\paragraph{输出结构(固定字段)}\label{subsec:output-structure-fields}

返回对象的字段定义如下:

\begin{enumerate}
\def\labelenumi{\arabic{enumi}.}
\tightlist
\item
  \texttt{chosen\_path\_ids:\ list{[}str{]}}
\item
  \texttt{explanation:\ str}
\item
  \texttt{confidence:\ float}(取值范围 \texttt{0.0..1.0})
\item
  \texttt{pair\_explanations:\ list{[}object{]}}
\end{enumerate}

字段语义说明:

\begin{itemize}
\tightlist
\item
  \texttt{chosen\_path\_ids{[}i{]}}:表示第 \texttt{i} 个 \texttt{pairs{[}i{]}} 中被选中的 \texttt{path\_id};
\item
  \texttt{explanation}:整条攻击链的全局解释文本;
\item
  \texttt{confidence}:攻击链解释结果的置信度评估;
\item
  \texttt{pair\_explanations}:按段对组织的解释数组,用于前端分段展示。
\end{itemize}

\paragraph{校验规则(固定)}\label{subsec:validation-rules}

输出需满足以下全部条件方可判定为"有效 LLM 输出":

\begin{enumerate}
\def\labelenumi{\arabic{enumi}.}
\tightlist
\item
  \texttt{chosen\_path\_ids} 字段存在且类型为 \texttt{list{[}str{]}};
\item
  \texttt{len(chosen\_path\_ids)\ ==\ len(pairs)} 长度匹配;
\item
  对每个索引 \texttt{i},\texttt{chosen\_path\_ids{[}i{]}} 必须属于 \texttt{pairs{[}i{]}.candidates{[}{]}.path\_id} 集合。
\end{enumerate}

\texttt{confidence} 字段的处理规则:

\begin{itemize}
\tightlist
\item
  当 LLM 输出 \texttt{confidence} 为数值时,系统将其裁剪至 \texttt{0.0..1.0} 区间;
\item
  当 LLM 未输出 \texttt{confidence} 或类型错误时,系统默认设置为 \texttt{0.5}。
\end{itemize}

\subparagraph{输出校验流程}\label{subsec:output-validation-flow}

\begin{figure}[htbp]
\centering
\includegraphics[width=\textwidth,height=0.5\textheight,keepaspectratio]{figures/analysis-72/analysis-72-03.pdf}
\caption{输出校验流程}
\label{fig:analysis-72-03}
\end{figure}

\subsubsection{失败判定条件}\label{subsec:failure-conditions}

当以下任一条件满足时,系统将自动进入回退模式:

\paragraph{Payload 层失败(固定)}\label{subsec:payload-layer-failure}

\begin{enumerate}
\def\labelenumi{\arabic{enumi}.}
\tightlist
\item
  \texttt{pairs\ 数组} 为空(仅包含单段攻击阶段,无需段间连接);
\item
  \texttt{pairs\ 数组} 非空但存在段对无候选路径(该段对输出空字符串)。
\end{enumerate}

\paragraph{LLM 调用层失败(固定)}\label{subsec:llm-call-failure}

\begin{enumerate}
\def\labelenumi{\arabic{enumi}.}
\tightlist
\item
  未注入 \texttt{chat\_complete}(即没有可用的大模型调用函数);
\item
  \texttt{chat\_complete(messages)} 抛出异常;
\item
  大模型返回内容无法提取 JSON 对象;
\item
  提取出的 JSON 对象未通过 \texttt{validate\_choose\_result()} 校验。
\end{enumerate}

\paragraph{Provider 创建失败(固定)}\label{subsec:provider-creation-failure}

LLM client 的创建入口为 \texttt{create\_llm\_client()},provider 选择逻辑如下:

\begin{enumerate}
\def\labelenumi{\arabic{enumi}.}
\tightlist
\item
  当 \texttt{LLM\_PROVIDER="mock"} 时,返回 \texttt{MockChooser};
\item
  当 \texttt{LLM\_PROVIDER="deepseek"} 且 \texttt{DEEPSEEK\_API\_KEY} 为空字符串时,返回 \texttt{MockChooser};
\item
  当 \texttt{LLM\_PROVIDER="deepseek"} 且 \texttt{DEEPSEEK\_API\_KEY} 非空时,返回带 \texttt{chat\_complete} 的 \texttt{LLMChooser};
\item
  当 \texttt{LLM\_PROVIDER} 为其它值时,返回 \texttt{MockChooser}。
\end{enumerate}

实现绑定点:\texttt{backend/app/services/analyze/killchain\_llm.py:create\_llm\_client()}。

\paragraph{上层兜底失败(固定)}\label{subsec:upper-layer-fallback}

当 \texttt{create\_llm\_client()} 因依赖缺失等原因抛出异常时,上层 \texttt{analyze\_killchain()} 会将 \texttt{llm\_client} 置为 \texttt{None},最终由 \texttt{killchain.py} 的内置回退逻辑完成段对连接。

代码实现绑定点:

\begin{itemize}
\tightlist
\item
  上层入口:\texttt{backend/app/services/analyze/\_\_init\_\_.py:analyze\_killchain()}
\item
  内置回退:\texttt{backend/app/services/analyze/killchain.py:select\_killchain\_with\_llm()}
\end{itemize}

\paragraph{LLM 选择器流程图}\label{subsec:llm-selector-flowchart}

\begin{figure}[htbp]
\centering
\includegraphics[width=\textwidth,height=0.5\textheight,keepaspectratio]{figures/analysis-72/analysis-72-04.pdf}
\caption{LLM 选择器流程图}
\label{fig:analysis-72-04}
\end{figure}

\subsubsection{回退算法}\label{subsec:fallback-algorithm}

回退算法包含两个层次:

\begin{enumerate}
\def\labelenumi{\arabic{enumi}.}
\tightlist
\item
  \texttt{killchain\_llm.py:fallback\_choose()}:\texttt{LLMChooser} 与 \texttt{MockChooser} 的统一回退机制;
\item
  \texttt{killchain.py:select\_killchain\_with\_llm()}:上层未提供可用 chooser 时的回退处理。
\end{enumerate}

\paragraph{回退层级结构}\label{subsec:fallback-hierarchy}

\begin{figure}[htbp]
\centering
\includegraphics[width=\textwidth,height=0.5\textheight,keepaspectratio]{figures/analysis-72/analysis-72-05.pdf}
\caption{回退层级结构}
\label{fig:analysis-72-05}
\end{figure}

\paragraph{LLMChooser 回退(fallback\_choose,固定)}\label{subsec:llmchooser-fallback}

回退算法对每个段对执行以下操作:

\begin{enumerate}
\def\labelenumi{\arabic{enumi}.}
\tightlist
\item
  在 \texttt{pair.candidates{[}{]}} 中选择 \texttt{len(steps)} 最小的候选路径;
\item
  将该候选的 \texttt{path\_id} 写入 \texttt{chosen\_path\_ids};
\item
  当段对候选集合为空时,该段对输出空字符串 \texttt{""}。
\end{enumerate}

回退输出的字段配置:

\begin{itemize}
\tightlist
\item
  \texttt{confidence\ 为\ 0.5}
\item
  \texttt{pair\_explanations={[}{]}}
\item
  \texttt{explanation} 使用 \texttt{fallback\_choose()} 内置解释文本(不依赖外部输入)。
\end{itemize}

\paragraph{killchain.py 内置回退(固定)}\label{subsec:killchain-builtin-fallback}

当 \texttt{llm\_client} 不可用或 LLM 选择失败时,\texttt{killchain.py} 会对每个段对的候选路径集合,选择边数最少(\texttt{len(edges)} 最小)的候选路径。

该回退模式下输出的 \texttt{confidence} 固定为 \texttt{0.5},解释文本由 \texttt{killchain.py} 内置常量生成。

\paragraph{回退机制对比}\label{subsec:fallback-comparison}

{\def\LTcaptype{none} % do not increment counter
\footnotesize
\begin{longtable}[]{@{}p{0.12\textwidth}p{0.15\textwidth}p{0.15\textwidth}p{0.18\textwidth}p{0.30\textwidth}@{}}
\toprule\noalign{}
LLM & 模型 & 用途 & 回退策略 & 触发条件 \\
\midrule\noalign{}
\endhead
\bottomrule\noalign{}
\endlastfoot
LLM-1 & DeepSeek & 主要解释 & 无回退 & {\scriptsize \texttt{LLM\_PROVIDER="deepseek"} 且 API Key 非空} \\
LLM-2 & Qwen & 备用解释 & LLM-1 失败时启用 & 预留扩展,当前未实现 \\
Rule-based & 规则引擎 & 兜底 & 所有 LLM 失败时启用 & {\scriptsize • Payload 层失败 • LLM 调用层失败 • Provider 创建失败 • 输出校验失败} \\
\end{longtable}
}

\subsubsection{配置参数}\label{subsec:configuration-parameters}

\paragraph{模型选择参数}\label{subsec:model-selection-params}

{\def\LTcaptype{none} % do not increment counter
\footnotesize
\begin{longtable}[]{@{}p{0.22\textwidth}p{0.12\textwidth}p{0.18\textwidth}p{0.38\textwidth}@{}}
\toprule\noalign{}
参数 & 类型 & 默认值 & 说明 \\
\midrule\noalign{}
\endhead
\bottomrule\noalign{}
\endlastfoot
{\scriptsize\texttt{LLM\_PROVIDER}} & string & {\scriptsize\texttt{"mock"}} & {\scriptsize LLM 提供商:可选 \texttt{mock} 或 \texttt{deepseek}} \\
{\scriptsize\texttt{DEEPSEEK\_API\_KEY}} & string & {\scriptsize\texttt{""}} & {\scriptsize DeepSeek API 密钥,为空时自动切换至 MockChooser} \\
\end{longtable}
}

\paragraph{裁剪参数}\label{subsec:truncation-params}

{\def\LTcaptype{none} % do not increment counter
\footnotesize
\begin{longtable}[]{@{}p{0.20\textwidth}p{0.10\textwidth}p{0.12\textwidth}p{0.23\textwidth}p{0.25\textwidth}@{}}
\toprule\noalign{}
参数 & 类型 & 默认值 & 位置 & 说明 \\
\midrule\noalign{}
\endhead
\bottomrule\noalign{}
\endlastfoot
{\scriptsize\texttt{max\_str\_len}} & int & {\scriptsize\texttt{200}} & PayloadReducer & 文本字段最大长度限制 \\
{\scriptsize\texttt{max\_steps\_per\_path}} & int & {\scriptsize\texttt{10}} & PayloadReducer & 单路径最大保留步骤数 \\
{\scriptsize\texttt{per\_pair\_keep}} & int & {\scriptsize\texttt{8}} & HeuristicPreselector & 每个段对最大保留候选数 \\
\end{longtable}
}

\paragraph{LLM 调用参数}\label{subsec:llm-call-params}

{\def\LTcaptype{none} % do not increment counter
\footnotesize
\begin{longtable}[]{@{}p{0.22\textwidth}p{0.15\textwidth}p{0.18\textwidth}p{0.35\textwidth}@{}}
\toprule\noalign{}
参数 & 类型 & 默认值 & 说明 \\
\midrule\noalign{}
\endhead
\bottomrule\noalign{}
\endlastfoot
{\scriptsize\texttt{temperature}} & float & {\scriptsize\texttt{0.3}} & DeepSeek 温度参数配置 \\
{\scriptsize\texttt{response\_format}} & string & {\scriptsize\texttt{json\_object}} & 强制 JSON 输出格式 \\
{\scriptsize\texttt{timeout}} & int & - & API 调用超时时间(待实现) \\
{\scriptsize\texttt{max\_retries}} & int & - & 最大重试次数(待实现) \\
\end{longtable}
}

\paragraph{配置示例}\label{subsec:configuration-examples}

\subparagraph{生产环境配置}\label{para:production-config}

\begin{verbatim}
## 使用 DeepSeek 进行 LLM 解释
export LLM_PROVIDER="deepseek"
export DEEPSEEK_API_KEY="sk-your-api-key-here"
\end{verbatim}

\subparagraph{开发/测试环境配置}\label{para:dev-test-config}

\begin{verbatim}
## 使用 Mock 模式(不调用真实 API)
export LLM_PROVIDER="mock"
export DEEPSEEK_API_KEY=""
\end{verbatim}

\subparagraph{禁用 LLM 功能}\label{para:llm-disabled}

\begin{verbatim}
## 设置为空或任意非 deepseek 值,系统自动回退到规则引擎
export LLM_PROVIDER=""
export DEEPSEEK_API_KEY=""
\end{verbatim}

\subsubsection{可复现性保证}\label{subsec:reproducibility-guarantee}

可复现性保证规则定义如下:

\begin{enumerate}
\def\labelenumi{\arabic{enumi}.}
\tightlist
\item
  reduced payload 的裁剪参数固定(参见 1.2 与 1.3 节);
\item
  回退算法仅依赖 reduced payload,不引入随机因素;
\item
  当候选路径的 \texttt{steps} 数量相同时,Python 的稳定排序会保留原始候选顺序,确保回退结果对同一输入保持一致性;
\item
  DeepSeek 调用固定使用 \texttt{temperature\ 为\ 0.3},在支持时启用 \texttt{response\_format=\{"type":"json\_object"\}},并通过严格校验将异常输出统一收敛至回退模式。
\end{enumerate}

DeepSeek 适配实现绑定点:\texttt{backend/app/services/analyze/killchain\_llm.py:\_create\_llm\_chat\_complete()}。
