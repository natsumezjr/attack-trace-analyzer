\subsubsection{KillChain 结果展示规范}\label{subsec:killchain-display-specs}

\paragraph{1. 展示模式}\label{para:display-modes}

\subparagraph{1.1 基础版视图(默认展开)}\label{subpara:basic-view}

基础版视图呈现 KillChain 核心信息的精要概览,包括以下内容:

\begin{enumerate}
\def\labelenumi{\arabic{enumi}.}
\item
  \textbf{可信度评分条}

  \begin{itemize}
  \tightlist
  \item
    可视化评分条(0.0 - 1.0)
  \item
    颜色规则:

    \begin{itemize}
    \tightlist
    \item
      \texttt{\textgreater{}\ 0.7}:绿色(高可信度)
    \item
      \texttt{0.4\ -\ 0.7}:黄色(中等可信度)
    \item
      \texttt{\textless{}\ 0.4}:红色(低可信度)
    \end{itemize}
  \item
    显示数值百分比
  \end{itemize}
\item
  \textbf{MITRE ATT\&CK 战术时间线}

  \begin{itemize}
  \tightlist
  \item
    横向流程图展示各战术阶段
  \item
    格式:\texttt{{[}Initial\ Access{]}\ →\ {[}Execution{]}\ →\ {[}...{]}\ →\ {[}Impact{]}}
  \item
    每个阶段显示:

    \begin{itemize}
    \tightlist
    \item
      战术名称(中文)
    \item
      时间范围(可选 hover 显示)
    \end{itemize}
  \end{itemize}
\item
  \textbf{LLM 解释摘要}

  \begin{itemize}
  \tightlist
  \item
    显示 explanation 的前 3 句话
  \item
    省略号提示有更多内容
  \item
    "查看详情"按钮
  \end{itemize}
\item
  \textbf{操作按钮}

  \begin{itemize}
  \tightlist
  \item
    "查看详情"(展开完整版)
  \item
    "导出 KillChain"(导出为 Markdown)
  \end{itemize}
\end{enumerate}

\subparagraph{1.2 完整版视图(点击展开)}\label{subpara:full-view}

完整版视图展示 KillChain 的完整分析细节,包含以下信息:

\begin{enumerate}
\def\labelenumi{\arabic{enumi}.}
\item
  \textbf{概览卡片}

  \begin{itemize}
  \tightlist
  \item
    kc\_uuid(可复制)
  \item
    confidence(评分条)
  \item
    segment\_count(段数量)
  \item
    selected\_path\_count(路径数量)
  \end{itemize}
\item
  \textbf{完整 MITRE ATT\&CK 战术时间线}

  \begin{itemize}
  \tightlist
  \item
    可交互的时间线
  \item
    点击阶段展开查看详情
  \item
    每个阶段显示:

    \begin{itemize}
    \tightlist
    \item
      战术状态(state)
    \item
      时间范围(t\_start, t\_end)
    \item
      入口/出口锚点(anchor\_in\_uid, anchor\_out\_uid)
    \item
      异常边数量(abnormal\_edge\_count)
    \end{itemize}
  \end{itemize}
\item
  \textbf{完整 LLM 解释文本}

  \begin{itemize}
  \tightlist
  \item
    10-20 句中文完整解释
  \item
    可折叠/展开
  \item
    关键信息高亮(实体、IP、进程等)
  \item
    "复制"按钮
  \item
    "导出为 Markdown"按钮
  \end{itemize}
\item
  \textbf{段间连接路径列表}

  \begin{itemize}
  \tightlist
  \item
    每条路径显示:

    \begin{itemize}
    \tightlist
    \item
      path\_id
    \item
      源锚点 → 目标锚点
    \item
      hop 数量
    \item
      边 ID 列表(可折叠)
    \item
      "在图谱中高亮"按钮
    \end{itemize}
  \end{itemize}
\item
  \textbf{返回基础版}

  \begin{itemize}
  \tightlist
  \item
    "收起详情"按钮
  \end{itemize}
\end{enumerate}

\begin{center}\rule{0.5\linewidth}{0.5pt}\end{center}

\paragraph{2. API 接口规范}\label{para:api-interface-specs}

\subparagraph{2.1 通过任务 ID 查询(主接口)}\label{subpara:query-by-task-id}

\textbf{接口}:\texttt{GET\ /api/v1/analysis/tasks/\{task\_id\}}

\textbf{响应扩展}

\begin{verbatim}
{
  "status": "ok",
  "task": {
    "task.id": "trace-abc-123",
    "task.status": "succeeded",
    "task.result": {
      "killchain_uuid": "uuid-xxx-yyy",  // 新增:KillChain UUID
      "killchain": {                      // 新增(可选):完整数据
        "kc_uuid": "uuid-xxx-yyy",
        "confidence": 0.85,
        "segments": [
          {
            "seg_idx": 0,
            "state": "INITIAL_ACCESS",
            "t_start": 1234567890.0,
            "t_end": 1234567900.0,
            "anchor_in_uid": "Process:pid=1234",
            "anchor_out_uid": "Host:host.id=web-001",
            "abnormal_edge_count": 3
          }
        ],
        "selected_paths": [
          {
            "path_id": "p-abc123",
            "src_anchor": "Host:host.id=web-001",
            "dst_anchor": "Process:pid=5678",
            "hop_count": 3,
            "edge_ids": ["event-id-1", "event-id-2", "event-id-3"]
          }
        ],
        "explanation": "攻击者进程 p_c2 (pid:1234) 从外部 IP 1.2.3.4..."
      }
    }
  }
}
\end{verbatim}

\subparagraph{2.2 通过 kc\_uuid 直接查询(可选)}\label{subpara:query-by-kc-uuid}

\textbf{接口}:\texttt{GET\ /api/v1/analysis/killchain/\{kc\_uuid\}}

\textbf{响应结构}

\begin{verbatim}
{
  "status": "ok",
  "killchain": {
    "kc_uuid": "uuid-xxx-yyy",
    "confidence": 0.85,
    "segments": [...],
    "selected_paths": [...],
    "explanation": "..."
  }
}
\end{verbatim}

\textbf{错误响应}

\begin{itemize}
\tightlist
\item
  \texttt{404\ Not\ Found}:kc\_uuid 不存在
\item
  \texttt{500\ Internal\ Server\ Error}:查询失败
\end{itemize}

\begin{center}\rule{0.5\linewidth}{0.5pt}\end{center}

\paragraph{3. 数据字段映射}\label{para:data-field-mapping}

\subparagraph{3.1 Segment 字段映射}\label{subpara:segment-field-mapping}

{\def\LTcaptype{none} % do not increment counter
\footnotesize
\begin{longtable}[]{@{}p{0.20\textwidth}p{0.12\textwidth}p{0.28\textwidth}p{0.30\textwidth}@{}}
\toprule\noalign{}
API 字段 & 显示名称 & 格式 & 说明 \\
\midrule\noalign{}
\endhead
\bottomrule\noalign{}
\endlastfoot
\texttt{seg\_idx} & 段索引 & 整数 & 从 0 开始 \\
\texttt{state} & 战术状态 & 枚举字符串 & 对应中文见下表 \\
\texttt{t\_start} & 起始时间 & ISO 8601 & 可转换为本地时间 \\
\texttt{t\_end} & 结束时间 & ISO 8601 & 可转换为本地时间 \\
\texttt{anchor\_in\_uid} & 入口锚点 & UID 字符串 & 可截取显示 \\
\texttt{anchor\_out\_uid} & 出口锚点 & UID 字符串 & 可截取显示 \\
\texttt{abnormal\_edge\_count} & 异常边数 & 整数 & top 6 异常边 \\
\end{longtable}
}

\subparagraph{3.2 战术状态中英文映射}\label{subpara:tactic-state-mapping}

{\def\LTcaptype{none} % do not increment counter
\footnotesize
\begin{longtable}[]{@{}p{0.35\textwidth}p{0.55\textwidth}@{}}
\toprule\noalign{}
state (英文) & 中文显示 \\
\midrule\noalign{}
\endhead
\bottomrule\noalign{}
\endlastfoot
\texttt{INITIAL\_ACCESS} & 初始入侵 \\
\texttt{EXECUTION} & 执行 \\
\texttt{PRIVILEGE\_ESCALATION} & 权限提升 \\
\texttt{LATERAL\_MOVEMENT} & 横向移动 \\
\texttt{COMMAND\_AND\_CONTROL} & 命令与控制 \\
\texttt{DISCOVERY} & 发现 \\
\texttt{IMPACT} & 影响 \\
\end{longtable}
}

\subparagraph{3.3 Path 字段映射}\label{subpara:path-field-mapping}

{\def\LTcaptype{none} % do not increment counter
\footnotesize
\begin{longtable}[]{@{}p{0.20\textwidth}p{0.12\textwidth}p{0.28\textwidth}p{0.30\textwidth}@{}}
\toprule\noalign{}
API 字段 & 显示名称 & 格式 & 说明 \\
\midrule\noalign{}
\endhead
\bottomrule\noalign{}
\endlastfoot
\texttt{path\_id} & 路径 ID & 字符串 & 可复制 \\
\texttt{src\_anchor} & 源锚点 & UID 字符串 & 简化显示 \\
\texttt{dst\_anchor} & 目标锚点 & UID 字符串 & 简化显示 \\
\texttt{hop\_count} & 跳数 & 整数 & 边的数量 \\
\texttt{edge\_ids} & 边 ID 列表 & 字符串数组 & 可折叠显示 \\
\end{longtable}
}

\begin{center}\rule{0.5\linewidth}{0.5pt}\end{center}

\paragraph{4. 图谱高亮规则}\label{para:graph-highlighting-rules}

\subparagraph{4.1 高亮触发方式}\label{subpara:highlight-trigger-modes}

\begin{enumerate}
\def\labelenumi{\arabic{enumi}.}
\tightlist
\item
  \textbf{自动高亮}:当 KillChain 面板展开时,自动高亮所有 killchain 路径
\item
  \textbf{手动高亮}:点击"在图谱中高亮"按钮,高亮特定路径
\item
  \textbf{清除高亮}:关闭 KillChain 面板时,清除所有高亮
\end{enumerate}

\subparagraph{4.2 高亮样式规范}\label{subpara:highlight-style-specs}

KillChain 路径边采用以下固定样式:

\begin{verbatim}
// 边样式
{
  stroke: '#FF6B6B',      // 红色
  lineWidth: 3,           // 线宽
  lineDash: [5, 5],       // 虚线
  opacity: 0.9            // 不透明度
}

// 节点样式(路径上的节点)
{
  fill: '#FFE5E5',        // 浅红色填充
  stroke: '#FF6B6B',      // 红色边框
  lineWidth: 2            // 边框宽度
}
\end{verbatim}

\subparagraph{4.3 高亮性能优化}\label{subpara:highlight-performance-optimization}

为防止性能瓶颈,高亮功能需满足以下约束:

\begin{enumerate}
\def\labelenumi{\arabic{enumi}.}
\tightlist
\item
  \textbf{最大高亮边数}:限制为 100 条
\item
  \textbf{延迟渲染}:使用 \texttt{requestAnimationFrame} 分批渲染
\item
  \textbf{虚拟滚动}:对于大量节点,使用虚拟滚动
\item
  \textbf{取消机制}:高亮请求可被后续请求取消
\end{enumerate}

\begin{center}\rule{0.5\linewidth}{0.5pt}\end{center}

\paragraph{5. UI 组件规范}\label{para:ui-component-specs}

\subparagraph{5.1 组件接口}\label{subpara:component-interfaces}

\begin{verbatim}
interface KillChainPanelProps {
  task: AnalysisTaskItem;           // 分析任务数据
  onHighlightPath?: (pathIds: string[]) => void;  // 图谱高亮回调
  onExport?: (killchain: KillChainData) => void;  // 导出回调
}

interface KillChainData {
  kc_uuid: string;
  confidence: number;
  segments: SegmentData[];
  selected_paths: PathData[];
  explanation: string;
}

interface SegmentData {
  seg_idx: number;
  state: string;
  t_start: string;  // ISO 8601
  t_end: string;
  anchor_in_uid: string;
  anchor_out_uid: string;
  abnormal_edge_count: number;
}

interface PathData {
  path_id: string;
  src_anchor: string;
  dst_anchor: string;
  hop_count: number;
  edge_ids: string[];
}
\end{verbatim}

\subparagraph{5.2 组件状态}\label{subpara:component-state}

\begin{verbatim}
interface KillChainPanelState {
  isExpanded: boolean;           // 是否展开完整版
  isLoading: boolean;            // 是否正在加载
  error: string | null;          // 错误信息
  highlightedPathId: string | null;  // 当前高亮的路径 ID
}
\end{verbatim}

\subparagraph{5.3 交互行为}\label{subpara:interaction-behaviors}

\begin{enumerate}
\def\labelenumi{\arabic{enumi}.}
\item
  \textbf{默认状态}

  \begin{itemize}
  \tightlist
  \item
    显示基础版视图
  \item
    \texttt{isExpanded\ =\ false}
  \end{itemize}
\item
  \textbf{点击"查看详情"}

  \begin{itemize}
  \tightlist
  \item
    展开完整版视图
  \item
    \texttt{isExpanded\ =\ true}
  \item
    自动高亮图谱上的所有 killchain 路径
  \item
    触发 \texttt{onHighlightPath(allPathIds)}
  \end{itemize}
\item
  \textbf{点击"收起详情"}

  \begin{itemize}
  \tightlist
  \item
    折叠到基础版视图
  \item
    \texttt{isExpanded\ =\ false}
  \item
    清除图谱高亮
  \item
    触发 \texttt{onHighlightPath({[}{]})}
  \end{itemize}
\item
  \textbf{点击特定路径的"在图谱中高亮"}

  \begin{itemize}
  \tightlist
  \item
    高亮该路径的边
  \item
    触发 \texttt{onHighlightPath({[}pathId{]})}
  \end{itemize}
\item
  \textbf{点击"导出 KillChain"}

  \begin{itemize}
  \tightlist
  \item
    生成 Markdown 格式的 killchain 报告
  \item
    触发文件下载
  \item
    触发 \texttt{onExport(killchainData)}
  \end{itemize}
\end{enumerate}

\begin{center}\rule{0.5\linewidth}{0.5pt}\end{center}

\paragraph{6. 报告导出格式}\label{para:report-export-format}

\subparagraph{6.1 Markdown 章节}\label{subpara:markdown-sections}

导出的 KillChain 章节格式规范如下:

\begin{verbatim}
## 6. KillChain 攻击链分析
## 6.1 概览
- kc_uuid: `uuid-xxx-yyy`
- confidence: `0.85`
- segment_count: `6`
- selected_path_count: `5`

## 6.2 MITRE ATT&CK 战术分段
[初始入侵] → [执行] → [权限提升] → [横向移动] → [命令与控制] → [影响]

##### 分段详情
**1. 初始入侵 (INITIAL_ACCESS)**
- 时间范围:2025-01-16T10:00:00Z → 2025-01-16T10:05:00Z
- 入口锚点:Process:pid=1234
- 出口锚点:Host:host.id=web-001
- 异常边数:3

**2. 执行 (EXECUTION)**
...

## 6.3 LLM 全链解释
攻击者进程 p_c2 (pid:1234) 从外部 IP 1.2.3.4 连接到受害主机 host_web (host.id:host-001)...
(完整 10-20 句解释)

## 6.4 段间连接路径
**路径 1**: p-abc123
- 源锚点:Host:host.id=web-001
- 目标锚点:Process:pid=5678
- hop 数量:3
- 边 ID 列表:event-id-1, event-id-2, event-id-3

**路径 2**: ...
\end{verbatim}

\subparagraph{6.2 导出触发条件}\label{subpara:export-trigger-conditions}

\begin{enumerate}
\def\labelenumi{\arabic{enumi}.}
\tightlist
\item
  \textbf{手动导出}:点击"导出 KillChain"按钮
\item
  \textbf{自动包含}:在完整溯源报告导出时自动包含该章节
\item
  \textbf{可选包含}:在报告导出设置中提供"包含 KillChain 分析"勾选项
\end{enumerate}

\begin{center}\rule{0.5\linewidth}{0.5pt}\end{center}

\paragraph{7. 实现绑定点}\label{para:implementation-binding-points}

\subparagraph{7.1 前端组件}\label{subpara:frontend-components}

\begin{itemize}
\tightlist
\item
  组件文件:\texttt{frontend/components/killchain/killchain-panel.tsx}
\item
  集成位置:\texttt{frontend/app/trace/page.tsx}
\item
  样式文件:\texttt{frontend/components/killchain/killchain-panel.css}(可选)
\end{itemize}

\subparagraph{7.2 API 客户端}\label{subpara:api-client}

\begin{itemize}
\tightlist
\item
  API 函数:\texttt{frontend/lib/api/analysis.ts}

  \begin{itemize}
  \tightlist
  \item
    \texttt{fetchKillChainByTaskId(taskId:\ string)}
  \item
    \texttt{fetchKillChainByUUID(kcUuid:\ string)}(可选)
  \end{itemize}
\end{itemize}

\subparagraph{7.3 报告生成}\label{subpara:report-generation}

\begin{itemize}
\tightlist
\item
  报告生成器:\texttt{frontend/lib/export/report-generator.ts}

  \begin{itemize}
  \tightlist
  \item
    在 \texttt{generateMarkdownReport()} 中添加 killchain 章节
  \end{itemize}
\end{itemize}

\begin{center}\rule{0.5\linewidth}{0.5pt}\end{center}

\paragraph{8. 测试验收标准}\label{para:test-acceptance-criteria}

\subparagraph{8.1 功能测试}\label{subpara:functional-tests}

\begin{enumerate}
\def\labelenumi{\arabic{enumi}.}
\item
  \textbf{基础版视图}

  \begin{itemize}
  \tightlist
  \item
    ✅ 可信度评分条颜色正确
  \item
    ✅ 战术时间线显示完整
  \item
    ✅ LLM 解释摘要显示前 3 句
  \end{itemize}
\item
  \textbf{完整版视图}

  \begin{itemize}
  \tightlist
  \item
    ✅ 展开/折叠功能正常
  \item
    ✅ 所有 segment 显示完整
  \item
    ✅ 所有 path 显示完整
  \item
    ✅ LLM 解释完整显示
  \end{itemize}
\item
  \textbf{图谱高亮}

  \begin{itemize}
  \tightlist
  \item
    ✅ 展开面板时自动高亮
  \item
    ✅ 点击路径高亮功能正常
  \item
    ✅ 收起面板时清除高亮
  \end{itemize}
\item
  \textbf{报告导出}

  \begin{itemize}
  \tightlist
  \item
    ✅ 导出的 Markdown 格式正确
  \item
    ✅ 包含所有必要信息
  \item
    ✅ 文件下载成功
  \end{itemize}
\end{enumerate}

\subparagraph{8.2 性能测试}\label{subpara:performance-tests}

\begin{enumerate}
\def\labelenumi{\arabic{enumi}.}
\item
  \textbf{渲染性能}

  \begin{itemize}
  \tightlist
  \item
    基础版渲染时间 \textless{} 100ms
  \item
    完整版展开时间 \textless{} 300ms
  \end{itemize}
\item
  \textbf{高亮性能}

  \begin{itemize}
  \tightlist
  \item
    高亮 100 条边不卡顿
  \item
    高亮操作响应时间 \textless{} 200ms
  \end{itemize}
\end{enumerate}

\subparagraph{8.3 兼容性测试}\label{subpara:compatibility-tests}

\begin{enumerate}
\def\labelenumi{\arabic{enumi}.}
\item
  \textbf{向后兼容}

  \begin{itemize}
  \tightlist
  \item
    不包含 killchain 的任务正常显示
  \item
    旧版本任务不报错
  \end{itemize}
\item
  \textbf{边界情况}

  \begin{itemize}
  \tightlist
  \item
    空segments处理
  \item
    超长explanation换行处理
  \item
    特殊字符转义
  \end{itemize}
\end{enumerate}
