\subsection{Filebeat 采集与 ECS 转换}\label{23-filebeat-collection-ecs-conversion}

\subsubsection{采集输入}\label{client-53-1-collection-input}

\paragraph{采集来源}\label{11-collection-source}

Filebeat 从以下宿主机挂载目录采集日志:

\begin{itemize}
\tightlist
\item
  \texttt{/var/log/host/auth.log}
\item
  \texttt{/var/log/host/syslog}
\item
  \texttt{/var/log/host/kern.log}
\end{itemize}

\paragraph{Filebeat 配置文件}\label{12-filebeat-config-file}

容器使用配置文件 \texttt{client/sensor/filebeat/filebeat-docker.yml},将 Filebeat 输出写入容器内的 \texttt{/tmp/filebeat-output/ecs\_logs.json} 文件。

\subsubsection{解析与转换规则}\label{2-parsing-conversion-rules}

\paragraph{两段式处理}\label{21-two-stage-processing}

Filebeat 采集链路由两个进程协同工作:

\begin{enumerate}
\def\labelenumi{\arabic{enumi}.}
\tightlist
\item
  \textbf{Filebeat}:采集日志并写入 \texttt{/tmp/filebeat-output/ecs\_logs.json}
\item
  \textbf{detector}:读取新增日志行,应用 Sigma 规则进行异常检测,并将结果发布到 RabbitMQ
\end{enumerate}

detector 实现位于 \texttt{client/sensor/filebeat/detector.py}。

\subsubsection{两段式处理流程}\label{211-two-stage-processing-flow}

\begin{figure}[htbp]
\centering
\includegraphics[width=\textwidth,height=0.5\textheight,keepaspectratio]{figures/client-53/client-53-01.pdf}
\caption{两段式处理流程}
\label{fig:client-53-01}
\end{figure}

\paragraph{ECS 关键字段补齐}\label{22-ecs-key-fields}

detector 对每条日志执行以下固定补齐:

\begin{itemize}
\tightlist
\item
  \texttt{ecs.version} 固定为 \texttt{9.2.0}
\item
  \texttt{event.ingested} 缺失时写入当前时间
\end{itemize}

为使三传感器数据在中心机侧汇聚到同一 \texttt{Host},detector 执行主机身份规范化:

\begin{itemize}
\tightlist
\item
  支持通过环境变量 \texttt{HOST\_NAME} 覆盖 \texttt{host.name}
\item
  优先使用环境变量 \texttt{HOST\_ID} 覆盖 \texttt{host.id};若 \texttt{HOST\_ID} 缺失,则按 \texttt{81-ECS字段规范.md} 回退生成 \texttt{host.id}:\texttt{h-} + sha1(host.name){[}:16{]}
\end{itemize}

当事件命中 Sigma 规则时,detector 将其标记为告警并补齐以下字段:

\begin{itemize}
\tightlist
\item
  \texttt{event.kind="alert"}
\item
  \texttt{event.category={[}"intrusion\_detection"{]}}
\item
  \texttt{event.type={[}"indicator"{]}}
\item
  \texttt{event.dataset="finding.raw.filebeat\_sigma"}
\item
  \texttt{rule.*}、\texttt{threat.*}、\texttt{custom.*} 等结构化字段
\end{itemize}

当日志来自 \texttt{auth.log} 且能够解析出 SSH 登录关键信息时,detector 还会输出 Telemetry(用于登录链路与横向移动分析):

\begin{itemize}
\tightlist
\item
  \texttt{event.kind="event"}
\item
  \texttt{event.dataset="hostlog.auth"}
\item
  \texttt{event.category={[}"authentication"{]}}
\item
  \texttt{event.action}:\texttt{user\_login} / \texttt{user\_logout} / \texttt{logon\_failed}
\item
  \texttt{event.outcome}:\texttt{success} / \texttt{failure}
\item
  \texttt{event.type}:\texttt{start} / \texttt{end} / \texttt{info}
\item
  \texttt{user.name}、\texttt{source.ip}:必须存在(若无法解析则不输出该 Telemetry,避免产生不符合 \texttt{81-ECS字段规范.md} 的无效事件)
\end{itemize}

\subsubsection{Sigma 规则检测流程}\label{221-sigma-detection-flow}

\begin{figure}[htbp]
\centering
\includegraphics[width=\textwidth,height=0.5\textheight,keepaspectratio]{figures/client-53/client-53-02.pdf}
\caption{Sigma 规则检测流程}
\label{fig:client-53-02}
\end{figure}

\subsubsection{Sigma 规则示例}\label{222-sigma-rule-example}

以下是一个 SSH 登录失败检测的 Sigma 规则示例:

\begin{verbatim}
title: SSH 登录失败
id: ssh-login-failed
description: 检测 SSH 登录失败尝试
status: experimental
author: ATA
date: 2026/01/14
logsource:
  service: auth
detection:
  keywords:
    - 'Failed password'
    - 'authentication failure'
  condition: keywords
level: low
tags:
  - attack.initial_access
  - attack.brute_force
fields:
  - user.name
  - source.ip
  - event.outcome
falsepositives:
  - 合法的登录失败
\end{verbatim}

\subsubsection{认证日志解析示例}\label{223-auth-log-parsing-example}

以下展示 auth.log 原始行到 ECS 事件的转换:

\textbf{原始日志}:

\begin{verbatim}
Jan 14 12:00:00 victim sshd[1234]: Failed password for root from 10.0.0.1 port 12345 ssh2
\end{verbatim}

\textbf{转换后的 ECS 事件}:

\begin{verbatim}
{
  "event": {
    "kind": "event",
    "dataset": "hostlog.auth",
    "category": ["authentication"],
    "action": "user_login",
    "outcome": "failure",
    "type": "info"
  },
  "user": {"name": "root"},
  "source": {"ip": "10.0.0.1", "port": 12345},
  "host": {"name": "client-01", "id": "h-1111111111111111"},
  "message": "Failed password for root from 10.0.0.1 port 12345 ssh2"
}
\end{verbatim}

\subsubsection{会话重建字段}\label{3-session-reconstruction-fields}

会话与进程实体标识由中心机入库阶段补齐:

\begin{itemize}
\tightlist
\item
  \texttt{session.id}:在 \texttt{event.dataset="hostlog.auth"} 的 Telemetry 中生成
\item
  \texttt{process.entity\_id}:在 \texttt{event.dataset="hostlog.process"} 的 Telemetry 中生成
\end{itemize}

具体规则参见 \texttt{../../80-规范/81-ECS字段规范.md}。

\subsubsection{队列投递}\label{client-53-4-queue-delivery}

Filebeat detector 将数据投递到 RabbitMQ:

\begin{itemize}
\tightlist
\item
  队列:\texttt{data.filebeat}
\item
  连接配置:\texttt{RABBITMQ\_URL}
\item
  队列名称:\texttt{RABBITMQ\_QUEUE}
\end{itemize}

拉取接口返回前会补齐稳定的 \texttt{event.id},规则参见 \texttt{../../80-规范/87-客户机与中心机接口.md}。

\subsubsection{故障处理}\label{client-53-5-error-handling}

\begin{enumerate}
\def\labelenumi{\arabic{enumi}.}
\tightlist
\item
  若 detector 未成功加载 Sigma 规则,则直接退出,避免产生无检测意义的数据流。
\item
  RabbitMQ 发布失败时,将重连并重试发布。
\item
  日志解析失败时跳过该行,继续处理后续日志,确保持续运行。
\end{enumerate}
