\subsection{报告导出}\label{subsec:report-export}

\subsubsection{导出内容结构}\label{subsec:export-content-structure}

报告导出功能服务于交付与验收场景,其导出产物、章节结构与字段规范均遵循固定规则。

\paragraph{导出产物}\label{para:export-artifacts}

前端导出产物为单个 Markdown 文件,文件命名规则如下:

\begin{itemize}
\tightlist
\item
  文件名格式:\texttt{attack-trace-report-\textless{}task\_id\textgreater{}.md}
\end{itemize}

其中 \texttt{\textless{}task\_id\textgreater{}} 代表溯源任务 ID(如 \texttt{trace-aaaaaaaa-bbbb-cccc-dddd-eeeeeeeeeeee})。

\paragraph{报告章节结构}\label{para:report-section-structure}

Markdown 报告遵循固定的章节结构:

\begin{figure}[htbp]
\centering
\includegraphics[width=\textwidth,height=0.5\textheight,keepaspectratio]{figures/frontend-76/frontend-76-01.pdf}
\caption{报告章节结构}
\label{fig:frontend-76-01}
\end{figure}

\textbf{章节清单:}

\begin{enumerate}
\def\labelenumi{\arabic{enumi}.}
\tightlist
\item
  报告元信息
\item
  输入与边界
\item
  告警与证据摘要
\item
  溯源关键路径
\item
  APT 相似度匹配结果
\item
  附录:原始数据(JSON)
\end{enumerate}

\paragraph{字段规范}\label{para:field-specifications}

报告字段规范遵循以下规则:

\begin{itemize}
\tightlist
\item
  ECS 字段:\texttt{../../80-规范/81-ECS字段规范.md}
\item
  图查询与任务接口:\texttt{../../80-规范/88-前端与中心机接口.md}
\item
  写回字段:\texttt{../../80-规范/85-溯源结果写回规范.md}
\end{itemize}

\subsubsection{数据来源}\label{subsec:data-sources}

报告导出依赖中心机后端接口提供数据,数据来源固定为以下 4 类请求。

\paragraph{数据来源总览}\label{para:data-sources-overview}

\begin{figure}[htbp]
\centering
\includegraphics[width=\textwidth,height=0.5\textheight,keepaspectratio]{figures/frontend-76/frontend-76-02.pdf}
\caption{数据来源总览}
\label{fig:frontend-76-02}
\end{figure}

\paragraph{报告数据映射表}\label{para:report-data-mapping-table}

{\def\LTcaptype{none} % do not increment counter
\small
\begin{longtable}[]{@{}p{0.15\textwidth}p{0.15\textwidth}p{0.35\textwidth}p{0.25\textwidth}@{}}
\toprule\noalign{}
报告章节 & 数据来源 & API 端点 & 核心字段 \\
\midrule\noalign{}
\endhead
\bottomrule\noalign{}
\endlastfoot
报告元信息 & 任务文档 & {\footnotesize\texttt{GET /api/v1/analysis/tasks/\{task\_id\}}} & {\footnotesize\texttt{task\_id}, \texttt{status}, \texttt{time\_window}} \\
溯源关键路径 & 图数据库 & {\footnotesize\texttt{POST /api/v1/graph/query}} & {\footnotesize\texttt{nodes{[}{]}}, \texttt{edges{[}{]}}} \\
告警摘要 & Finding 索引 & {\footnotesize\texttt{POST /api/v1/findings/search}} & {\footnotesize\texttt{tactic}, \texttt{technique}, \texttt{event.id}} \\
TTP 相似度 & 任务文档/LLM & {\footnotesize 任务文档或 \texttt{POST /api/v1/analysis/ttp-similarity}} & {\footnotesize\texttt{apt\_id}, \texttt{similarity}, \texttt{reasoning}} \\
\end{longtable}
}

\paragraph{任务信息}\label{para:task-information}

\begin{itemize}
\tightlist
\item
  请求方法:\texttt{GET}
\item
  请求路径:\texttt{/api/v1/analysis/tasks/\{task\_id\}}
\item
  接口用途:获取任务状态机字段、时间窗与目标节点 UID
\end{itemize}

\paragraph{溯源关键路径}\label{para:critical-trace-path}

\begin{itemize}
\tightlist
\item
  请求方法:\texttt{POST}
\item
  请求路径:\texttt{/api/v1/graph/query}
\item
  请求体字段:

  \begin{itemize}
  \tightlist
  \item
    \texttt{action="analysis\_edges\_by\_task"}
  \item
    \texttt{task\_id=\textless{}task\_id\textgreater{}}
  \item
    \texttt{only\_path=false}(默认值;前端按 \texttt{analysis.is\_path\_edge=true} 筛选关键路径边)
  \end{itemize}
\item
  接口用途:获取该任务写回的关键路径边与涉及节点
\end{itemize}

\begin{quote}
说明:后端响应固定包含 \texttt{nodes{[}{]}}(按边集合中的 UID 去重拉取),无需额外的 \texttt{include\_nodes} 参数;字段结构遵循 \texttt{../../80-规范/88-前端与中心机接口.md} 规范。
\end{quote}

\paragraph{Canonical Finding 摘要}\label{para:canonical-finding-summary}

\begin{itemize}
\tightlist
\item
  请求方法:\texttt{POST}
\item
  请求路径:\texttt{/api/v1/findings/search}
\item
  请求体字段:

  \begin{itemize}
  \tightlist
  \item
    \texttt{stage="canonical"}
  \item
    \texttt{start\_ts/end\_ts} 使用任务时间窗
  \item
    \texttt{host\_id} 指定为报告主机
  \end{itemize}
\item
  接口用途:获取时间窗内 Canonical Finding 的 tactic/technique 覆盖情况与证据引用
\end{itemize}

\paragraph{TTP 相似度}\label{para:ttp-similarity}

TTP 相似度结果的读取遵循以下规则:

\begin{enumerate}
\def\labelenumi{\arabic{enumi}.}
\tightlist
\item
  \textbf{优先}从任务文档 \texttt{task.result.ttp\_similarity.*} 读取(确保与任务执行时结果一致,避免重复计算);
\item
  若任务文档缺失该结果,则\textbf{回落}调用 \texttt{POST\ /api/v1/analysis/ttp-similarity} 重新计算(参数为 \texttt{host\_id\ +\ start\_ts/end\_ts})。
\end{enumerate}

\subsubsection{生成规则与可复现性}\label{subsec:generation-rules-reproducibility}

\paragraph{生成流程}\label{para:generation-flow}

\begin{figure}[htbp]
\centering
\includegraphics[width=\textwidth,height=0.5\textheight,keepaspectratio]{figures/frontend-76/frontend-76-03.pdf}
\caption{生成流程}
\label{fig:frontend-76-03}
\end{figure}

\paragraph{报告主机选择规则}\label{para:report-host-selection-rules}

报告主机 \texttt{host\_id} 的取值遵循以下规则:

\begin{enumerate}
\def\labelenumi{\arabic{enumi}.}
\tightlist
\item
  若 \texttt{task.target.node\_uid} 的 UID 中包含 \texttt{host.id}(如 \texttt{Host:host.id=...} 或 \texttt{File:host.id=...;file.path=...}),直接从 UID 解析 \texttt{host.id};
\item
  否则,从 \texttt{POST\ /api/v1/graph/query} 的 \texttt{nodes{[}{]}} 中定位 \texttt{uid\ ==\ task.target.node\_uid} 的目标节点,优先读取 \texttt{node.key{[}"host.id"{]}},其次读取 \texttt{node.props{[}"host.id"{]}};
\item
  若上述两项均无法获取 \texttt{host.id},导出流程终止并提示错误。
\end{enumerate}

\begin{quote}
说明:中心机后端在节点响应中将唯一键字段拆分为 \texttt{key} 与 \texttt{props} 两部分,部分节点的 \texttt{host.id} 可能出现在 \texttt{key} 中。为保证导出稳定性,必须按上述优先级读取。
\end{quote}

\paragraph{排序与格式化}\label{para:sorting-formatting}

为确保同一输入导出得到一致的关键内容,报告采用固定排序规则:

\begin{figure}[htbp]
\centering
\includegraphics[width=\textwidth,height=0.5\textheight,keepaspectratio]{figures/frontend-76/frontend-76-04.pdf}
\caption{排序与格式化规则}
\label{fig:frontend-76-04}
\end{figure}

\textbf{排序规则详表:}

{\def\LTcaptype{none} % do not increment counter
\begin{longtable}[]{@{}llll@{}}
\toprule\noalign{}
数据类型 & 主排序键 & 次排序键 & 第三排序键 \\
\midrule\noalign{}
\endhead
\bottomrule\noalign{}
\endlastfoot
Canonical Findings & \texttt{@timestamp} 升序 & \texttt{event.id} 升序 & - \\
图节点 & \texttt{uid} 升序 & - & - \\
图边 & \texttt{props.ts\_float} 升序 & \texttt{src\_uid} 升序 & \texttt{rtype} / \texttt{dst\_uid} 升序 \\
Tactics/Techniques & 字符串升序 & - & - \\
\end{longtable}
}

报告中不包含后端响应的 \texttt{server\_time} 字段,仅保留与输入相关的稳定字段。

\paragraph{证据引用呈现}\label{para:evidence-reference-presentation}

报告中每条关键路径边均需呈现以下字段:

\begin{figure}[htbp]
\centering
\includegraphics[width=\textwidth,height=0.5\textheight,keepaspectratio]{figures/frontend-76/frontend-76-05.pdf}
\caption{证据引用呈现规则}
\label{fig:frontend-76-05}
\end{figure}

\textbf{必呈字段清单:}

{\def\LTcaptype{none} % do not increment counter
\begin{longtable}[]{@{}lll@{}}
\toprule\noalign{}
字段类别 & 必呈字段 & 说明 \\
\midrule\noalign{}
\endhead
\bottomrule\noalign{}
\endlastfoot
基础字段 & \texttt{event.id} & 事件唯一标识 \\
& \texttt{event.dataset} & 数据集来源 \\
证据字段 & \texttt{custom.evidence.event\_ids{[}{]}} & 关联证据 ID 列表 \\
& \texttt{analysis.task\_id} & 所属溯源任务 \\
& \texttt{analysis.is\_path\_edge} & 是否关键路径边 \\
告警字段\textless br/(条件性) & \texttt{is\_alarm\ 为\ true} & 告警标识 \\
& \texttt{rule.*} & 规则信息 \\
& \texttt{threat.*} & 威胁情报 \\
& \texttt{custom.finding.*} & Finding 扩展字段 \\
\end{longtable}
}

这些字段支撑现场演示中"从告警到证据再到路径"的解释闭环。
