\subsection{检测与告警融合}\label{center-63-detection-fusion}

\subsubsection{检测触发机制}\label{center-63-trigger}

中心机的检测与融合模块\textbf{在每个 tick 内自动执行},无需手动开关。轮询服务完成 Step 1/2(拉取并入库)后,立即启动 Step 3(检测与融合)。

触发点位于:

\begin{itemize}
\tightlist
\item
  \texttt{backend/app/services/client\_poller.py}
\end{itemize}

\subsubsection{检测与融合入口}\label{center-63-entry}

检测与融合通过统一的入口函数完成:

\begin{itemize}
\tightlist
\item
  \texttt{backend/app/services/opensearch/analysis.py:run\_data\_analysis()}
\end{itemize}

该函数执行两段核心操作:

\begin{enumerate}
\def\labelenumi{\arabic{enumi}.}
\tightlist
\item
  \textbf{关联分析}:基于 Correlation Rules 实现跨事件关联(必要时自动创建或更新规则),产出高层攻击场景和关联告警;
\item
  \textbf{融合去重}:对 \texttt{raw-findings-*} 执行融合去重,生成 Canonical Findings 并写入 \texttt{canonical-findings-*} 索引。
\end{enumerate}

\begin{quote}
注:底层调用 OpenSearch 插件 API(Security Analytics / Correlation Rules)的细节属于工程实现,可能随版本调整;本文档仅约束入口、输入输出与审计口径。
\end{quote}

\subsubsection{手动触发(仅调试)}\label{center-63-manual-trigger}

开发与联调阶段允许手动调用 \texttt{run\_data\_analysis()} 单独执行一次,便于排障与回放;但靶场与验收环境仍以``tick 内自动触发''为准。

\subsection{Raw Finding 结构与写入}\label{center-63-raw-finding}

Raw Finding 来源于两个渠道:

\begin{enumerate}
\def\labelenumi{\arabic{enumi}.}
\tightlist
\item
  客户机侧直接产生的告警事件:Filebeat Sigma detector、Falco 或 Suricata(\texttt{event.kind="alert"});
\item
  中心机侧 OpenSearch Security Analytics 产生的 findings(经拉取并转换为 ECS 告警事件)。
\end{enumerate}

Raw Finding 的字段结构与最小必填字段遵循以下规范:

\begin{itemize}
\tightlist
\item
  \texttt{../../80-规范/83-告警数据规范.md}
\end{itemize}

\subsubsection{Security Analytics Finding 转换}\label{center-63-sa-conversion}

当告警来源为 OpenSearch Security Analytics 时,中心机按照既定规则将原始 finding 转换为 ECS 告警事件,并写入 \texttt{raw-findings-*} 索引:

\begin{itemize}
\tightlist
\item
  \texttt{event.kind="alert"}
\item
  \texttt{event.dataset="finding.raw.security\_analytics"}
\item
  \texttt{rule.id/rule.name} 从 detector 信息派生
\item
  \texttt{threat.*} 从 finding 的 ATT\&CK tags 派生(或从固定映射表派生)
\end{itemize}

实现位置:\texttt{backend/app/services/opensearch/analysis.py:\_convert\_security\_analytics\_finding\_to\_ecs()}。

\subsection{融合指纹规则}\label{center-63-fingerprint}

融合去重以固定时间窗口内的 Raw Finding 为输入,按照指纹规则聚合生成 Canonical Finding。

指纹字段与生成规则遵循以下规范:

\begin{itemize}
\tightlist
\item
  \texttt{../../80-规范/83-告警数据规范.md}
\end{itemize}

\subsubsection{时间桶参数}\label{center-63-time-bucket}

融合指纹采用固定的时间桶参数:

\begin{itemize}
\tightlist
\item
  \texttt{TIME\_WINDOW\_MINUTES\ =\ 3}
\end{itemize}

实现位置:\texttt{backend/app/services/opensearch/analysis.py:TIME\_WINDOW\_MINUTES}。

\subparagraph{时间桶示意图}\label{center-63-time-bucket-diagram}

\begin{figure}[htbp]
\centering
\includegraphics[width=\textwidth,height=0.5\textheight,keepaspectratio]{figures/center-63/center-63-01.pdf}
\caption{时间桶聚合示例}
\label{fig:center-63-01}
\end{figure}

\textbf{说明}:

\begin{itemize}
\tightlist
\item
  X 轴表示 Unix 时间戳(秒)
\item
  每个 3 分钟桶(180 秒)内的 Raw Findings 会被分配到同一个桶中
\item
  只有同一时间桶内且指纹相同的 Raw Findings 才会被融合
\item
  跨桶的 Raw Findings 即使内容完全相同也不会被融合
\end{itemize}

\subsubsection{指纹构造要素}\label{center-63-fingerprint-elements}

指纹 key 由以下要素构成:

\begin{itemize}
\tightlist
\item
  \texttt{threat.technique.id}
\item
  \texttt{host.id}
\item
  \texttt{entity\_id}(取值顺序:\texttt{process.entity\_id} → \texttt{destination.ip{[}/destination.domain{]}} → \texttt{file.hash.sha256})
\item
  \texttt{time\_bucket}
\end{itemize}

实现位置:\texttt{backend/app/services/opensearch/analysis.py:generate\_fingerprint()} 与 \texttt{fingerprint\_id\_from\_key()}。

\subsection{字段映射:Raw → Canonical}\label{center-63-field-mapping}

Raw Finding 转换为 Canonical Finding 时的关键字段映射关系:

\begin{figure}[htbp]
\centering
\includegraphics[width=\textwidth,height=0.5\textheight,keepaspectratio]{figures/center-63/center-63-02.pdf}
\caption{字段映射关系}
\label{fig:center-63-02}
\end{figure}

\textbf{说明}:

\begin{itemize}
\tightlist
\item
  \textbf{event.dataset}:从 \texttt{finding.raw.*} 转换为 \texttt{finding.canonical}
\item
  \textbf{provider 字段}:Raw Finding 中的单一 provider 扩展为数组 \texttt{provider.*},记录所有来源
\item
  \textbf{finding.fingerprint}:基于指纹要素生成唯一标识符
\item
  \textbf{custom.evidence.event\_ids}:收集所有被融合的 Raw Finding 的 \texttt{event.id}
\item
  \textbf{finding.dedup\_count}:记录被融合的 Raw Finding 数量
\end{itemize}

\subsection{Canonical 生成与覆盖规则}\label{center-63-canonical-rules}

Canonical Finding 的生成规则、\texttt{event.id} 生成规则以及覆盖行为遵循以下规范:

\begin{itemize}
\tightlist
\item
  \texttt{../../80-规范/83-告警数据规范.md}
\end{itemize}

\subsubsection{融合写入动作}\label{center-63-write-actions}

融合写入按以下步骤执行:

\begin{enumerate}
\def\labelenumi{\arabic{enumi}.}
\tightlist
\item
  从 \texttt{raw-findings-*} 拉取 Raw Findings;
\item
  按指纹 key 分组;
\item
  每组多条 Raw Findings 合并为一个 Canonical Finding;
\item
  每组单条 Raw Finding 直接规范化为 Canonical Finding;
\item
  批量写入 \texttt{canonical-findings-*} 索引并刷新。
\end{enumerate}

实现位置:\texttt{backend/app/services/opensearch/analysis.py:deduplicate\_findings()}。

\subsubsection{融合算法流程}\label{center-63-algorithm-flow}

\begin{figure}[htbp]
\centering
\includegraphics[width=\textwidth,height=0.5\textheight,keepaspectratio]{figures/center-63/center-63-03.pdf}
\caption{融合算法流程}
\label{fig:center-63-03}
\end{figure}

\textbf{流程说明}:

\begin{enumerate}
\def\labelenumi{\arabic{enumi}.}
\tightlist
\item
  \textbf{Raw Findings 聚合}:从 \texttt{raw-findings-*} 索引拉取时间窗口内的所有 Raw Findings
\item
  \textbf{时间桶分组}:按 3 分钟桶分组,确保只有同一时间桶内的 Raw Findings 会被融合
\item
  \textbf{指纹生成}:基于固定要素生成唯一指纹 key
\item
  \textbf{去重融合}:相同指纹的 Raw Findings 合并为一个 Canonical Finding
\item
  \textbf{Provider 信息合并}:将所有来源的 provider 信息合并到 \texttt{provider.*} 字段数组
\item
  \textbf{写入 OpenSearch}:批量写入 \texttt{canonical-findings-*} 索引
\end{enumerate}

\subsection{审计与可回溯性}\label{center-63-audit}

审计与可回溯性要求:

\begin{enumerate}
\def\labelenumi{\arabic{enumi}.}
\tightlist
\item
  Raw 与 Canonical Finding 必须包含证据引用 \texttt{custom.evidence.event\_ids};
\item
  Canonical Finding 的证据引用必须能回溯到 Telemetry 的 \texttt{event.id};
\item
  任意 Canonical Finding 必须能定位到其来源 providers 与融合指纹。
\end{enumerate}

字段规范详见:

\begin{itemize}
\tightlist
\item
  \texttt{../../80-规范/81-ECS字段规范.md}
\end{itemize}
