\subsection{TTP 相似度匹配}\label{sec:ttp-similarity-matching}

\subsubsection{输入与特征抽取}\label{subsec:input-feature-extraction}

TTP 相似度匹配以 Canonical Finding 作为输入样本,提取 ATT\&CK 的 tactic 与 technique 集合,计算与 CTI 中 intrusion-set(APT 组织)的相似度。

实现绑定点(以代码为准):

\begin{itemize}
\tightlist
\item
  入口服务:\texttt{backend/app/services/analyze/ttp\_similarity/service.py}
\item
  HTTP 接口:\texttt{backend/app/services/analyze/ttp\_similarity/router.py}
\end{itemize}

\paragraph{输入参数(固定)}\label{subsec:input-parameters}

HTTP 接口输入字段固定如下:

\begin{itemize}
\tightlist
\item
  \texttt{host\_id}:ECS \texttt{host.id}
\item
  \texttt{start\_ts}:ISO 8601 起始时间(包含边界)
\item
  \texttt{end\_ts}:ISO 8601 结束时间(包含边界)
\end{itemize}

当 \texttt{end\_ts\ \textless{}\ start\_ts} 时,接口返回 HTTP 400 错误。

\paragraph{Canonical Finding 拉取范围(固定)}\label{subsec:canonical-finding-range}

系统从 OpenSearch 的 Canonical Findings 索引中拉取样本,查询条件固定如下:

\begin{itemize}
\tightlist
\item
  \texttt{event.dataset\ ==\ "finding.canonical"}
\item
  \texttt{host.id\ ==\ host\_id}
\item
  \texttt{@timestamp} 在 \texttt{{[}start\_ts,\ end\_ts{]}}(包含边界)
\end{itemize}

实现绑定点:\texttt{backend/app/services/analyze/ttp\_similarity/service.py:fetch\_attack\_ttps\_from\_canonical\_findings()}。

\paragraph{Technique 提取与规范化(固定)}\label{subsec:technique-extraction-normalization}

每条 Canonical Finding 的 technique 提取规则固定如下:

\begin{enumerate}
\def\labelenumi{\arabic{enumi}.}
\tightlist
\item
  按固定顺序读取:优先读取嵌套字段 \texttt{threat.technique.id};
\item
  若嵌套字段不存在,则读取平铺字段 \texttt{threat.technique.id};
\item
  将 technique id 标准化为大写,满足正则表达式:\texttt{Tdddd} 或 \texttt{Tdddd.ddd};
\item
  过滤占位值:\texttt{UNKNOWN}、\texttt{TBD}、\texttt{T0000}、\texttt{T0000.xxx};
\item
  若 technique 为子技术(如 \texttt{T1055.012}),则展开为集合 \texttt{\{T1055.012,\ T1055\}}。
\end{enumerate}

实现绑定点:

\begin{itemize}
\tightlist
\item
  解析与过滤:\texttt{backend/app/services/analyze/ttp\_similarity/service.py:\_normalize\_technique\_id()}
\item
  子技术展开:\texttt{backend/app/services/analyze/ttp\_similarity/service.py:\_expand\_technique\_ids()}
\end{itemize}

\paragraph{Tactic 提取与规范化(固定)}\label{subsec:tactic-extraction-normalization}

每条 Canonical Finding 的 tactic 提取规则固定如下:

\begin{enumerate}
\def\labelenumi{\arabic{enumi}.}
\tightlist
\item
  按固定顺序读取:优先读取平铺字段 \texttt{threat.tactic.id};
\item
  若平铺字段不存在,则读取嵌套字段 \texttt{threat.tactic.id};
\item
  标准化为大写,满足正则表达式:\texttt{TAdddd};
\item
  过滤占位值:\texttt{UNKNOWN}、\texttt{TBD}、\texttt{TA0000}。
\end{enumerate}

实现绑定点:\texttt{backend/app/services/analyze/ttp\_similarity/service.py:\_normalize\_tactic\_id()}。

\subsubsection{CTI 数据与预处理}\label{subsec:cti-data-preprocessing}

\paragraph{CTI 文件路径解析(固定)}\label{subsec:cti-file-path-resolution}

系统加载本地 ATT\&CK Enterprise CTI(STIX bundle JSON),路径解析规则固定如下:

\begin{enumerate}
\def\labelenumi{\arabic{enumi}.}
\tightlist
\item
  读取环境变量 \texttt{ATTACK\_CTI\_PATH};
\item
  若 \texttt{ATTACK\_CTI\_PATH} 为空字符串,则使用默认路径:

  \begin{itemize}
  \tightlist
  \item
    \texttt{backend/app/services/analyze/ttp\_similarity/cti/enterprise-attack.json}
  \end{itemize}
\item
  若 \texttt{ATTACK\_CTI\_PATH} 为绝对路径,则直接使用该路径;
\item
  若 \texttt{ATTACK\_CTI\_PATH} 为相对路径,则按顺序尝试:

  \begin{itemize}
  \tightlist
  \item
    以当前工作目录为基准的相对路径;
  \item
    以 \texttt{backend} 根目录为基准的相对路径(若路径以 \texttt{backend/} 开头,则自动去除该前缀)。
  \end{itemize}
\item
  若最终路径不存在,则抛出 \texttt{FileNotFoundError},接口返回 HTTP 500 错误。
\end{enumerate}

实现绑定点:\texttt{backend/app/services/analyze/ttp\_similarity/service.py:get\_enterprise\_cti\_index()}。

\paragraph{CTI 索引构建(固定)}\label{subsec:cti-index-construction}

系统将 STIX bundle 构建为内存索引,索引内容固定如下:

\begin{enumerate}
\def\labelenumi{\arabic{enumi}.}
\tightlist
\item
  \texttt{intrusion-set}:建立 \texttt{intrusion\_set\_stix\_id\ -\textgreater{}\ (display\_id,\ name)} 映射;其中 \texttt{display\_id} 的取值规则为:若存在 ATT\&CK 外部编号(\texttt{Gdddd}),则使用该编号,否则使用 stix id;
\item
  \texttt{attack-pattern}:建立 \texttt{attack\_pattern\_stix\_id\ -\textgreater{}\ technique\_id} 映射;technique id 来自 \texttt{external\_references.source\_name="mitre-attack"} 的 \texttt{external\_id};
\item
  \texttt{x-mitre-tactic}:建立 tactic shortname(或 name)到 tactic id(\texttt{TAdddd})的映射;
\item
  \texttt{attack-pattern.kill\_chain\_phases}:派生 \texttt{technique\_id\ -\textgreater{}\ tactics} 映射;子技术与父技术共享该 tactic 映射;
\item
  \texttt{relationship(uses)}:派生 \texttt{intrusion\_set\_stix\_id\ -\textgreater{}\ techniques} 映射(并将子技术展开到父技术)。
\end{enumerate}

实现绑定点:\texttt{backend/app/services/analyze/ttp\_similarity/service.py:build\_enterprise\_cti\_index()}。

\paragraph{TF-IDF 权重(固定公式)}\label{subsec:tf-idf-weights}

系统将 intrusion-set 集合作为"文档集合"计算 IDF:

\begin{itemize}
\tightlist
\item
  设 intrusion-set 文档数为 \texttt{N};
\item
  对任意 technique id,设文档频次为 \texttt{df};
\item
  technique 的 IDF 定义为:\texttt{idf\ =\ ln((N\ +\ 1)\ /\ (df\ +\ 1))\ +\ 1.0};
\item
  tactic 的 IDF 使用相同公式(以 tactic 集合作为特征)。
\end{itemize}

并计算每个 intrusion-set 的 L2 范数:

\begin{itemize}
\tightlist
\item
  \texttt{group\_norm}:technique TF-IDF 向量的 L2 范数;
\item
  \texttt{group\_tactic\_norm}:tactic TF-IDF 向量的 L2 范数。
\end{itemize}

\subparagraph{向量化示意}\label{subsec:vectorization-illustration}

TTP 标签到 TF-IDF 向量的转换过程如图所示:

\begin{figure}[htbp]
\centering
\includegraphics[width=\textwidth,height=0.5\textheight,keepaspectratio]{figures/analysis-73/analysis-73-01.pdf}
\caption{TTP 向量化示意}
\label{fig:analysis-73-01}
\end{figure}

\subsubsection{相似度计算}\label{subsec:similarity-calculation}

\paragraph{算法流程}\label{subsec:algorithm-flow}

TTP 相似度匹配的完整处理流程如图所示:

\begin{figure}[htbp]
\centering
\includegraphics[width=\textwidth,height=0.5\textheight,keepaspectratio]{figures/analysis-73/analysis-73-02.pdf}
\caption{TTP 相似度匹配算法流程}
\label{fig:analysis-73-02}
\end{figure}

\paragraph{相似度向量(固定)}\label{subsec:similarity-vectors}

系统使用二值 TF(出现记 1,不出现记 0)的 TF-IDF 向量,并采用余弦相似度计算。

余弦分子 dot 的定义固定如下:

\begin{itemize}
\tightlist
\item
  仅对交集项求和:\texttt{dot\ =\ Σ\ (idf(t)\^{}2)}。
\end{itemize}

实现绑定点:\texttt{backend/app/services/analyze/ttp\_similarity/service.py:\_cosine\_from\_intersection\_weights()}。

\paragraph{综合分数(固定)}\label{subsec:composite-score}

每个 intrusion-set 的综合相似度分数由两部分组成:

\begin{enumerate}
\def\labelenumi{\arabic{enumi}.}
\tightlist
\item
  \texttt{tech\_score}:attack techniques 与 intrusion-set techniques 的余弦相似度;
\item
  \texttt{tactic\_score}:attack tactics 与 intrusion-set tactics 的余弦相似度。
\end{enumerate}

综合分数计算公式为:

\begin{itemize}
\tightlist
\item
  \texttt{score\ =\ (1\ -\ tactic\_weight)\ *\ tech\_score\ +\ tactic\_weight\ *\ tactic\_score}
\item
  其中 \texttt{tactic\_weight\ =\ 0.5}
\end{itemize}

当 \texttt{score\ \textless{}=\ 0.0} 时,该 intrusion-set 不进入候选集合。

实现绑定点:\texttt{backend/app/services/analyze/ttp\_similarity/service.py:rank\_similar\_intrusion\_sets()}。

\subsubsection{Top3 输出结构}\label{subsec:top3-output-structure}

\paragraph{TopK 与解释字段(固定)}\label{subsec:topk-explanation-fields}

输出规则固定如下:

\begin{itemize}
\tightlist
\item
  TopK:\texttt{top\_k\ =\ 3}
\item
  每个候选的解释字段个数:\texttt{explain\_top\_n\ =\ 5}
\end{itemize}

对于每个 TopK intrusion-set,系统输出以下内容:

\begin{enumerate}
\def\labelenumi{\arabic{enumi}.}
\tightlist
\item
  \texttt{intrusion\_set.id}:若 CTI 中存在 ATT\&CK 外部编号(如 \texttt{G0016}),则输出该编号,否则输出 intrusion-set 的 stix id;
\item
  \texttt{intrusion\_set.name}:组织名称;
\item
  \texttt{similarity\_score}:综合相似度;
\item
  \texttt{top\_techniques{[}{]}}:交集 technique 按 \texttt{technique\_idf} 降序排列的前 \texttt{explain\_top\_n} 个;
\item
  \texttt{top\_tactics{[}{]}}:交集 tactic 按 \texttt{tactic\_idf} 降序排列的前 \texttt{explain\_top\_n} 个。
\end{enumerate}

\paragraph{HTTP Response(固定字段)}\label{subsec:http-response-fields}

接口返回字段固定如下:

\begin{itemize}
\tightlist
\item
  \texttt{host\_id}
\item
  \texttt{start\_ts}
\item
  \texttt{end\_ts}
\item
  \texttt{attack\_tactics{[}{]}}(排序后输出)
\item
  \texttt{attack\_techniques{[}{]}}(过滤后输出)
\item
  \texttt{similar\_apts{[}{]}}(最多 3 项)
\end{itemize}

\subparagraph{Top3 输出示例}\label{subsec:top3-output-example}

\begin{verbatim}
{
  "host_id": "i-0abc123def456",
  "start_ts": "2024-01-01T00:00:00Z",
  "end_ts": "2024-01-01T23:59:59Z",
  "attack_tactics": [
    {"id": "TA0001", "name": "Initial Access"},
    {"id": "TA0006", "name": "Credential Access"}
  ],
  "attack_techniques": [
    {"id": "T1078", "name": "Valid Accounts"},
    {"id": "T1110", "name": "Brute Force"},
    {"id": "T1021", "name": "Remote Services"}
  ],
  "similar_apts": [
    {
      "intrusion_set": {
        "id": "G0016",
        "name": "APT28"
      },
      "similarity_score": 0.85,
      "top_techniques": [
        {"technique_id": "T1078", "technique_idf": 2.1},
        {"technique_id": "T1110", "technique_idf": 1.8}
      ],
      "top_tactics": [
        {"tactic_id": "TA0001", "tactic_idf": 1.2},
        {"tactic_id": "TA0006", "tactic_idf": 0.9}
      ]
    },
    {
      "intrusion_set": {
        "id": "G0015",
        "name": "APT29"
      },
      "similarity_score": 0.72,
      "top_techniques": [
        {"technique_id": "T1078", "technique_idf": 2.1}
      ],
      "top_tactics": [
        {"tactic_id": "TA0001", "tactic_idf": 1.2}
      ]
    },
    {
      "intrusion_set": {
        "id": "G0035",
        "name": "Lazarus Group"
      },
      "similarity_score": 0.65,
      "top_techniques": [
        {"technique_id": "T1021", "technique_idf": 1.5}
      ],
      "top_tactics": []
    }
  ]
}
\end{verbatim}

字段结构的权威口径见:

\begin{itemize}
\tightlist
\item
  \texttt{../../80-规范/88-前端与中心机接口.md}
\end{itemize}
