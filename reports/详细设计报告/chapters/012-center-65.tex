\subsection{图谱回标与边属性}\label{center-65-writeback}

\subsubsection{告警映射与回标规则}\label{center-65-alert-mapping}

\paragraph{输入边界(固定)}\label{center-65-input-boundary}

Neo4j 入图时仅接受以下两类 ECS 文档(严格执行):

\begin{enumerate}
\def\labelenumi{\arabic{enumi}.}
\tightlist
\item
  Telemetry:\texttt{event.kind="event"}
\item
  Canonical Finding:\texttt{event.kind="alert"} 且 \texttt{event.dataset="finding.canonical"}
\end{enumerate}

该边界定义属于图谱口径规范,详见:

\begin{itemize}
\tightlist
\item
  \texttt{../../80-规范/84-Neo4j实体图谱规范.md} 0.1
\end{itemize}

\paragraph{Canonical Finding → 告警边(固定)}\label{center-65-canonical-edge}

\begin{figure}[htbp]
\centering
\includegraphics[width=\textwidth,height=0.5\textheight,keepaspectratio]{figures/center-65/center-65-01.pdf}
\caption{告警边映射流程}
\label{fig:center-65-01}
\end{figure}

Canonical Finding 在入图阶段映射为一组关系边,其边属性需满足以下固定规则:

\begin{enumerate}
\def\labelenumi{\arabic{enumi}.}
\tightlist
\item
  每条边都携带 \texttt{event.id}、\texttt{event.kind}、\texttt{event.dataset}、\texttt{ts\_float}、\texttt{custom.evidence.event\_ids{[}{]}};
\item
  当输入为 Canonical Finding 时,每条边额外携带 \texttt{is\_alarm=true} 标记;
\item
  Canonical Finding 的解释字段写入边属性,用于前端展示与溯源:

  \begin{itemize}
  \tightlist
  \item
    \texttt{rule.*}
  \item
    \texttt{threat.*}
  \item
    \texttt{event.severity}
  \item
    \texttt{custom.finding.*}
  \end{itemize}
\end{enumerate}

代码实现位置:

\begin{itemize}
\tightlist
\item
  入图转换:\texttt{backend/app/services/neo4j/ecs\_ingest.py:ecs\_event\_to\_graph()}
\item
  告警边标记:\texttt{backend/app/services/neo4j/ecs\_ingest.py} 中 \texttt{edge\_props.setdefault("is\_alarm",\ True)}
\end{itemize}

\paragraph{证据引用缺失的处理(固定)}\label{center-65-evidence-missing}

Canonical Finding 必须携带 \texttt{custom.evidence.event\_ids{[}{]}},否则该事件不会入图,视为``无效告警''。

代码实现位置:

\begin{itemize}
\tightlist
\item
  \texttt{backend/app/services/neo4j/ecs\_ingest.py} 中对 \texttt{event\_kind=="alert"} 的 \texttt{evidence\_ids} 校验
\end{itemize}

\subsubsection{写回字段集合与覆盖规则}\label{center-65-writeback-fields}

\paragraph{属性写回流程}\label{center-65-writeback-flow}

\begin{figure}[htbp]
\centering
\includegraphics[width=\textwidth,height=0.5\textheight,keepaspectratio]{figures/center-65/center-65-02.pdf}
\caption{属性写回流程}
\label{fig:center-65-02}
\end{figure}

\paragraph{字段映射关系}\label{center-65-field-mapping}

{\def\LTcaptype{none} % do not increment counter
\footnotesize
\begin{longtable}[]{@{}p{0.25\textwidth}p{0.12\textwidth}p{0.53\textwidth}@{}}
\toprule\noalign{}
算法生成字段 & Neo4j 边属性 & 说明 \\
\midrule\noalign{}
\endhead
\bottomrule\noalign{}
\endlastfoot
\texttt{analysis.is\_path\_edge} & \texttt{analysis.is\_path\_edge} & 是否为关键路径边 \\
\texttt{analysis.score} & \texttt{analysis.score} & 边的权重评分 \\
\texttt{analysis.remarks} & \texttt{analysis.remarks} & 备注信息 \\
\texttt{task\_id} & \texttt{analysis.task\_id} & 关联的溯源任务ID \\
\end{longtable}
}

\paragraph{工程实现绑定点}\label{center-65-implementation}

写回字段集合与覆盖规则属于规范文档的权威口径,本文件不再重复字段表,仅说明工程实现绑定点:

\begin{itemize}
\tightlist
\item
  权威写回字段口径:\texttt{../../80-规范/85-溯源结果写回规范.md}
\item
  写回入口:\texttt{backend/app/services/neo4j/db.py:write\_analysis\_results()}
\item
  覆盖写实现:\texttt{backend/app/services/neo4j/db.py:\_write\_analysis\_result\_tx()}(先清空字段后写入)
\end{itemize}

溯源任务在算法侧生成写回字段的入口:

\begin{itemize}
\tightlist
\item
  \texttt{backend/app/services/analyze/trace.py:compute\_trace()}(生成 \texttt{analysis.*} 字段)
\item
  \texttt{backend/app/services/analyze/pipeline.py:run\_analysis\_task()}(调用写回)
\end{itemize}

\subsubsection{前端读取与过滤规则}\label{center-65-frontend-read}

前端读取``告警边''与``溯源写回边''时使用不同的查询动作:

\begin{enumerate}
\def\labelenumi{\arabic{enumi}.}
\tightlist
\item
  告警边展示:\texttt{POST\ /api/v1/graph/query},\texttt{action="alarm\_edges"}
\item
  任务写回结果:\texttt{POST\ /api/v1/graph/query},\texttt{action="analysis\_edges\_by\_task"},并传入 \texttt{task\_id}
\end{enumerate}

当 \texttt{action="analysis\_edges\_by\_task"} 时:

\begin{itemize}
\tightlist
\item
  \texttt{only\_path=true}:只返回 \texttt{analysis.is\_path\_edge=true} 的关键路径边
\item
  \texttt{only\_path=false}:返回该任务写回的全部边(包含非关键路径边)
\end{itemize}

接口字段的权威定义详见:

\begin{itemize}
\tightlist
\item
  \texttt{../../80-规范/88-前端与中心机接口.md}
\end{itemize}
