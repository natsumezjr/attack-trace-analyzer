\subsection{候选路径构造与评分}\label{candidate-path-construction}

\subsubsection{输入子图定义}\label{candidate-path-input}

溯源算法从 Neo4j 获取两类输入参数:

\begin{itemize}
\tightlist
\item
  目标节点 UID:\texttt{task.target.node\_uid}
\item
  时间窗:\texttt{{[}task.window.start\_ts,\ task.window.end\_ts{]}}
\end{itemize}

算法基于目标节点的 1-hop incident edges 提取边集合。

\paragraph{算法总览}\label{candidate-path-algorithm-overview}

\begin{figure}[htbp]
\centering
\includegraphics[width=\textwidth,height=0.5\textheight,keepaspectratio]{figures/analysis-71/analysis-71-01.pdf}
\caption{算法总览}
\label{fig:analysis-71-01}
\end{figure}

\paragraph{输入配置示例}\label{candidate-path-input-example}

\begin{verbatim}
{
  "target": {
    "node_uid": "process-12345"
  },
  "window": {
    "start_ts": 1704067200.0,
    "end_ts": 1704153600.0
  }
}
\end{verbatim}

代码实现位于 \texttt{backend/app/services/analyze/trace.py} 的 \texttt{compute\_trace()} 函数。

\subsubsection{时间窗过滤(固定)}\label{candidate-path-time-window}

输入边集合按 \texttt{ts\_float} 字段过滤至时间窗内(闭区间):

\begin{itemize}
\tightlist
\item
  \texttt{task.window.start\_ts\ \textless{}=\ edge.ts\_float\ \textless{}=\ task.window.end\_ts}
\end{itemize}

时间窗单位与转换由 Neo4j 模块统一处理,算法层仅对 \texttt{ts\_float} 数值进行比较。

\begin{figure}[htbp]
\centering
\includegraphics[width=\textwidth,height=0.5\textheight,keepaspectratio]{figures/analysis-71/analysis-71-02.pdf}
\caption{时间窗过滤}
\label{fig:analysis-71-02}
\end{figure}

\subsubsection{锚点与段划分(固定)}\label{candidate-path-segments}

当前实现不执行攻击阶段划分,边集合按事件时间排序后直接用于数据写回与前端展示。

\subsubsection{风险评分与排序(固定)}\label{candidate-path-scoring}

风险评分采用固定的兜底策略:

\begin{enumerate}
\def\labelenumi{\arabic{enumi}.}
\tightlist
\item
  若边属性包含 \texttt{event.severity},取其数值作为风险分;
\item
  若 \texttt{is\_alarm=true},风险分取 \texttt{50.0};
\item
  其他情况风险分取 \texttt{0.0}。
\end{enumerate}

\paragraph{评分规则可视化}\label{candidate-path-scoring-visual}

\begin{figure}[htbp]
\centering
\includegraphics[width=\textwidth,height=0.5\textheight,keepaspectratio]{figures/analysis-71/analysis-71-03.pdf}
\caption{评分规则}
\label{fig:analysis-71-03}
\end{figure}

\paragraph{评分示例}\label{candidate-path-scoring-example}

{\def\LTcaptype{none} % do not increment counter
\footnotesize
\begin{longtable}[]{@{}p{0.20\textwidth}p{0.12\textwidth}p{0.28\textwidth}p{0.30\textwidth}@{}}
\toprule\noalign{}
场景 & event.severity & is\_alarm & 风险评分 \\
\midrule\noalign{}
\endhead
\bottomrule\noalign{}
\endlastfoot
高危事件 & 90.0 & false & \textbf{90.0} \\
告警边(无 severity) & null & true & \textbf{50.0} \\
普通事件 & null & false & \textbf{0.0} \\
中危事件 & 60.0 & true & \textbf{60.0} \\
\end{longtable}
}

\paragraph{代码示例}\label{candidate-path-code-example}

\begin{verbatim}
def _derive_risk_score(edge: Edge) -> float:
    """计算边的风险评分"""
    if hasattr(edge, 'event') and edge.event.severity is not None:
        return float(edge.event.severity)
    if edge.is_alarm:
        return 50.0
    return 0.0
\end{verbatim}

代码实现位于 \texttt{backend/app/services/analyze/trace.py} 的 \texttt{\_derive\_risk\_score()} 函数。

\subsubsection{同分处理规则}\label{candidate-path-tiebreaking}

同分处理采用固定的排序策略:

\begin{enumerate}
\def\labelenumi{\arabic{enumi}.}
\tightlist
\item
  首先按 \texttt{ts\_float} 时间升序排序;
\item
  时间相同则按边的 \texttt{src\_uid} 与 \texttt{dst\_uid} 字典序排序,确保输出稳定。
\end{enumerate}

\paragraph{排序流程}\label{candidate-path-sorting-flow}

\begin{figure}[htbp]
\centering
\includegraphics[width=\textwidth,height=0.5\textheight,keepaspectratio]{figures/analysis-71/analysis-71-04.pdf}
\caption{排序流程}
\label{fig:analysis-71-04}
\end{figure}

\paragraph{排序示例}\label{candidate-path-sorting-example}

假设边集合如下:

\begin{verbatim}
edges = [
    Edge(ts_float=1000, src_uid="B", dst_uid="D"),
    Edge(ts_float=900, src_uid="A", dst_uid="C"),
    Edge(ts_float=1000, src_uid="A", dst_uid="B"),
    Edge(ts_float=1000, src_uid="A", dst_uid="C"),
]
\end{verbatim}

排序结果如下:

{\def\LTcaptype{none} % do not increment counter
\footnotesize
\begin{longtable}[]{@{}p{0.18\textwidth}p{0.10\textwidth}p{0.20\textwidth}p{0.20\textwidth}p{0.22\textwidth}@{}}
\toprule\noalign{}
顺序 & ts\_float & src\_uid & dst\_uid & 排序依据 \\
\midrule\noalign{}
\endhead
\bottomrule\noalign{}
\endlastfoot
1 & 900 & A & C & 时间戳最小 \\
2 & 1000 & A & B & 时间相同,src\_uid 最小 \\
3 & 1000 & A & C & 时间相同,src\_uid 相同,dst\_uid 排序 \\
4 & 1000 & B & D & 时间相同,src\_uid 较大 \\
\end{longtable}
}
